%
% Thesis template conforming to Williams College rules.
% Thanks to Ben Wood '08 and other contributors.
%

\documentclass[twoside]{report}
\usepackage[top=1.0in, bottom=1in, left=1.5in, right=1in, includehead]{geometry}
\pagestyle{headings}

\usepackage{setspace}

%% Special math fonts and symbols
\usepackage{amssymb}
\usepackage{amsfonts}
\usepackage{amsmath}
\usepackage{amsthm}
%% Rotate tables and figures
\usepackage{rotating}
%% Used for TODO items
\usepackage{color}
%% used for code listings.
\usepackage{float}
%% Used to replace LaTeX's ugly emptyset with diameter, which looks nicer.
\usepackage{wasysym}
%% Nicely formatted algorithms.
\usepackage{algorithmicx}
\usepackage[chapter]{algorithm}
\usepackage{algpseudocode}
%% Nicely formatted listings.
\usepackage{listings}
%% More kinds of arrow with stuff
\usepackage{empheq}
\usepackage{multicol}
\usepackage{subfigure}
%% Used for citations
\usepackage{cite}

%%%%%%%%%%%%%%%%%%%%%%%%%%%%%
%% Thesis body %%
%%%%%%%%%%%%%%%%%%%%%%%%%%%%%

\begin{document}

%%%%%%%%%%%%%%%%%%%%%%%%%%%%%
%% Title page %%
%%%%%%%%%%%%%%%%%%%%%%%%%%%%%
\begin{titlepage}
  $\;$
  \vskip1.5in
  \onehalfspacing
  \begin{center}
    {\LARGE
      Predictive Pod Autoscaling in the Kubernetes Container Cluster Manager
    }
    \large
    \vskip.25in
    by\\
    Matt McNaughton\\
    \vskip.125in
    Professor Jeannie Albrecht, Advisor\\
    \singlespacing
    \vskip.5in
    \small
    A thesis submitted in partial fulfillment\\
    of the requirements for the\\
    Degree of Bachelor of Arts with Honors\\
    in Computer Science\\
    \vskip.5in
    Williams College\\
    Williamstown, Massachusetts\\
    \vskip.5in
    \today
    \vskip.5in
    {\Huge \textbf{DRAFT}}
  \end{center}
\end{titlepage}
%%%%%%%%%%%%%%%%%%%%%%%%%%%%%

\tableofcontents
% \listoffigures
% \listoftables

\onehalfspacing

\chapter*{Abstract}

\chapter*{Acknowledgments}

%%%%%%%% Chapters %%%%%%%%%%%%

%%%%% Introduction

\chapter{Introduction}

Over the past few decades, an explosion in the need for computing power and
memory, and the existence of cheap, interconnected computers, has
driven a significant increase in the feasibility and benefits of distributed
systems.\cite[pg. 1]{distributed-systems-principles-and-paradigms}
First, we consider the origin of distributed systems as a field of computer
science. Before the availability of cheap, powerful microprocessors and reliable,
efficient local-area networks (LANs), computational tasks could only be
performed on a singular computer.\cite[pg.
1]{distributed-systems-principles-and-paradigms} If a task was too
computationally expensive, the only solution was to run it on a larger, more
powerful supercomputer. However, as cheap microprocessors
made computers more abundant, and LANs allowed computers to communicate quickly, a new
model of performing resource intensive computation, entitled \textbf{distributed
systems} arose. In the distributed systems model a collection of individual
computers function as a single computer to solve a given computing task.\cite[pg.
2]{distributed-systems-principles-and-paradigms}

Second, we consider the ever-growing interest in unlocking and implementing the
benefits of distributed systems. A number of forces drove, and continue to drive,
increased interest in distributed systems over
the past decade. The first, and most obvious, factor is the Internet, itself a
distributed system. As more people connected to the Internet, through computers,
mobile phones, and tablets, an increasing number of human interactions became
computerized. Consumption, communication, research, and many more are all
possible through the Internet. Naturally, large amounts of computing resources
are needed to store the data, and perform the computation, related to these
interactions. Closely coupled with this trend is the rise of ``Big Data''.
In 2013, the digital universe contained 4.4 zettabytes of data.\footnote{A
  zettabyte equals $10^{21}$ bytes, which equals 1 billion
terabytes.}\cite{the-digital-universe-of-opportunities} Naturally, without
multiple computers working together it would be impossible to store and process
this incredible volume of danger. Overall, it is nearly impossible to do
anything in modern society without interacting with a distributed system and
creating new digital data. Driving a car, trading a stock, visiting a doctor,
checking an email, and even playing a simple video game, are all activities that
distributed systems facilitate and improve.\cite[pg.
4]{distributed-systems-concepts-and-design} As life becomes more and more
computerized, and as the volume of data humans generate and hope to process
grows and grows, distributed systems will only become increasingly more
important. Furthermore, research into distributed systems makes it possible to
continue to unlock, and make available to the general public,
the incredible power of networked computers. As distributed systems supplying
massive computational powers become more
accessible, both because of decreased cost and increased ease of use, we can
address an ever increasing number of challenging, important problems computationally.

There are a number of different models of computing tasks requiring high levels
of computing resources, including supercomputing, cluster computing, and grid
computing. In this thesis, we focus on cluster computing. Specifically, we focus on the
cluster manager, an integral component of cluster computing. Cluster managers
are responsible for abstracting all of the management details of the distinct
nodes in the cluster, and instead present a single mass of computing resources
to the user on which they can run jobs on the application. In other words,
a cluster manager ``admits, schedules, starts, restarts, and monitors the full
range of applications'' run on the cluster.\cite[pg. 1]{borg} There are a
variety of different cluster managers, the most important of which will be
discussed in the background chapter, each pursuing different objectives. This
thesis will ultimately focus on \textbf{Kubernetes}, an open-source cluster
manager from Google.\cite{k8s-website}.

Cluster managers seek to accomplish a number of different goals, and as a
result, multiple metrics indicate success. For example, Microsoft's Autopilot is
predominantly concerned with application uptime, and thus success is measured
with respect to reliability and downtime.\cite[pg. 1]{autopilot}
Alternatively, a number of cluster managers measure themselves based on
efficient resource usage.\cite[pg. 7]{borg} Essentially, efficient cluster
resource utilization relates to the percent of cluster resources which are
actually being used. For example, one such measurement of this metric, cluster
compaction, examines how many computers could be removed from the cluster, while
still comfortably running the cluster's current application load.\cite[pg.
5]{evaluating-job-packing-in-warehouse-scale-computing} This metric is
particularly important, because the more efficient the cluster management is at
utilizing resources, the less resources cost, and the more accessible cluster
computing becomes to the general public. A final important
cluster management metric is quality of service. Quality of service measures the
ability of an application to function at a specified level, despite ever-changing
external factors. Again, this metric is particularly important, because
increasing the robustness of applications running on cluster managers means
these applications can be trusted with increasingly important tasks. Cluster
managers predominantly differ with respect to which of these metrics optimize
for, and the process by which this optimization occurs.

\section{Goals}

This thesis is most concerned with maximizing the efficient resource utilization
and quality of service metrics with respect to the Kubernetes cluster manager.
As such, this thesis pursues three goals:

\begin{enumerate}
  \item Given an application running on a Kubernetes cluster, we seek to
    determine a method which ensures quality of
    service stays consistently high regardless of external factors. While it is
    difficult to make guarantees regarding quality of service, it is possible to
    ensure each application has, and is utilizing, the resources it needs to
    function property. Given the cluster manager grants the application the
    resources it needs to function given the current external load,
    the cluster manager has done all it can to ensure application quality of
    service.
  \item Given a certain number of applications running on a Kubernetes cluster,
    we seek to determine a method which ensures the cluster is
    as small as possible, while still comfortably
    supporting the application's current, and future, resource needs.
  \item Given Kubernetes is an open-source project, we seek to implement, test, and
    evaluate a proposed enactment of the previous two goals.
    Thus, the methods we pursue will
    in part be dictated by the current structure and implementation of
    Kubernetes. Tests be conducted using the Google Compute
    Engine\cite{google-compute-engine} and both
    simulated and real Kubernetes user data.
\end{enumerate}

\section{Contributions}

This thesis presents our given contributions to Kubernetes. Kubernetes seeks to
ensure high application quality of service and efficient use of computing
resources, and our contributions look to further its ability to accomplish these
goals. As such, we present not only new methodology, but also new, working
implementations, with accompanying evaluation. We demonstrate the effectiveness
of our modifications in comparison to the non-modified Kubernetes using both
simulated and real-world datasets. Finally, we
discuss the experiences of making these modifications on Kubernetes, as well as
the avenues for future improvements with respect to Kubernetes and cluster managers in
general.

\section{Contents}

@TODO - This section can not be written until the thesis is completed.

\chapter{Background}

\section{Resource Intensive Computing Paradigms}

As was briefly mentioned in the introduction, a number of different paradigms
exist for undertaking computing tasks too resource intensive for a singular
computer. They are discussed in detail below:

\begin{enumerate}
  \item \underline{Supercomputing}: The supercomputing model responds to
    increased demands for computing resources by increasing the specs of the
    computer far beyond the range of the traditional commodity computer.
    While supercomputers are able to avoid the majority of the complications the
    introduction of networks introduce, most prominently reliability and
    security, there are naturally limits on the power of supercomputers.
    Importantly, constructing supercomputers is extremely expensive, and thus
    their computing power is not available to the general public. Furthermore,
    it is difficult to scale a supercomputer should the need arise. Finally,
    supercomputers offer a single point of failure, meaning they are not
    particularly robust to error. These limitations have decreased the usage of
    supercomputers to provide the mass of computing power needed in the ``Big
    Data'' era.

  \item \underline{Cluster Computing}: Cluster computing is defined as utilizing
    ``a collection of similar workstations of PCs, losely connected by means of
    a high-speed local-area network [where] each node runs the same operating
    system.''\cite[pg. 17-18]{distributed-systems-principles-and-paradigms}
    Cluster computing can provide a mass of computing power similar to
    that contained in a supercomputer. Cluster computing also offers many
    advantages over the single supercomputer. First, and perhaps most
    importantly, they are much more cost-efficient. Second, clusters are easy to
    scale by simply adding new nodes. Finally, cluster computing is much more fault
    tolerant, as a singular failing commodity computer will simply be removed
    from the cluster. Cluster computing is used in the
    implementation of what is colloquially referred to as \textit{Cloud
    computing}, in which large amounts of computing resources are offered on a
    per-usage basis.\cite[pg. 13]{distributed-systems-concepts-and-design}
    Cloud computing, as implemented by Amazon Web
    Services,\cite{amazon-web-services} Microsoft Azure,\cite{microsoft-azure}
    and Google Compute Engine,\cite{google-compute-engine} continue to
    revolutionize the development and deployment of computing applications, as
    developers gain access to cheap, easily accessible, quickly scalable
    computing power.

  \item \underline{Grid Computing}: Grid computing is similar to concept in
    cluster computing, except it foregoes the requirement that all computers
    within the grid be homogeneous. As such, the grid computing model accounts
    for a large degree of heterogeneity with respect to network membership,
    operating system, hardware, and more.\cite[pg.
    18]{distributed-systems-principles-and-paradigms} While grid computing
    systems weaken the homogeneity requirements imposed by cluster computing,
    the resulting heterogeneity introduces significant complexity.

\end{enumerate}

Ultimately, because of simplicity, cost, and scalability, cluster computing is
the most prominent resource intensive computing paradigm. Thus, cluster
computing, and the accompanying cluster manager, is the focus of this
thesis.

\section{Cluster Management Paradigms}

\subsection{Borg}

\subsection{Omega}

\subsection{Mesos}

\subsection{YARN}

\subsection{Kubernetes}

\section{Properties of Cluster Managers}

\subsection{Efficiency of Resource Utilization}

\subsection{Quality of Service}

\section{Summary}

\chapter{Appendix}

\section{Technologies Underlying Kubernetes}

\subsection{Containerization}

\subsubsection{Virtualization}

\subsubsection{Containerization}

\subsubsection{Docker}

%%%%%%%% References %%%%%%%%%%
\bibliographystyle{acm}
\bibliography{thesis}
%%%%%%%% End References %%%%%%

\end{document}
