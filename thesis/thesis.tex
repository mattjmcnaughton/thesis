%
% Thesis template conforming to Williams College rules.
% Thanks to Ben Wood '08 and other contributors.
%

\documentclass[twoside]{report}
\usepackage[top=1.0in, bottom=1in, left=1.5in, right=1in, includehead]{geometry}
\pagestyle{headings}

\usepackage{setspace}

%% Special math fonts and symbols
\usepackage{amssymb}
\usepackage{amsfonts}
\usepackage{amsmath}
\usepackage{amsthm}
%% Rotate tables and figures
\usepackage{rotating}
%% Used for TODO items
\usepackage{color}
%% used for code listings.
\usepackage{float}
%% Used to replace LaTeX's ugly emptyset with diameter, which looks nicer.
\usepackage{wasysym}
%% Nicely formatted algorithms.
\usepackage{algorithmicx}
\usepackage[chapter]{algorithm}
\usepackage{algpseudocode}
%% Nicely formatted listings.
\usepackage{listings}
%% More kinds of arrow with stuff
\usepackage{empheq}
\usepackage{multicol}
\usepackage{subfigure}
%% Used for citations
\usepackage{cite}

%%%%%%%%%%%%%%%%%%%%%%%%%%%%%
%% Thesis body %%
%%%%%%%%%%%%%%%%%%%%%%%%%%%%%

\begin{document}

%%%%%%%%%%%%%%%%%%%%%%%%%%%%%
%% Title page %%
%%%%%%%%%%%%%%%%%%%%%%%%%%%%%
\begin{titlepage}
  $\;$
  \vskip1.5in
  \onehalfspacing
  \begin{center}
    {\LARGE
      Predictive Pod Autoscaling in the Kubernetes Container Cluster Manager
    }
    \large
    \vskip.25in
    by\\
    Matt McNaughton\\
    \vskip.125in
    Professor Jeannie Albrecht, Advisor\\
    \singlespacing
    \vskip.5in
    \small
    A thesis submitted in partial fulfillment\\
    of the requirements for the\\
    Degree of Bachelor of Arts with Honors\\
    in Computer Science\\
    \vskip.5in
    Williams College\\
    Williamstown, Massachusetts\\
    \vskip.5in
    \today
    \vskip.5in
    {\Huge \textbf{DRAFT}}
  \end{center}
\end{titlepage}
%%%%%%%%%%%%%%%%%%%%%%%%%%%%%

\tableofcontents
% \listoffigures
% \listoftables

\onehalfspacing

\chapter*{Abstract}

\chapter*{Acknowledgments}

%%%%%%%% Chapters %%%%%%%%%%%%

%%%%% Introduction

\chapter{Introduction}

Over the past few decades, an explosion in the need for computing resources,
and the existence of cheap, interconnected computers, has
driven a significant increase in the feasibility and benefits of distributed
systems.\cite[pg. 1]{distributed-systems-principles-and-paradigms}

First, we consider the origin of distributed systems as a field of computer
science. Before the availability of cheap, powerful microprocessors and reliable,
efficient local-area networks (LANs), computational tasks could only be
performed on a singular computer.\cite[pg.
1]{distributed-systems-principles-and-paradigms} If a task was too
computationally expensive for a commodity PC, the only solution was to run
it on a larger, more powerful supercomputer. However, as cheap microprocessors
increased computers' availability, and LANs fostered quick inter-computer
communication, a new
model of performing resource intensive computation, distributed
systems, arose. In the distributed systems model, a collection of individual
computers function as a single computer to solve a given computational task.\cite[pg.
2]{distributed-systems-principles-and-paradigms}

Second, we consider the ever-growing interest in unlocking and implementing the
benefits of distributed systems. A number of forces drove, and continue to drive,
increased interest in distributed systems over
the past decade. The first, and most obvious, factor is the Internet.
As more people connected to the Internet, through computers,
mobile phones, and tablets, an increasing number of human interactions became
computerized. Consumption, communication, research, and more all
became possible on the Internet. Naturally, large amounts of computing resources
were needed to store the data, and perform the computational tasks, related to these
interactions. Closely coupled with this trend is the rise of ``Big Data''.
In 2013, the digital universe contained 4.4 zettabytes of data.\footnote{A
  zettabyte equals $10^{21}$ bytes, which equals 1 billion
terabytes.}\cite{the-digital-universe-of-opportunities} Naturally, without
multiple computers working together it would be impossible to store and process
this incredible volume of data. Today, it is nearly impossible to do
anything in modern society without interacting with a distributed system and
creating new digital data. Driving a car, trading a stock, visiting a doctor,
checking an email, and even playing a simple video game, are all activities that
distributed systems facilitate and improve.\cite[pg.
4]{distributed-systems-concepts-and-design} As life becomes more
computerized, and as the volume of data humans generate and hope to process
grows, distributed systems will only increase in importance.
Furthermore, research into distributed systems makes it possible to
continue to unlock, and make available to the general public,
the incredible power of networked, cooperating computers. As the distributed systems
supplying massive computational power become more
accessible, both because of decreased cost and increased ease of use and
reliability, we can
computationally address an ever increasing number of challenging, important problems.

There are a number of different models for computing tasks requiring high levels
of computing resources, including supercomputing, cluster computing, and grid
computing. In this thesis, we focus on cluster computing. Cluster computing
groups together similar commodity PCs on the same LAN to offer a singular mass
of computing resources. Specifically, we focus on the
cluster manager, an integral component of cluster computing. Cluster managers
are responsible for abstracting all of the management details of the distinct
nodes in the cluster, and instead presenting a single mass of computing resources
on which the user can run jobs or applications. In other words,
a cluster manager ``admits, schedules, starts, restarts, and monitors the full
range of applications'' on the cluster.\cite[pg. 1]{borg} There are a
variety of different cluster managers, the most important of which will be
discussed in the background chapter, each pursuing different objectives. This
thesis will ultimately focus on Kubernetes, an open-source cluster
manager from Google.\cite{k8s-website}

Cluster managers seek to accomplish a number of different goals, and as a
result, multiple metrics indicate success. For example, Microsoft's Autopilot is
predominantly concerned with application uptime, and thus success is measured
with respect to reliability and downtime.\cite[pg. 1]{autopilot}
Alternatively, a number of cluster managers measure themselves based on
efficient resource utilization (ERU).\cite[pg. 7]{borg} Essentially, efficient
resource utilization relates to the percent of cluster resources which are
actually being used. One such measurement of this goal, cluster
compaction, examines how many computers could be removed from the cluster, while
still comfortably running the cluster's current application load.\cite[pg.
5]{evaluating-job-packing-in-warehouse-scale-computing} This metric is
particularly important, because the more efficient the cluster management is at
utilizing resources, the less clusters cost, and the more accessible cluster
computing becomes to the general public. A final important
cluster management metric is quality of service (QOS). Quality of service measures the
ability of an application to function at a specified
performance level, despite ever-changing
external factors. Again, this metric is particularly important because
increasing the robustness of applications run on cluster managers means
these applications can be trusted with increasingly important tasks. Cluster
managers predominantly differ with respect to which metrics they optimize
for, and the process by which this optimization occurs.

\section{Goals}

This thesis is most concerned with maximizing the efficient resource utilization
(ERU) and quality of service (QOS) metrics with respect to the Kubernetes cluster manager.
As such, this thesis pursues three goals:

\begin{enumerate}
  \item Given an application running on a Kubernetes cluster, we seek to
    determine a method which ensures quality of
    service stays consistently high regardless of external factors. While it is
    difficult to make guarantees regarding quality of service, because
    application performance is dependent on a number of uncontrollable, varying
    external factors, it is possible to
    ensure each application has, and is utilizing, the resources it needs to
    function property. Given the cluster manager grants the application the
    resources it needs to function given the current external load,
    the cluster manager has done all it can to ensure a high quality of service.
  \item A simplistic solution to the first goal of ensuring a high application
    quality of service is to just give each application many more resources than
    it requests. Yet, this overallocation is inefficient and costly. Thus, our
    methods for ensuring a high quality of service must also ensure the
    maintenance, or improvement, of the efficient resource utilization metric.
    Thus, we add an additional goal: given a certain number of applications
    running on a Kubernetes cluster,
    we seek to determine a method which ensures the cluster is
    as small as possible, while still comfortably
    supporting the application's current, and future, resource needs.
  \item Given Kubernetes is an open-source project, we seek to implement, test, and
    evaluate a proposed enactment of the previous two goals.
    Thus, the methods we pursue will
    in part be dictated by the current structure and implementation of
    Kubernetes. Tests will be conducted using the Google Compute
    Engine\cite{google-compute-engine} on both
    simulated and real Kubernetes user data. The eventual goal is for this
    thesis' improvements to be merged into the production version of Kubernetes
    used to run 1000s of applications at Google everyday.
\end{enumerate}

\section{Contributions}

This thesis presents our given contributions to Kubernetes. Kubernetes seeks to
ensure high application quality of service and efficient resource utilization,
and our contributions look to further its ability to accomplish these
goals. As such, we present not only new methodology, but also new, working
implementations with the accompanying evaluation. We demonstrate the effectiveness
of our modifications in comparison to the non-modified Kubernetes using both
simulated and real-world datasets. Finally, we
discuss the experiences of making these modifications to Kubernetes, as well as
avenues for future improvements with respect to Kubernetes and cluster managers in
general.

\section{Contents}

@TODO - This section can not be written until the thesis is completed.

\chapter{Background}

\section{Resource Intensive Computing Paradigms}

As was briefly mentioned in the introduction, a number of different paradigms
exist for undertaking computing tasks too resource intensive for a single
computer. They are discussed in detail below:

\begin{enumerate}
  \item \underline{Supercomputing}: The supercomputing model responds to
    increased demands for computing resources by increasing the technical
    specifications of the computer far beyond the range of the
    traditional commodity PC.
    While supercomputers are able to avoid the majority of the complications
    resulting from the introduction of networks, most prominently reliability and
    security, there are naturally limits on the power of supercomputers.
    Importantly, constructing supercomputers is extremely expensive, and thus
    their computing power is not available to the general public. Furthermore,
    it is difficult to scale a supercomputer should the need arise. Finally,
    supercomputers offer a single point of failure, meaning they are not
    particularly robust to error. These limitations have decreased the usage of
    supercomputers to provide the mass of computing power needed in the ``Big
    Data'' era.

  \item \underline{Cluster Computing}: Cluster computing is defined as utilizing
    ``a collection of similar workstations of PCs, losely connected by means of
    a high-speed local-area network [where] each node runs the same operating
    system.''\cite[pg. 17-18]{distributed-systems-principles-and-paradigms}
    Cluster computing can provide a mass of computing power similar to
    that contained in a supercomputer. Cluster computing also offers many
    advantages over the single supercomputer. First, and perhaps most
    importantly, they are much more cost-efficient, and thus much more
    accessible. Second, clusters are easy to
    scale by simply adding new commodity PCs as nodes.
    Finally, cluster computing is much more fault
    tolerant, as a single failing commodity computer will simply be removed
    from the cluster. Cluster computing is used in the
    implementation of what is colloquially referred to as \textit{Cloud
    computing}, in which large amounts of computing resources are offered on a
    per-usage basis.\cite[pg. 13]{distributed-systems-concepts-and-design}
    Cloud computing, as implemented by Amazon Web
    Services,\cite{amazon-web-services} Microsoft Azure,\cite{microsoft-azure}
    and Google Compute Engine,\cite{google-compute-engine} continue to
    revolutionize the development and deployment of computing applications, as
    developers gain access to cheap, easily accessible, and quickly scalable
    computing power.

  \item \underline{Grid Computing}: Grid computing is similar to concept in
    cluster computing, except it foregoes the requirement that all computers
    within the grid be relatively homogeneous. As such, the grid computing model accounts
    for a large degree of heterogeneity with respect to network membership,
    operating system, hardware, and more.\cite[pg.
    18]{distributed-systems-principles-and-paradigms} While grid computing
    systems lack of homogeneity requirements increase flexibility,
    the resulting heterogeneity introduces significant complexity.

\end{enumerate}

Ultimately, because of simplicity, cost, and scalability, cluster computing is
the most prominent resource intensive computing paradigm. Thus, cluster
computing, and the accompanying cluster manager, is the focus of this
thesis.

\section{Cluster Management Paradigms}

As was briefly mentioned in the introduction, cluster managers are responsible
for admitting, scheduling, running, maintaining, and monitoring all applications
and jobs a user wishes to run on the cluster. Naturally, cluster managers are
extremely diverse, both in the types of applications and jobs they are most
suited to running, and the method in which they seek execute their duties.
At the most basic level, there are two types of workload that may be submitted
to a cluster manager: production and batch. Production tasks are long-running
with strict performance requirements and penalties to downtime. Batch tasks are
more flexible in their ability to handle short-term performance variance. In the
context of a large company like Google, a production task would be serving a
large website like Gmail, which must be continuously accessible with low-latency
and little downtime. A batch task would be analysing advertising analytics data
with MapReduce, which can fail or slow without significant external
costs.\cite[pg. 1]{borg} The type of tasks a cluster management system
predominantly seeks to run dictate the cluster manager's implementation details.

One important decision in the implementation of a cluster manager is the manner
by which the cluster manager will assign a job resources. Two predominant
methods exist. In the first, jobs request resources from a cluster manager and
the cluster manager schedules the jobs on the cluster as efficiently as
possible. In the second, the cluster manager offers resources to cluster
computing frameworks. The ultimate determined method of assigning a job
resources effects the type of applications and distributed computing frameworks
runnable on the cluster manager and the efficiency with which these applications
and frameworks run.

A final distinction is licensing and availability of the cluster manager's code.
Because entities need cluster managers only to process and store massive amounts
of data and human-computer interaction, mainly large
corporations develop and utilize cluster managers. Often these cluster
managers are kept within the confines of the corporation, or only explained by a
brief paper or conference talk, with little source code available. In more
unique cases, the company will open-source the source code, allowing anyone to
view, modify, and run the cluster manager. Such open-sourcing presents a unique opportunity
for researchers wishing to experiment with cluster managers, yet without the
resources to create their own from scratch. In rarer instances, a
fully-developed cluster manager will originate from academic research. In unique
scenarios, a large corporation will use this cluster manager and the full code will
be open-sourced. The availability of source code directly impacts the
feasibility of pursuing experiments with an already existing cluster management
system.

Naturally, cluster managers can vary in multiple additional ways. However,
the previous three differences recognize the most important distinctions in this
thesis' context. Given this understanding, we can know begin to examine specific
cluster management implementations and defend our choice of Kubernetes as the
cluster manager on which we will ask and answer our research question.

\subsection{Borg}

\subsection{Omega}

\subsection{Mesos}

\subsection{YARN}

\subsection{Kubernetes}


\section{Auto-scaling Paradigms}

\subsection{Threshold-based Rule Policies}

\subsection{Time-series Analysis}

\subsection{Control-Theory}

\subsubsection{Model Predictive Control}

\section{Summary}

\chapter{Appendix}

\section{Technologies Underlying Kubernetes}

\subsection{Containerization}

\subsubsection{Virtualization}

\subsubsection{Containerization}

\subsubsection{Docker}

%%%%%%%% References %%%%%%%%%%
\bibliographystyle{acm}
\bibliography{thesis}
%%%%%%%% End References %%%%%%

\end{document}
