%
% Thesis template conforming to Williams College rules.
% Thanks to Ben Wood '08 and other contributors.
%

\documentclass[twoside]{report}
\usepackage[top=1.0in, bottom=1in, left=1.5in, right=1in, includehead]{geometry}
\pagestyle{headings}

\usepackage{setspace}

%% Special math fonts and symbols
\usepackage{amssymb}
\usepackage{amsfonts}
\usepackage{amsmath}
\usepackage{amsthm}
%% Rotate tables and figures
\usepackage{rotating}
%% Used for TODO items
\usepackage{color}
%% used for code listings.
\usepackage{float}
%% Used to replace LaTeX's ugly emptyset with diameter, which looks nicer.
\usepackage{wasysym}
%% Nicely formatted algorithms.
\usepackage{algorithmicx}
\usepackage[chapter]{algorithm}
\usepackage{algpseudocode}
%% Nicely formatted listings.
\usepackage{listings}
%% More kinds of arrow with stuff
\usepackage{empheq}
\usepackage{multicol}
\usepackage{subfigure}
%% Used for citations
\usepackage{cite}

%%%%%%%%%%%%%%%%%%%%%%%%%%%%%
%% Thesis body %%
%%%%%%%%%%%%%%%%%%%%%%%%%%%%%

\begin{document}

%%%%%%%%%%%%%%%%%%%%%%%%%%%%%
%% Title page %%
%%%%%%%%%%%%%%%%%%%%%%%%%%%%%
\begin{titlepage}
  $\;$
  \vskip1.5in
  \onehalfspacing
  \begin{center}
    {\LARGE
      Predictive Pod Autoscaling in the Kubernetes Container Cluster Manager
    }
    \large
    \vskip.25in
    by\\
    Matt McNaughton\\
    \vskip.125in
    Professor Jeannie Albrecht, Advisor\\
    \singlespacing
    \vskip.5in
    \small
    A thesis submitted in partial fulfillment\\
    of the requirements for the\\
    Degree of Bachelor of Arts with Honors\\
    in Computer Science\\
    \vskip.5in
    Williams College\\
    Williamstown, Massachusetts\\
    \vskip.5in
    \today
    \vskip.5in
    {\Huge \textbf{DRAFT}}
  \end{center}
\end{titlepage}
%%%%%%%%%%%%%%%%%%%%%%%%%%%%%

\tableofcontents
% \listoffigures
% \listoftables

\onehalfspacing

\chapter*{Abstract}

\chapter*{Acknowledgments}

%%%%%%%% Chapters %%%%%%%%%%%%

%%%%% Introduction

\chapter*{Introduction}

\section{Motivation}

The Kubernetes cluster manager\cite{k8s-website} controls all aspects of a
cluster through admitting, running, and restarting applications, monitoring and
displaying application health, and maintaining the underlying machines composing
the cluster. Currently the Kubernetes cluster manager implements autoscaling,
ensuring applications scale to meet varying external
demands.\cite{k8s-horizontal-pod-autoscaler-proposal} We seek to
improve autoscaling to decrease the amount of time
replicated applications must handle too much or too little demand, increasing
the efficiency and responsiveness of Kubernetes.

\section{Process}

\subsection{Implementation}

Kubernetes is uniquely both completely open source under the Apache License, and
in production use at Google.\cite{google-container-engine} Ultimately, the goal
is for the Kubernetes project to accept this work into their master distribution.
Thus, the norms and
requirements of the Kubernetes community, as well as the current implementation
of Kubernetes, will influence the implementation of our autoscaling
modifications. The implementation steps are as follows:

\begin{itemize}
  \item Propose autoscaling modifications to the Kubernetes community.
  \item Make regularly scheduled pull requests to gradually introduce proposed
    changes.
    \begin{itemize}
      \item Importantly, Kubernetes is a extremely active project, and it will
        be important to continuously merge our changes to decrease the cost on
        ourselves and on the project maintainers of combining two divergent
        branches.
    \end{itemize}
  \item Evaluate proposed changes to determine their validity and performance.
  \item Make necessary changes and extensions.
\end{itemize}

\subsection{Data Collection}

\section{Goals}

%%%%%% Background

\chapter*{Background}

\section{Cluster Management}

\subsection{History and Motivation}

\subsection{Alternatives to Kubernetes}

\subsubsection{Borg}

\subsubsection{Omega}

\subsubsection{Others}

\subsection{Kubernetes}

\section{Containerization}

\section{Prediction Models}

%%%%%%%% References %%%%%%%%%%
\bibliographystyle{acm}
\bibliography{thesis}
%%%%%%%% End References %%%%%%

\end{document}
