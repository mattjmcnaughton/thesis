As was briefly mentioned in the introduction, cluster managers are responsible
for admitting, scheduling, running, maintaining, and monitoring all applications
and jobs a user wishes to run on the cluster. Naturally, cluster managers are
extremely diverse, both in the types of applications and jobs they are most
suited to running, and the method in which they seek execute their duties.
At the most basic level, there are two types of workload that may be submitted
to a cluster manager: production and batch. Production tasks are long-running
with strict performance requirements and penalties to downtime. Batch tasks are
more flexible in their ability to handle short-term performance variance. In the
context of a large company like Google, a production task would be serving a
large website like Gmail, which must be continuously accessible with low-latency
and little downtime. A batch task would be analysing advertising analytics data
with MapReduce, which can fail or slow without significant external
costs.\cite[pg. 1]{borg} The type of tasks a cluster management system
predominantly seeks to run dictate the cluster manager's implementation details.

One important decision in the implementation of a cluster manager is the manner
by which the cluster manager schedules jobs.\footnote{Scheduling jobs on
machines simply equates to assigning jobs to resources on a machine.}  There exist three
different methods of scheduling: monolithic, two-level, and
shared state. With monolithic scheduling, a single algorithm
is responsible for taking the resource requests of all jobs and assigning them
to the proper machine. With two-level scheduling, the cluster manager simply
offers resources, which can then be accepted or rejected by the distributed
computing frameworks.\footnote{Distributed computing frameworks are frameworks
built to function over multiple machines: Apache Hadoop, Apache Spark\dots}
Finally, with shared state scheduling, multiple different algorithms concurrently work to
schedule jobs on the cluster.\cite[pg. 1]{omega} Naturally, all
of these methods have positives and negatives. While monolithic scheduling is
simple to initially, a single-threaded monolithic scheduler does not allow
nuanced processing of jobs. Attempts to add this nuance can create an incredibly
complicated algorithm that is difficult to extend.\cite[pg. 353-354]{omega}
While two-level scheduling is lightweight, simple, and offers advantages with respect to
data locality, it is not effective for long-running,
production jobs. Finally, while shared state scheduling removes the scheduler as
both a computational and complexity bottleneck, yet must take steps to guarantee
global properties of the cluster.\cite[pg. 363]{omega}
The ultimate determined method of assigning a job
resources effects the type of applications and distributed computing frameworks
runnable on the cluster manager and the efficiency with which these applications
and frameworks run.

A final distinction is licensing and availability of the cluster manager's code.
Because entities need cluster managers only to process and store massive amounts
of data and human-computer interaction, mainly large
corporations develop and utilize cluster managers. Often these cluster
managers are kept within the confines of the corporation, or only explained by a
brief paper or conference talk, with little source code available. In more
unique cases, the company will open-source the source code, allowing anyone to
view, modify, and run the cluster manager. Such open-sourcing presents a unique opportunity
for researchers wishing to experiment with cluster managers, yet without the
resources to create their own from scratch. In rarer instances, a
fully-developed cluster manager will originate from academic research. In unique
scenarios, a large corporation will use this cluster manager and the full code will
be open-sourced. The availability of source code directly impacts the
feasibility of pursuing experiments with an already existing cluster management
system.

Naturally, cluster managers can vary in multiple additional ways. However,
the previous three differences recognize the most important distinctions in this
thesis' context. Given this understanding, we can know begin to examine specific
cluster management implementations and defend our choice of Kubernetes as the
cluster manager on which we will ask and answer our research question.

\subsection{Borg}

While just recently described to the public in a 2015 paper, the
\textit{Borg} cluster manager from Google has been in use for over a
decade.\cite[pg. 14]{borg} Borg incorporates
many different objectives, seeking to abstract resource management and failure
handling, maintain high availability, and efficiently utilize resources on the
cluster.\cite[pg. 1]{borg} It is responsible for managing the hundreds of
thousands of batch and production jobs run everyday at Google. Additionally,
Borg utilizes a monolithic scheduler, as jobs specify the
resources they need, and a single Borg scheduler decides whether to admit and
schedule said jobs. Finally, Borg is not open-source. In fact, it was not even
publicly announced or described until 2015, despite running at Google for over a
decade. However, Kubernetes, Google's open-source cluster manager, incoporates
much of the work done to implement and improve Borg.


\subsection{Omega}

Also originating at Google, Omega is a cluster manager seen as
an extension of Borg \cite{omega}. While retaining the mission and goals of Borg,
Omega differs in implementation. Specifically, it considers a new method of
assigning jobs the resources they need on the cluster by implementing
shared state, instead of monolithic, scheduling. As previously mentioned,
shared state scheduling allows multiple scheduling algorithms to work
in parallel to assign tasks to the resources they request. Research on Omega shows
this parallel scheduling implementation both eases the complexity of adding new
scheduling behaviors and offers competitive scaling performance.
Additionally, in Omega, all schedulers are aware of the entire
state of the cluster, meaning that a job's resource allocation can be
varied after the job begins to execute. This flexibility offers significant
performance improvements. Like Borg, Omega is not
open-source. However, like Borg, much of the work done on Omega is incorporated
into Kubernetes.


%%% Cluster manager table.

\begin{table}[]
\centering
\caption{Overview of Cluster Management
Paradigms.}\label{table:cluster-management-paradigms-comparison-table}
\begin{tabular}{|l|l|l|l|}
\hline
                    & \textbf{Job Type} & \textbf{Scheduling Model} &
                      \textbf{Open Source} \\ \hline
\textbf{Borg \cite{borg}}       & Both              & Monolithic                & No
\\ \hline
\textbf{Omega \cite{omega}}      & Both              & Shared state              & No
\\ \hline
\textbf{Mesos \cite{mesos}}      & Batch             & Two-level                 & Yes
\\ \hline
\textbf{YARN \cite{yarn}}       & Batch             & Monolithic                & Yes
\\ \hline
\textbf{Kubernetes \cite{k8s-website}} & Production        & Monolithic                & Yes
\\ \hline
\end{tabular}
\end{table}



\subsection{Mesos}

Lest we think all cluster managers originate at Google, we now examine Apache
Mesos, a cluster manager originating at University of California, Berkeley
\cite{mesos}. Mesos is considerably more lightweight than either Borg or Omega.
We can think of Mesos as functioning like the kernal of an operating system
(i.e, the Linux kernel) while a system like Borg is like an entire Linux
distribution (i.e, Red Hat, Ubuntu, etc\ldots)
It does not seek to monitor the health of jobs or provide user-interfaces for viewing the
current state of a job, leaving those tasks to the distributed computing
framework. Additionally, Mesos predominantly focuses on quick-running,
high-volume batch jobs, and is not particularly suited to long-running,
high-availability production jobs. In part, Mesos'
job scheduling implementation dictates the singularity of the job types
Mesos efficiently processes. Mesos utilizes two-level scheduling, in which the
cluster management system simply offers, not assigns, available
resources to the distributed computing framework. Predominantly, this decision
is made to ensure data locality. \footnote{Data locality is a measure
of if the task has the necessary data on its machine, or if it must make a
costly request across the network for the data. Mesos works to ensure data
locality by allowing multiple frameworks to function on the same machines, and
thus share data, and also for frameworks to dictate their own resources, such
that they can work to ensure data locality. If running a large number of data
processing tasks with extremely high volumes of data, data locality can be
essential to efficient cluster operation.} Finally, Mesos is interesting in
that it is entirely open-source, yet still in use at some of the largest tech
companies such as AirBNB and Twitter.


\subsection{YARN}

We now briefly discuss Apache YARN. YARN, an acronym for Yet Another
Resource Negotiator, is a cluster manager initially built for use with Apache
Hadoop \cite{yarn}. However, it is now possible to use YARN with a variety
of distributed computing frameworks. Unsurprisingly given YARN's initial use
case, YARN is predominantly used for running batch jobs.\footnote{At the time of
the publication, YARN was just beginning to be used for production jobs;
however, its main focus from creation has been batch processing.}
Like Mesos, YARN aims to support data-locality,
again a predominant advantage for batch processing. YARN
schedules resources by allowing distributed computing framework application
masters to request resources from a single resource master.
The application master can then assign these resources to specific tasks at its
leisure. As a single resource master is assigning all of these resources, YARN is a
monolithic scheduler. Similar to Mesos, YARN is both entirely open-source and in use
at major corporations like Yahoo.



\subsection{Kubernetes}

Finally, we arrive at the cluster manager that is the focus of this thesis:
Kubernetes. Kubernetes also originates at Google, although it is open source.
Kubernetes is also the most
recent of the cluster managers we consider in this background chapter.\footnote{Kubernetes
became public in the summer of 2014.} The recent
explosion in popularity of containerization\footnote{We discuss both the motivations and
technology behind containerization in the Appendix. Briefly, containerization is
packing everything an application needs to run into a single \textit{container}
and then running that container on any desired computer.}
heavily impacts the development and implementation of Kubernetes. Specifically,
Kubernetes is seen as part of a new paradigm of developing applications
through the use of microservices.\footnote{Again, we will
discuss microservices in greater detail
in the Appendix. Essentially, microservices are the division of applications
into small, easily scalable services which communicate with each other across
the network.} Kubernetes predominantly focuses on effectively running service
jobs, which require high-availability and potentially varying amounts of resources.
Additionally, Kubernetes currently utilizes a simple
monolithic scheduler, although
there are plans to grow the scheduler in the future \cite{k8s-design-overview}.
Finally, Kubernetes is an open source project, yet is also used in production at
Google, and available to the public through the Google Container
Engine \cite{google-container-engine}.

We choose Kubernetes as the cluster manager on which to conduct our experiments
for a number of reasons. First, unlike many of the aforementioned cluster
managers, Kubernetes focuses on service jobs. Services jobs
have particularly stringent requirements for availability and are also the most
likely to have varying resource needs. Both of these conditions are closely linked
with the previously stated goals of this thesis, and Kubernetes is the cluster
manager that stands to benefit the most if we achieve our goals. Additionally,
Kubernetes is the only open source cluster manager focusing on long-running
services. Working with a open source cluster manager allows us to benefit from
the previous work of others, as well as expand the potential benefits of any
successful work.

