We now briefly discuss Apache YARN. YARN, an acronym for Yet Another
Resource Negotiator, is a cluster manager initially built for use with Apache
Hadoop.\cite[pg. 1]{yarn} However, it is now possible to use YARN with a variety
of distributed computing frameworks. Unsurprisingly given YARN's initial use
case, YARN is predominantly used for running batch jobs.\footnote{At the time of
the publication, YARN was just beginning to be used for production jobs;
however, its main focus from creation has been batch processing.\cite[pg.
11]{yarn}} Like Mesos, YARN aims to support data-locality,
again a predominant advantage for batch processing.\cite[pg. 3]{yarn} YARN
schedules resources by allowing distributed computing framework application
masters to request resources from a single resource master.\cite[pg. 5]{yarn}
The application master can then assign these resources to specific tasks at its
leisure. As a single resource master is assigning all of these resources, YARN is a
monolithic scheduler. Similar to Mesos, YARN is both entirely open-source and in use
at major corporations like Yahoo.\cite[pg. 9]{yarn}

