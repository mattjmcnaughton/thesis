While just recently described to the public in a 2015 paper, the
\textit{Borg} cluster manager from Google has been in use for over a
decade.\cite[pg. 14]{borg} Borg incorporates
many different objectives, seeking to abstract resource management and failure
handling, maintain high availability, and efficiently utilize resources on the
cluster.\cite[pg. 1]{borg} It is responsible for managing the hundreds of
thousands of batch and production jobs run everyday at Google. Additionally,
Borg utilizes a monolithic scheduler, as jobs specify the
resources they need, and a single Borg scheduler decides whether to admit and
schedule said jobs. Finally, Borg is not open-source. In fact, it was not even
publicly announced or described until 2015, despite running at Google for over a
decade. However, Kubernetes, Google's open-source cluster manager, incoporates
much of the work done to implement and improve Borg.
