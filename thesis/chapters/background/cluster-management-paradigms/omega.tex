Also originating at Google, Omega is a cluster manager seen as
an extension of Borg. While retaining the mission and goals of Borg,
Omega differs in implementation. Specifically, it considers a new method of
assigning jobs the resources they need on the cluster by implementing
shared state, instead of monolithic, scheduling. As mentioned in the general
overview, shared state scheduling allows multiple scheduling algorithms to work
in parallel to assign tasks the resources they request. Research on Omega shows
this parallel scheduling implementation both eases the complexity of adding new
scheduling behaviors and offers competitive scaling performance.\cite[pg.
358-359]{omega} Additionally, in Omega, all schedulers are aware of the entire
state of the cluster, meaning that a job's resource allocation can be
varied after the job begins to execute. This flexibility offers significant
performance improvements.\cite[pg.352]{omega} Like Borg, Omega is not
open-source. However, like Borg, much of the work done on Omega is incorporated
into Kubernetes, Google's open-source cluster manager.
