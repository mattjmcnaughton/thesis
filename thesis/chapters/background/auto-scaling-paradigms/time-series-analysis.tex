We now examine an additional, substantially more complex, form of auto-scaling
situated upon predictive time-series analysis. Time-series analysis seeks to
find a repeating pattern in application load, and then horizontally auto-scale
the application based on these
patterns
\cite{auto-scaling-techniques-for-elastic-applications-in-cloud-environments}.
For example, if time-series analysis indicated a pattern in the application needed
$2x$ resources every Friday at 5pm,
it would be possible to auto-scale the application to $2x$
resources at this time. If we are able to compose a number of these
observations, we can create a policy for the entire auto-scaling behavior of
the given application by evaluating the application's predicted external environment
and determining the resources the application will need to operate in said
environment.

There are a variety of techniques for conducting time-series analysis
auto-scaling including pattern matching, signal processing, and
auto-correlation
\cite{auto-scaling-scaling-techniques-for-elastic-applications-in-cloud-environments}.\footnote{Netflix utilizes a combination of methods in
their predictive time-series analysis auto-scaler,
Scryer \cite{netflix-scryer-part-ii}.}
Like threshold-based rules for auto-scaling, there are significant
advantages and disadvantages to time-series analysis. Unlike threshold-based
rules which are marked by simplicity, time-series analysis is significantly more
complex. This complexity allows time-series analysis to be particularly 
fine-grained and effective
at responding to external changes when said changes have a pattern.\footnote{An
example of a change with a pattern would be Netflix users who are more likely to
watch TV at 10pm than 10am.} However, time-series analysis requires a
large amount of data and also substantial mathematical
knowledge; it certainly cannot be implemented as easily as specifying a few
simple thresholds. Additionally, while time-series analysis works well for
auto-scaling with respect to patterns, it does not work well when the external
application load is random, or incorporates elements of randomness. As such,
predictive time-series analysis is
often combined with reactive threshold-based rule auto-scaling to ensure the benefits of
both.\footnote{Netflix's Scryer implements this combination of
time-series analysis and threshold-based rule
auto-scaling.\cite{netflix-scryer-part-ii}}

