The simplest method of auto-scaling is threshold-based rule policies.
Threshold-based rule policies are reactive, as they perform scaling behaviors if
the current state of the application and host machine
is not in accordance with predefined rules. The rules
predominantly relate to per machine resource utilization levels. For example, a
rule could be that if the average CPU utilization percent for all of the
machines is above $80\%$, then a new machine should be created. This rule would
be accompanied with an additional scale-down rule stating that if the average
CPU utilization percentage for all of the machines is below $20\%$, then a machine
should be deleted. The most popular implementation of threshold-based rule
policies for auto-scaling comes from Amazon Web Services
\cite{amazon-auto-scaling-developer-guide}.\footnote{More
specifically, Amazon Web Services uses threshold-based rule policies for
auto-scaling with respect to EC2 instances. EC2 instances are essentially
rentable cloud
virtual machines \cite{amazon-ec2}}

Threshold-based rule policies offer both advantages and disadvantages with
respect to auto-scaling. Predominantly, these advantages and disadvantages arise
from threshold-based rule policies' conceptual simplicity.
Because threshold-based rule policies are reactive and based on simple metrics
like CPU utilization percentage and memory usage, they are simple to write.
However, they are difficult to write well, as it is difficult to predict how
certain rules will respond to the varying external circumstances. One particular
difficulty arises with respect to handling the nebulous time between when the
threshold is crossed and auto-scaling triggers the creation of a new
application, and when the newly created application can start running and
balancing the load.

Overall, there are a number of variables that must be considered when
determining the impact of threshold-based rule policies, reinforcing
that while it is easy to conceive of a threshold-based rule for auto-scaling,
it can be difficult to write a threshold-based rule having the desired effect if
an application will face varying external metrics.

