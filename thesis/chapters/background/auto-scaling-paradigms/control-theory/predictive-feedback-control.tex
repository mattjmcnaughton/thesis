As previously mentioned, feedback control systems are typically reactive,
meaning that the metrics used are based on the current state of the system.
However, we can also consider a feedback control system that is predictive. We
call this Model Predictive Control.\cite[pg.
27]{auto-scaling-techniques-for-elastic-applications-in-cloud-environments}
Again, we will spend significantly more time discussing predictive feedback
control later, but at its simplest, it is an implementation of feedback control
based auto-scaling, but the outputs are predictions about the future state of
the application instance, instead of the current state of the application instance.

Adding prediction to feedback control offers significant benefits. The most
significant benefit is accounting for the pod's creation time when auto-scaling.
With predictive feedback control, we can create pods so they are ready as soon
as they are needed. Ultimately, we hypothesize adding a predictive component to
Kubernetes current feedback control auto-scaling will allow us to auto-scale in
a manner that ensures both quality of service and efficient resource
utilization.

