As previously mentioned, feedback control systems are typically reactive,
meaning that the metrics used are based on the current state of the system.
However, we can also consider a feedback control system that is predictive. We
call this Model Predictive Control.\cite[pg.
27]{auto-scaling-techniques-for-elastic-applications-in-cloud-environments}
Again, we will spend significantly more time discussing predictive feedback
control later, but at its simplest, it is an implementation of feedback control
based auto-scaling, but the outputs are predictions about the future state of
the application instance, instead of the current state of the application instance.

Adding prediction to feedback control offers significant benefits. The most
significant benefit is accounting for the application's start up time when auto-scaling.
With predictive feedback control, we can create instances of the application
so they are ready as soon as they are needed. Again, if it takes 10
minutes for our application to be created and ready to operate, and we predict
we will need to application at 4pm, we can begin building it at 3:50 pm, so it
is ready as soon as needed.\footnote{If we were using reactive auto-scaling, we
would not know we needed the application until 4pm, and it would not be ready to
run until 4:10 pm.}
Ultimately, we hypothesize adding a predictive component to
Kubernetes current feedback control auto-scaling will allow us to auto-scale in
a manner that maintains efficient resource utilization and improves quality of
service.
