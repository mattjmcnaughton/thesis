We define efficient resource utilization as a measure of whether an application
has enough, but not too many resources given to it by the operating system or
the cluster manager. For example, editing a text file in Vim on a super computer
would be terrible efficient resource utilization, while running a web browser on
a laptop is proper efficient resource utilization. In the context of Kubernetes,
an application with poor efficient resource utilization would be a web server
that uses ten replica pods, each reserving considerable resources,
to serve a very low volume of web requests.

Specifically, we measure efficient resource utilization with respect to
Kubernetes through examining the amount of CPU that the application currently uses in
comparison to the amount of CPU that the application reserves. We define the
amount of CPU that an application reserves as the summation of the CPU that all
replica pods reserve. The amount of CPU that a pod reserves is the summation of
all resources that containers within the pod reserve
\cite{k8s-compute-resources}. Our specific metric for efficient resource
utilization will be measured in percentage of CPU and we name it \textit{CPU
utilization}.

The efficient resource utilization metric has direct links to the costs of
running applications on a cluster manager. If applications are given resources
they do not need, and the cluster manager does not reclaim these unused
resources, then additional applications added to the cluster must claim new
resources. The inability to utilize inefficient applications' wasted resources
requires the expansion of the cluster and an increase in cost that will be felt
both by those running the cluster and those running an application on the
cluster.
