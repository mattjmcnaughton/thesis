We seek to measure variations in a web server application's
efficient resource utilization and quality of service across a variety of
scaling methods. Many of the previous sections of this thesis, and the majority
of this thesis' contributions to Kubernetes, relate to implementing predictive
horizontal auto-scaling as a new scaling method. Yet, to evaluate we must also
create a web server application. Ultimately, the created web server application,
entitled \textit{test-server}, allows us refined control over the independent
variables we wish to control during experimentation.

\textit{test-server} is a HTTP server written in Go. It defines two API
endpoints, ``/" and "/ready".

\begin{itemize}
  \item \textbf{``/" - Load}: Traffic generators simulating user requests will
    send GET requests to the endpoint at ``\/", also known as the \textit{Load} endpoint.
    This endpoint has two functions. First, it must perform a task that is
    somewhat CPU intensive so as our application uses enough CPU that
    auto-scaling may be triggered. Second, it must record its CPU utilization
    percentage and an estimation of request response time as measures of
    efficient resource utilization and quality of service respectively. The
    function assigned to handle the GET request accomplishes these tasks as
    follows. Upon receiving a GET request to this endpoint, the handler function
    records the start time of the function as well as the amount of CPU clock
    already used by this process. It then executes a computationally intensive
    task, which we define as repeatedly encrypting a passphrase. Once this task
    is over, we once again record the time and CPU clock used by the process.
    Having these two values allows us to calculate the CPU utilization
    percentage and function exeuction time, which function as measurements of
    efficient resource utilization and quality of service. The function then
    records these values as a new point in the database, along with a
    an indicator of the values of independent variables in place for this trial
    (i.e. pod initialization time, scaling method). The function finishes by
    returning a status code of 200.
  \item \textbf{``/ready" - Ready}: The \textit{Ready} endpoint is used to allow
    us fine grained control over pod initialization time during experimentation.
    Kubernetes determines if a pod is initiliazed based on a
    \textit{ReadinessProbe}. This probe was discussed in greater detail earlier,
    but suffice to say it works by defining an HTTP endpoint that will return a
    successful status code if the pod is initialized. During experimentation, we
    are particularly interested in how predictive auto-scaling is effected by
    different pod initialization times. Thus, we utilize the following setup.
    First, when defining our pod, we specify the \textit{ReadinessProbe} should
    use ``/ready" as the HTTP endpoint. The function handling responses for
    requests to ``/ready" works by reading in an environment variable indicating
    the pod initialization time value we wish to test.\footnote{Any method of
    running containers, whether locally during testing or in a pod on Kubernetes
    will allow us to easily set environment variables.} The function parses this
    time, and waits that amount of time before returning a status code of 200.
    This wait ensures that we can control exactly when the
    \textit{ReadinessProbe} will receive a successful response, and thus can
    control exactly the amount of time before a pod initializes.
\end{itemize}

This application closely fits within the guidelines microservice theory
establishes and it is easy to containerize it and then run it on Kubernetes. It
is \textit{focused}, as it performs the single task of returning to HTTP
requests. It is \textit{stateless}, as all containers communicate with an
external database, that is not stored within the container. The container can be
deleted and restarted at anytime with no errors or sustained interrupts.
Finally, the container can easily be replicated and run concurrently, as
a load balancer can distribute discrete requests across multiple replica HTTP
servers with no threat of race conditions. It is thus simple to package this
webserver into a container, and then package said container into a pod.

As was previously mentioned, the pod containing the container for this
application implements a \textit{ReadinessProbe} pointing to the ``/ready"
endpoint. Additionally, the pod limits and requests an explicit amount of CPU,
currently .5 cores, for the container running the application. This ensures
measures of efficient resource utilization are consistent across scaling method.
Finally, our pod contains a number of environment variables including
\textit{SCALING\_METHOD} and \textit{INITIALIZATION\_TIME} which allow us the
fine grained control and measuring previously mentioned. Additionally,
regardless of the scaling method, we define a replication controller for
ensuring whatever the scaled amount of pods exist, and a service for balancing
requests across all \textit{test-server} containers and exposing
\textit{test-server} to an external IP address that can receive our generated
traffic.
