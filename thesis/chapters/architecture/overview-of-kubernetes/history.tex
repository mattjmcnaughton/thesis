Active development on Kubernetes began at Google in 2014, and as a result,
open-source developers have invested approximately 2 years of effort at the time
of this thesis being written. However,
as was mentioned in the background chapter, Kubernetes is not the first cluster
manager developed at Google. Over a decade and a half of Google's experience
informs Kubernetes. In addition, Kubernetes incorporates communally agreed upon new
innovations in cluster management \cite{what-is-k8s}.

It is important to remember that, in the realm of cluster managers, Kubernetes
is still young. While Kubernetes development has been ongoing for the past two
years, version 1.0 of Kubernetes was released July 21, 2015
\cite{k8s-v1-release}. Again, this recent release means that at the time of this
thesis' work, non-Google users have been running Kubernetes in production for
less than a year. In contrast, Mesos and YARN, the other open-source cluster
managers, have their origins in projects beginning approximately seven and ten years
ago respectively. This history confers both advantages and disadvantages. Long
running projects like Mesos and YARN have books, conferences, and multiple
companies with business models situated upon the success of this open-source
project. While Kubernetes enjoys some of these advantages, particularly related
to Google's sustained support, it does not have the history of some other
open-source cluster managers.\footnote{However, Kubernetes has a tremendous
amount of momentum and particularly impressive engagement considering its
limited history.} However, Kubernetes originality means there is lots of
exciting innovation left to come, particularly with respect to auto-scaling.
Part of why working on Kubernetes during this thesis is so exciting is that it
presents the opportunity to contribute to an already impressive project with
considerable future potential.
