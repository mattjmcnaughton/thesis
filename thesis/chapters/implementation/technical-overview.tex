As has been mentioned frequently throughout this thesis, Kubernetes is an
open-source project sponsored by Google. Unsurprisingly, given the scope of the
problems Kubernetes seeks to solve, the code base is very large. Overall,
Kubernetes contains approximately three million lines of application specific
code and thousands more of configuration scripts and documentation. The majority
of application specific code is written in Go, a relatively recent open-source
language also supported by Google. The influence of the Go programming language
is seen throughout the project, particularly in Go's native support for parallel
processing.

Given the size of Kubernetes, larger contributions adding substantial new functionality,
including the contributions made by this thesis, follow a standardized contribution process.
First, the developer must submit a proposal with their idea to the community.
The community can make comments, and it must receive approval from core members
of the Kubernetes if development is to be merged into the master release. Often
this discussion involves members of the Kubernetes team focusing on the area
relevant to the proposed changes. For this thesis, the proposal was considered
by the internal auto-scaling team. After the proposal is approved, code
containing the new feature, in this case predictive horizontal auto-scaling,
will be added in parts over the course of several smaller contributions. Again,
each request must be approved by the Kubernetes core team if it is to become
part of the project. Development on Kubernetes follows the standardized Git workflow.

This thesis' specific contributions to Kubernetes, and their justifications,
are described in the remainder of this section. The majority of the
contributions are in the relatively small horizontal podautoscaler controller
segment of the code, although an understanding of a greater portion of the code
base was needed to make these changes. Overall, approximately X lines of codes
were contributed to Kubernetes over the course of this thesis.

@TODO Describe the current state of the changes - are they implemented and
usable? Has the version been released yet?
