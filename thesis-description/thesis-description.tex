\documentclass[letterpaper,11pt,twocolumn]{article}
\usepackage{graphicx,times}
\usepackage{cite} % Cite allows the use of bibtex.

\begin{document}

\title{Predictive Autoscaling of Pods in the Kubernetes Container Cluster
Manager}
\date{\today}

\author{
  {\rm Matthew McNaughton}\\
  Advisor Jeannie Albrecht\\
  Computer Science Thesis, Williams College
}

\maketitle

\thispagestyle{empty}
%\pagestyle{empty}

\section{Motivation}
Society is becoming increasingly dependent on computation. Every facet of human life,
including health care, finance, and social interactions now involves computers,
resulting in a boom in the demand for computational power.
This explosion in need, which shows little signs of abating,
drives an interest in cluster computing.

Cluster computing combines the resources of thousands of
commodity computers to create the mass of computing power necessary to serve
high traffic sites, process massive MapReduce jobs, etc\dots While the biggest firms
in industry have used computing clusters,
and the requisite cluster managers, for approximately the last decade, this incredible
computing power is increasingly becoming available to the general public.
Kubernetes, an open source cluster manager from Google, leads this
effort.\cite{k8s-website}

Kubernetes manages many aspects of the cluster, including running processes,
maintaining machine health, and providing simple methods of querying the cluster
state. Kubernetes runs similar
processes in a unit of computing called a pod.
Pods have various performance requirements and face ever
changing external factors (ie. a variable number of server requests).
Given both these concerns, Kubernetes implements horizontal pod autoscaling, through pod
replication.\cite{k8s-horizontal-pod-autoscaler-proposal} Kubernetes
autoscales pods to ensure each pod honors a given threshold metric
(ie. max 80\% CPU usage). Replicating pods
essentially duplicates the number of processes being run, allowing a division
of labor among computing resources and decreasing the individual load on any
specific pod. For example, if a single pod was responsible for all web requests,
replicating another pod would allow each pod to now handle half the requests.
With horizontal pod autoscaling, a computational metric functions as a threshold
which, if surpassed, triggers pod replication until the metric returns below
the threshold value.

Introducing horizontal pod autoscaling greatly increases the responsiveness of
Kubernetes to a variance in external environment.
However, it is possible to make Kubernetes even more responsive. Specifically,
the current method of autoscaling does not account for the time necessary to
replicate pods. For example, if creation time for a pod is five minutes, even if
the autoscaler triggers replication as soon as the metric threshold is
crossed, it is still necessary to wait for the pod to be created before the work can
effectively be shared. This latency becomes especially problematic when the
costs of the pod being past the threshold are exceptionally high, pod
replication takes a non-trivial amount of time, or the demands on pods are
particularly variable. Clearly, while horizontal pod autoscaling offers
considerable benefits to the Kubernetes cluster manager, there are still steps
to be taken to address the previously listed issues.

\section{Question}

In this thesis, I will seek to address the deficiencies previewed in the
motivation section. Specifically, I will answer the following question: Given a
pod computation threshold metric triggering autoscaling, is it possible
to decrease the amount of time spent above the metric's threshold,
without substantially increasing the
net overall pod runtime? In other words, is it possible to predict when the
threshold metric will be crossed, so the replicated pod becomes available
exactly when needed?

\section{Hypothesis}

I hypothesize the answer to the stated question is yes. Confidence in this
hypothesis derives from the possibility of implementing preemptive
scaling.\cite{brendan-burns-conversation} Preemptive scaling measures the rate
of change for the given pod threshold metric, and predicts when the threshold
will be crossed. Combined with knowledge of the replication time for the pod,
the creation process for the replica can begin before the threshold is ever
crossed. When preemptive scaling works, as I hypothesize it will, the
replication pod is available exactly when needed, and the initial pod must spend
minimal time operating above the threshold metric while waiting for the pod
replication to complete.

\section{Implementation and Methodology}

Kubernetes is a rare and interesting project in that it is both completely open
source and in production use at Google and many other
companies.\cite{google-container-engine} Thus, I will be implementing my work
with preemptive autoscaling on the actual Kubernetes code, with the goal of my
work being accepted into the master branch, as part of the main
Kubernetes distribution.

A fairly defined methodology exists for evaluating cluster managers like
Kubernetes.\cite[pg. 355]{omega} The first aspect of evaluation utilizes a
lightweight simulator based on the statistical structure of the data I wish to
measure. For example, if I wanted to test the ability of Kubernetes with
preemptive autoscaling to respond to a spike in web traffic to a given pod, I
would simulate the highly variable incoming web requests with statistical
modelling. The second evaluation replays historic data, in order to make exact
conclusions about how Kubernetes with preemptive autoscaling would have operated in a
specific situation. This analysis is typically more restrictive, because it can
be difficult, and time-consuming, to obtain and replay exact data.

Evaluation will involve tracking the internal metrics Kubernetes already
records, including the amount of time a pod is running, and specific
computational metrics for the pod (ie. CPU, memory\dots). Some work will be
necessary to decide which metrics are best for measuring scaling responsiveness.

\section{Current State}

Working with Kubernetes has been extremely encouraging. Most importantly,
the lead developer of Kubernetes, Brendan Burns '98, is a Williams graduate who has
been very generous with his time and expertise, as we have discussed potential
areas of exploration. I am hopeful that I will be able to continue
working with Brendan going forward.

Naturally, Kubernetes is a large open source project, and as such there are
certain standards and customs which must be followed.
However, I have contributed to open source for over a
year, and am familiar with the workflow; I have already used my previous experience to
make small contributions to Kubernetes. I am optimistic and excited about
continuing to work with Kubernetes and the code I write for my
thesis hopefully being accepted into the main branch of the project.

\bibliography{thesis-description}
\bibliographystyle{plain}

\end{document}
