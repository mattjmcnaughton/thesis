As mentioned in the discussion of independent variables, we need to generate
substantial traffic conforming to a specific pattern. Fortunately there a
variety of open-source tools that can assist us in this task. Ultimately, the
one selected for this thesis is Apache JMeter \cite{apache-jmeter}. JMeter has
considerable functionality, but most appealing is its \textit{Throughput Shaping
Timer} plugin \cite{throughput-shaping-timer-plugin}. This plugin allows us to
use JMeter to send HTTP requests with a vary specific pattern, by specifying
exactly how many requests should be sent per second, and for what length of time
they should be sent. Using this tool, it is easy to create our previously
described patterns of traffic that we seek to test.

We containerize JMeter and place it into a pod which we run on Kubernetes. This
gives us considerably more resources with which to generate traffic, as opposed
to trying to generate a high volume of network requests from a laptop.
Importantly, we run JMeter within a different namespace than
\textit{test-server} pods, and also impose CPU limits on JMeter, to ensure that
our \textit{test-server} pods are isolated from JMeter's traffic generation
regardless of how much traffic JMeter generates.
