Additionally, we need a method for configuring
\textit{test-server} to incorporate the different variables that we wish to
test. Specifically, we need a way to ensure that \textit{test-server} can be run
on a Kubernetes cluster utilizing a variety of different methods for scaling,
and also that we can control the amount of time it takes for a pod to
initialize. In addition, we need \textit{test-server} to know the exact values
of its independent variables so that it can record them to the database,
ensuring all data is properly labelled. Our \textit{test-server} application
reads all of these dynamic values from Unix \textit{environment} variables.

It is possible to utilize Kubernete's configuration language to work with this
method of controlling independent variable values through environment variables.
We place our containerized \textit{test-server} application within a pod, and
Kubernetes allows the specification of environment variables within a pod. The
only issue is that these environment variables in the pod configuration file
must be static. We solve this issue by creating a template of our pod
configuration file, with indications of the dynamic environment variables. We
can then run a custom python that reads in configuration values, and creates
distinct configuration files incorporating each of these values. Thus, each
different file specifies a different pod configuration. Utilizing the proper
set of independent variables for a pod is as simple as creating a pod from the
correct configuration file. This entire process has been automated, meaning this
implementation detail has been largely abstracted.
