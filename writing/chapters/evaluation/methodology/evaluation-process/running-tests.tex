Given the powerful tooling described in the previous section, the process for
running a set of tests is completely automatable and quite simple. Each test
must specify a traffic pattern, a scaling method, and a pod initialization time.
These variables influence Kubernetes' \textit{ReplicationController} and
\textit{HorizontalPodAutoscaler} objects. Thus, while some configuration files
for Kubernetes test objects are consistent between tests, the configuration
files for the aforementioned varying \textit{ReplicationControllers} and
\textit{HorizontalPodAutoscalers} must be selected based on which test we are
running.

We do this selection using environment variables, which allow us to run the
tests with the following single \textit{make} command.

\begin{minted}{bash}
  export TS_RC=test-server-controller-reactive-5s.yaml;
  export TG_RC=traffic-generator-increase-decrease-test-plan.yaml; export HPA=TRUE; make
  test_start
\end{minted}

The above command indicates a wish to start a test instance with reactive
auto-scaling, a five second pod initialization time, and an
\textit{increase-decrease} traffic pattern. It also rebuilds all containerized
applications and ensures the configuration files are up to date.

Once complete, all of the pods, replication controllers, services, and
autoscalers on Kubernetes can be destroyed with the following \textit{make}
command.

\begin{minted}{bash}
  make test_stop
\end{minted}

As such, running a single test requires very little human involvement.
