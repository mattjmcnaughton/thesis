We define efficient resource utilization as a measure of whether an application
has enough, but not too many resources given to it by the operating system or
the cluster manager. For example, editing a text file in Vim on a super computer
would be terrible efficient resource utilization, while running a web browser on
a laptop is proper efficient resource utilization. In the context of Kubernetes,
an application with poor efficient resource utilization would be a web server
that uses many replica pods, each reserving considerable resources,
to serve a very low volume of web requests.

@TODO We will just have to make it clear that if we are using Idle CPU
percentage of measure efficient resource utilization, than the lower the value,
the better the efficient resource utilization.

Specifically, we measure efficient resource utilization with respect to
Kubernetes through examining the percentage of idle CPU.
The amount of CPU that a pod reserves is the summation of
all resources that containers within the pod reserve
\cite{k8s-compute-resources}. If our application is only using a small amount of
that reserved CPU to run, than a large amount of CPU will be left idle. The
larger the percentage of CPU that is left idle, the worse the efficient resource
utilization, as many resources are just sitting. Our specific metric for efficient resource
utilization will be measured in percentage of CPU that is idle and we name it
\textit{Idle CPU}.

The efficient resource utilization metric has direct links to the costs of
running applications on a cluster manager. If applications are given resources
they do not need, and the cluster manager does not reclaim these unused
resources, then additional applications added to the cluster must claim new
resources. The inability to utilize inefficient applications' wasted resources
requires the expansion of the cluster and an increase in cost that will be felt
both by those running the cluster and those running an application on the
cluster.
