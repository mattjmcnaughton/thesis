Importantly, it is an improvement of the summation of efficient resource
utilization and quality of service metrics that we are most interested in. It is
easy to improve quality of service by decreasing efficient resource
utilization, as we can just assign the application the largest amount of
resources it could ever need. It is equally easy to improve efficient resource
utilization by decreasing quality of service, as we can just assign an
application the fewest amount of resources it will ever use. As such, we want to
ensure that this thesis improves efficient resource utilization or quality of
service, without negatively impacting the other. This realization leads us to
evaluating our modifications to auto-scaling by measuring its impact on the
summation of efficient resource utilization and quality of service.

While conceptually summing efficient resource utilization and quality of service
is simple, care must be taken when combining the specific metrics of
\textit{idle CPU} and \textit{response time}. The metrics are measured in unrelated
units. Furthermore, the scale for these metrics may be entirely different,
meaning that small changes in one could completely overshadow larger changes in
the other.

@TODO How will we actually combine?
