In all, our evaluation makes a number of contributions. First, it outlines the
evaluatory goals of our thesis and how they will be measured. It then discusses
in depth how we will measure the success of our implementation, through ERU,
QoS, and the sum of ERU and QoS. It then introduces reactive auto-scaling as the
predominant control group, and different traffic patterns as the predominant
independent variables. It highlights a representative application and pod
initialization time on which to perform our evaluation. It also describes the
considerable tooling created to make evaluation possible, and the process for
using said tools. Finally, it discusses our results, which show while not
universally beneficial, there are scenarios in which predictive auto-scaling is
beneficial compared to reactive auto-scaling, and extrapolates the real-world
impacts of our specific results.
