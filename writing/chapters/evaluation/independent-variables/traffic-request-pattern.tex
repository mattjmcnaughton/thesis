We are also interested in the impacts of different traffic patterns on scaling
performance. As such, we send our test web server application web requests in a
variety of patterns and examine which traffic pattern the scaling method handles
well, and which traffic pattern the scaling method does not handle well.
Moreover, we also examine scaling methods in relation to each other with respect
to different traffic patterns, seeking to answer for which traffic patterns
predictive auto-scaling is beneficial and which traffic patterns predictive
auto-scaling is detrimental or meaningless.

@TODO Discuss time length for pattern and max requests per second and why these
values were chosen.

In this thesis we examine four different traffic patterns that we feel are
fairly indicative of the different traffic patterns a web application may face.
We entitle these patterns \textit{constant},
\textit{step-increase-decrease}, \textit{constant-increase-maintain}, and
\textit{flash-crowd}.\footnote{There are of course an infinite number of traffic
patterns that we could examine, and examining other options is an exciting
opportunity for future work.} We describe each of these patterns, and offer a
visual representation for each, below.

@TODO Once the time length and the max request per second values are set, then
include screenshotted graphs from JMeter to show the patterns.

@TODO Once these patterns are finalized, include a description of why they were
selected and what they represent, as well as technical info about the amount of
requests.

\begin{itemize}
  \item \textbf{constant}:
  \item \textbf{step-increase-decrease}:
  \item \textbf{constant-increase-maintain}:
  \item \textbf{flash-crown}:
\end{itemize}
