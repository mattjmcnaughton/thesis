As will be discussed in detail in the later \textit{Tools} section, we've created a web
server application that allows us to specify any \textit{pod
initialization time} we desire. As such, we have considerable flexibility with
respect to what initialization time values we test. Thus, we decided to test the
following values: 1s, 5s, 60s, 100s.

@TODO Find pit values indicative of certain use cases and test those (i.e. light
weight Go web server, heavy rate Rails server, application registering with
manager, etc.). Create a list stating why each value was chosen
and the use case it represents.

We believe that there will be a \textit{sweet spot} of pod initialization times
for which predictive auto-scaling demonstrates the most benefits over reactive
auto-scaling. Obviously the smaller the pod intialization time value, the lesser
the difference between reactive and predictive auto-scaling, as predictive
auto-scaling predicts into the future a time closer and closer to the current
moment. However, the larger the pod initialization time value, the further into
the future we must predict the state of the application. While large pod
initialization times have the potential for considerable benefits when using
predictive auto-scaling, we can only realize that potential if predictions of
future application state are accurate. As the prediction window gets larger and
larger, accuracy becomes substantially more difficult to obtain.
