We examine two independent variables, traffic request pattern and pod
initialization time, with respect to the different scaling types' performance.
In other words, for under-, average-, and over-static, reactive, and predictive
auto-scaling, we examine the impacts of varying the request pattern of traffic
to our testing application and the impacts of differing pod initialization
times. This allows us to determine under what combinations of traffic request
patterns and pod initialization time predictive auto-scaling is most effective,
and also under what combinations it is the least effective. Furthermore, because
we utilize the same independent variables for all of the different scaling
types, and because these independent variables are relevant to all scaling
types, we can make comparisons across the scaling types. For example, we could
determine that predictive auto-scaling outperforms reactive auto-scaling most
when pod initialization time is lengthy and the traffic pattern is a simple
linear slope, but predictive auto-scaling exhibits very little difference from
reactive auto-scaling when pod initialization time is very small and we see a
\textit{flash crowd} traffic pattern.

\subsection{Pod Initialization Time}

As will be discussed in detail in the later \textit{Tools} section, we've created a web
server application that allows us to specify any \textit{pod
initialization time} we desire. As such, we have considerable flexibility with
respect to what initialization time values we test. Thus, we decided to test the
following values: 1s, 5s, 60s, 100s.

@TODO Find pit values indicative of certain use cases and test those (i.e. light
weight Go web server, heavy rate Rails server, application registering with
manager, etc.). Create a list stating why each value was chosen
and the use case it represents.

We believe that there will be a \textit{sweet spot} of pod initialization times
for which predictive auto-scaling demonstrates the most benefits over reactive
auto-scaling. Obviously the smaller the pod intialization time value, the lesser
the difference between reactive and predictive auto-scaling, as predictive
auto-scaling predicts into the future a time closer and closer to the current
moment. However, the larger the pod initialization time value, the further into
the future we must predict the state of the application. While large pod
initialization times have the potential for considerable benefits when using
predictive auto-scaling, we can only realize that potential if predictions of
future application state are accurate. As the prediction window gets larger and
larger, accuracy becomes substantially more difficult to obtain.


\subsection{Traffic Request Pattern}

We are also interested in the impacts of different traffic patterns on scaling
performance. As such, we send our test web server application web requests in a
variety of patterns and examine which traffic pattern the scaling method handles
well, and which traffic pattern the scaling method does not handle well.
Moreover, we also examine scaling methods in relation to each other with respect
to different traffic patterns, seeking to answer for which traffic patterns
predictive auto-scaling is beneficial and which traffic patterns predictive
auto-scaling is detrimental or meaningless.

@TODO Discuss time length for pattern and max requests per second and why these
values were chosen.

In this thesis we examine four different traffic patterns that we feel are
fairly indicative of the different traffic patterns a web application may face.
We entitle these patterns \textit{constant},
\textit{step-increase-decrease}, \textit{constant-increase-maintain}, and
\textit{flash-crowd}.\footnote{There are of course an infinite number of traffic
patterns that we could examine, and examining other options is an exciting
opportunity for future work.} We describe each of these patterns, and offer a
visual representation for each, below.

@TODO Once the time length and the max request per second values are set, then
include screenshotted graphs from JMeter to show the patterns.

@TODO Once these patterns are finalized, include a description of why they were
selected and what they represent, as well as technical info about the amount of
requests.

\begin{itemize}
  \item \textbf{constant}:
  \item \textbf{step-increase-decrease}:
  \item \textbf{constant-increase-maintain}:
  \item \textbf{flash-crown}:
\end{itemize}

