Given the evaluation of a typical application benefiting from horizontal
auto-scaling on two distinct test patterns, we can comment on the general
scenarios in which predictive auto-scaling is effective and beneficial, and the
situations in which reactive auto-scaling is either less effective or
detrimental.

To begin, we consider when predictive auto-scaling is less
effective. Predictive auto-scaling offers little distinction from reactive
auto-scaling when the pod initialization time approaches 0s. Thus, predictive
auto-scaling is not particularly interesting for containerized web server
applications. This lack of difference is particularly noticeable given the
multi-minute threshold Kubernetes imposes between when auto-scalings can occur.
Our decision to not even evaluate a 5s pod initialization time reflected our
understanding of the lack of utility for predictive auto-scaling for web servers
that start particularly quickly.

Predictive auto-scaling has a greater positive and negative impact
with a longer pod initialization time, as we saw when auto-scaling on an
application simulating downloading shard data. From our graphs, we could see
repeatable performance differences, and scenarios in which predictive
auto-scaling offered the most benefits and scenarios in which reactive
auto-scaling offered the most benefits. Specifically, predictive auto-scaling
allows us to respond quickly to increases in load, ensuring that we
substantially react to initial spikes. While such behavior is beneficial in the
immediate aftermath of increased load, it can raise challenges if an even
greater scaling would otherwise be performed at a time within the
threshold Kubernetes imposes in
which scaling cannot occur. Particularly if predictive auto-scaling
underestimated the initial amount to scale, or only created a small replica
number of pods, scaling too early can lead to
performance decreases during the threshold period. Reactive auto-scaling avoids
these performance decreases by performing its scaling actions during period of
peak load, and thus creating greater replica pods. In short, predictive
auto-scaling offers a trade off between quickly and immediately responding to
increased load, while taking the risk that even greater increases during the
threshold time will go unaddressed until the threshold period ends.

With an eye towards the real-world, there are a variety of scenarios in which
predictive auto-scaling would be particularly useful. Specifically, the ability
to ensure we respond aggressively to an initial burst in traffic corresponds
nicely with notification services which will notify users and then achieve a
very steep, quick increase in users, before an equally steep decline. If this
burst occurs within the span of the threshold in which auto-scaling is
prevented, there will be no penalty to predictive auto-scalings aggressive
scaling. Furthermore, we can examine other scenarios in which we would rather
have high performance at the beginning of a heavy load period than the end. For
example, if we have a peer to peer system established in which all nodes in the
system originally query a single node, until they themselves have the
information and can be queriable, we would prefer a predictive auto-scaling
implementation. Predictive auto-scaling should sufficiently handle the initial
burst, and if there are any performance degradations during the threshold
period, they will be in part diminished by the previous successful queries of
the first node ensuring other nodes have the data and can thus serve as
additional replicas. This performance is in comparison to reactive auto-scaling,
which may not be able to handle the additional burst, and thus the system would
crash as no nodes would have been able to get the shared data. There are likely
a variety of additional operations and systems matching this general theme in
which predictive auto-scaling is particularly useful.
