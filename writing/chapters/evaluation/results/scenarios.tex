Given the evaluation of a typical application benefitting from horizontal
auto-scaling on two distinct test patterns, we can comment on the general
scenarios in which predictive auto-scaling is effective and beneficial, and the
situations in which reactive auto-scaling is either less effective or
detrimental.

To begin, we consider when predictive auto-scaling is less
effective. Predictive auto-scaling offers little distinction from reactive
auto-scaling when the pod initialization time approaches 0s. Thus, predictive
auto-scaling is not particularly interesting for containerized web server
applications. This lack of difference is particularly noticeable given the
multi-minute threshold Kubernetes imposes between when auto-scalings can occur.
Our decision to not even evaluate a 5s pod initialization time reflected our
understanding of the lack of utility for predictive auto-scaling for web servers
that start particularly quickly.

Predictive auto-scaling has a greater positive and negative impact
with a longer pod initialization time, as we saw when auto-scaling on an
application simulating downloading shard data. From our graphs, we could see
repeatable performance differences, and scenarios in which predictive
auto-scaling offered the most benefits and scenarios in which reactive
auto-scaling offered the most benefits. Specifically, predictive auto-scaling
allows us to respond to quickly to increases in load, ensuring that we
substantially react to initial spikes. While such behavior is beneficial in the
immediate aftermath of increased load, it can raise challenges if an even
greater scaling would otherwise be performed at a time within the
threshold Kubernetes imposes in
which scaling cannot occur. Particularly if predictive auto-scaling
underestimated the initial amount to scale, or only created a small replica
number of pods, scaling too early can lead to
performance decreases during the threshold period. Reactive auto-scaling avoids
these performance decreases by performing its scaling actions during period of
peak load, and thus creating greater replica pods. In short, predictive
auto-scaling offers a tradeoff between quickly and immediately responding to
increased load, while taking the risk that even greater increases during the
threshold time while go unaddressed until the threshold period ends.
