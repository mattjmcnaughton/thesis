Given the evaluation of a typical application benefiting from horizontal
auto-scaling on four distinct test patterns, we can comment on the general
scenarios in which predictive auto-scaling is effective and beneficial, and the
situations in which reactive auto-scaling is either less effective or
detrimental.

To begin, we consider when predictive auto-scaling is less
effective. Predictive auto-scaling offers little distinction from reactive
auto-scaling when the pod initialization time approaches 0s. Thus, predictive
auto-scaling is not particularly interesting for containerized web server
applications. This lack of difference is particularly noticeable given the
multi-minute threshold Kubernetes imposes between when auto-scalings can occur.
Our decision to not even evaluate a 5s pod initialization time reflected our
understanding of the lack of utility for predictive auto-scaling for web servers
that start particularly quickly.

Predictive auto-scaling has a greater positive, and negative, impacts
with longer pod initialization times, as we saw when auto-scaling on an
application simulating downloading shard data. From our graphs, we could see
repeatable performance differences, and scenarios in which predictive
auto-scaling offered the most benefits and scenarios in which reactive
auto-scaling offered the most benefits. Specifically, predictive auto-scaling
performed well in comparison when there were general long-trending linear
patterns which it could recognize. This benefit was particularly noticeable with
respect to reactive auto-scaling when there were temporary deviations from the
pattern, such as in \textit{step-ladder} and \textit{jagged-edge}. Reactive
auto-scaling would not be able to detect these as deviations, and would
misallocate accordingly, while predictive auto-scaling would stay true to the
overall pattern.

In contrast, predictive auto-scaling at times suffered when it was to eager to
recognize patterns or when multiple patterns existed and we used observations
from the previous pattern to try and predict the new pattern. The troublesome
impacts of this improper prediction can be seen in the
\textit{increase-decrease} and \textit{flash-crowd} traffic patterns.

In the end, predictive auto-scaling presents a choice. If one is fairly
confident one's application load will face load following a fairly consistent
linear pattern,\footnote{One of the first components of future work would
be expanding our prediction beyond just linear lines-of-best-fit, in which case
the possible patterns predictive auto-scaling would be useful for would
drastically expand.} then predictive auto-scaling is a powerful tool. Again, we
could expect this consistency in the instances we describe as being reflected by
the \textit{step-ladder} and \textit{jagged-edge} patterns, such as individuals
consistently entering a video streaming website to watch a regularly scheduled
video. If one's application load will vary between different patterns, or has no
pattern in any form, than predictive auto-scaling may not make sense as a
solution. It will actively deny reality to try and conform to whatever pattern
it best recognizes, which can have negative results. Again, this difficulty is
particularly reflected in traffic patterns like \textit{flash-crowd}, suggesting
this iteration of predictive auto-scaling may not be the best idea for news
providers, or other similar applications.
