In the introduction section of this thesis, we defined success as implementing
modifications to Kubernetes that will increase the summation of the
efficient resource utilization and quality of service metrics. Careful
evaluation will confirm whether we succeeded in this objective. In all, we will
seek to determine both the extent and significance of the addition of predictive
auto-scaling in comparison to other methods of allocations resources with
respect to our goal summation, and highlight the scenarios in which the addition
of predictive auto-scaling is the most, and the least, beneficial.

\subsection{Predictive Auto-scaling's Impact}

One important aspect of evaluation will be presenting a number of different
scenarios, based on independent variables that will be discussed later, in which
to evaluate predictive auto-scaling in comparison to the alternative methods
of assigning resources. This type of evaluation will focus on how predictive
auto-scaling is an improvement, or regression, on the currently existing
options. Furthermore, this type of evaluation will focus on to what extent
predictive auto-scaling's difference from current implementations,
is statistically and practically significant. Finally, this type of evaluation will
provide a number of visualizations and summary statistics, allowing for
easily accessible comparisons between different types of resource assignment
in different scenarios.


\subsection{Scenario Analysis}

Our evaluation will also attempt to provide
further insight into predictive auto-scaling
through analyzing the impacts of predictive auto-scaling in different scenarios.
In other words, we will seek to identify the scenarios in which benefits or
detriments of predictive auto-scaling are the most pronounced, and also the
scenarios in which predictive auto-scaling has little impact. This method of
analysis will be useful in recommending when to enable predictive auto-scaling
and also will likely suggest avenues for future work.

