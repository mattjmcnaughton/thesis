Before progressing any further, it is important to state the metrics we will use
to measure the efficacy of predictive auto-scaling. We seek specific metrics relating to
the general concepts of efficient resource utilization and quality of service.
A selection of a ``typical'' service to run on Kubernetes influences how exactly
we will measure efficient resource utilization and quality of service. We posit
a web application as the best choice for a representative application. A number
of factors support our decision. First, Kubernetes is built for running
long-term, stable service jobs, and a well-built web application desires to be
both long-running and crash free \cite{k8s-design-overview}. Second, Kubernetes
focuses on running ephemeral, containerized applications which can be started or
stopped at any time. Containerized web applications achieve these goals as, as
long as the database layer is abstracted, web applications are stateless and can
restart with few ramifications. Third, applications on Kubernetes should be
concurrent, meaning replicas can be added or removed to divide the work any
individual application must handle. Stateless web applications are easily
parallelized, as the requests can be easily distributed across all of the
replicas. Finally, much of the appeal of the simplicity of Kubernetes is that a
user can construct a containerized application and then pass it to an
externally managed Kubernetes cluster for hosting. This hosting simplicity is
particularly appreciated by burgeoning startups, who typically develop web
applications and may face extreme variance in external demand.

\subsection{Efficient Resource Utilization}

We define efficient resource utilization as a measure of whether an application
has enough, but not too many resources given to it by the operating system or
the cluster manager. For example, editing a text file in Vim on a supercomputer
would be terrible efficient resource utilization since editing a text file
requires only a small fraction of the supercomputer's available CPU and memory,
meaning the unneeded resources are wasted. In contrast, running a web browser on
a laptop is proper efficient resource utilization, because a web browser requires
an appropriate percentage of the laptop's available resources.
In the context of Kubernetes, an application with
poor efficient resource utilization would be a web server
that uses many replica pods, each reserving considerable resources,
to serve a very low volume of web requests. In such a situation, the resources
reserved for the application would be entirely underutilized.

Specifically, we measure efficient resource utilization with respect to
Kubernetes through examining the percentage of idle CPU.
The amount of CPU that a pod reserves is the summation of
all resources that containers within the pod reserve
\cite{k8s-compute-resources}. If our application is only using a small amount of
that reserved CPU to run, than a large amount of CPU will be left idle. The
larger the percentage of CPU that is left idle, the worse the efficient resource
utilization, as many resources are just sitting unused.
We measure our specific metric for efficient resource
utilization in percentage of CPU that is idle and we name it
\code{IdleCPU}. When we use this metric to indicate efficient resource utilization,
if the ERU value is high, the application is not using resources efficiently. If
the metric is low, the application is using resources efficiently.\footnote{In a
similar vein, a low measure for quality of service is actually preferable to a
high value for quality of service when we are measuring quality of service
through request response time. However, when summing ERU and QoS we consider
the inversion of this measurement of ERU, meaning a large summation of ERU and
QoS is preferable to a small summation of ERU and QoS.}

Our decision to use idle load to measure efficient resource utilization
in Kubernetes makes the assumption that
creating pods equates to reserving the pod's resources such
that they cannot be used by other pods.
In the default case, the previous statement is not necessarily true,
yet it is possible to craft specialized pods which validate this equality. To
start, remember that pods contain containers. When Kubernetes receives a pod, it
seeks to schedule all of its containers on a physical node within the cluster.
By default, containers within the pod run with no bounds on their CPU and memory
beyond the constrictions of the physical node on which they are scheduled. As such,
declaring $x$ number of pods does not give any guarantees of resource usage, as
the amount of resources available to the containers within the pod vary
drastically based on their specific node \cite{k8s-limit-range}. Without
modification, this variability would undermine \code{IdleCPU} as our metric for
ERU.

Fortunately, there is a way to configure Kubernetes such that a pod
equates to a static number of resources.\footnote{Right now in Kubernetes,
resources relates to either CPU or memory. CPU is requested in cores. Memory is
requested in bytes of RAM.} This configuration involves
setting resource requests and limits for each container within the pod. A
resource request for a container indicates the minimum amount of resources that
should always be available. A pod will not be scheduled on a node within the
cluster, unless that node can guarantee the requested amount of resources to all
containers within the pod. A resource limit for a container indicates the
maximum amount of resources that a container can claim. Depending on the
resource, a container exceeding the maximum amount of resources will either be
throttled (CPU) or killed (memory).\footnote{It is also possible to configure
Kubernetes such that a container using too much CPU is killed.} A pod's resource
request/limit is the summation of the resource request/limit for all of its
containers. Setting a container's, or pod's, resource request equal to its
resource limit essentially guarantees that the existence of a pod represents the
claiming and utilization of a static amount of resources
\cite{k8s-compute-resources}. Ensuring static provisioning
is reserving a consistent amount of resources
allows us to still examine idle CPU percentage, our way of investigating
efficient resource utilization.

The efficient resource utilization metric has direct links to the costs of
running applications on a cluster manager. If applications are given resources
they do not need, and the cluster manager does not reclaim these unused
resources, then additional applications added to the cluster must claim new
resources. The inability to utilize inefficient applications' wasted resources
requires the expansion of the cluster. This increase in cost will be felt
both by those running the cluster and those running an application on the
cluster.


\subsection{Quality of Service}

We additionally define quality of service as a measure of how well an application
is accomplishing its goal. There does not exist a singular consistent specific
metric for measuring quality of service, as measures of quality of service are dependent on
the specific application. Furthermore, it is difficult to measure quality of
service as a variety of difficult-to-control-for external factors impact an
application's ability to perform its goal.

In the context of the typical web application run on Kubernetes, we measure
quality of service based on the server side
\textit{response time} to an HTTP request. An application
with a high quality of service will have a low response time, while an
application with a low quality of service will have a high response time. Our
specific metric for quality of service will be measured in seconds.

The quality of service metric has links to the type of application which can be
run on Kubernetes. As Kubernetes supports as best as possible a high quality of
service, more and more important applications will run on Kubernetes. For
example, if Kubernetes works to improve the quality of service of an
application, important web applications serving vital medical data or political
information will seek Kubernetes as a platform on which to run.


\subsection{Summation of ERU and QOS}

We are most interested in testing for an improvement in the summation of
efficient resource utilization and quality of service metrics. It is
easy to improve quality of service by decreasing efficient resource
utilization, as we can just assign the application the largest amount of
resources it could ever need. It is equally easy to improve efficient resource
utilization by decreasing quality of service, as we can just assign an
application the fewest amount of resources it will ever use. As such, we want to
ensure that this thesis improves efficient resource utilization or quality of
service, without negatively impacting the other. This realization leads us to
evaluating our modifications to auto-scaling by measuring its impact on the
summation of efficient resource utilization and quality of service.

While conceptually summing efficient resource utilization and quality of service
is simple, care must be taken when combining the specific metrics of
\code{IdleCPU} and \code{ResponseTime}. The metrics are measured in unrelated
units. Furthermore, the scale for these metrics may be entirely different,
meaning that small changes in one could completely overshadow larger changes in
the other. Additionally, for both \code{IdleCPU} and \code{ResponseTime} low
measures are desirable, while intuitively with summation of ERU and QoS, high
measures are desirable.

We combine these ERU and QoS measurements through the following process.
First, we gather all measurements of ERU and QoS,
distinguished by the variables $E_{A}$ and $Q_{A}$ respectively, from our
predictive and reactive measurements for a single combination of independent
variables. Next, we define $e_{t}$ and $q_{t}$ as the
respective ERU and QoS measurements at time $t$. We first
normalize these measurements through calculating their respective z-scores,
by first subtracting the individual observation value
from the mean of all observations, and then dividing by the standard deviation of all
observations. This operation leaves us with $ne_{t}$ and $nq_{t}$ respectively.
Our next step relates to how we interpret measurements for ERU and QoS, and how
we interpret measurements for the summation of ERU and QoS. With the current
metrics we use for measuring ERU and QoS individually, smaller values indicate
``better'' performance. However, with the summation of ERU and QoS, it makes
intuitive sense that higher values should indicate ``better'' performance.
Thus, we negate $ne_{t}$ and $nq_{t}$, before we finally sum
$-ne_{t}$ and $-nq_{t}$ together to get $s_{t}$, where $s_{t}$ is the summation
of ERU and QoS at time $t$.
In short, we add the negation of the z-score for
ERU and QoS. Mathematically, this process can be written as
follows:

\begin{align*}
  ne_{t} &= ((e_{t} - \mbox{MEAN}(E_{A})) / \mbox{STDDEV}(E_{A})) \\
  nq_{t} &= ((q_{t} - \mbox{MEAN}(Q_{A})) / \mbox{STDDEV}(Q_{A})) \\
  s_{t} &= -ne_{t} + -nq_{t}
\end{align*}

Given a measurement of the summation of ERU and QoS for a set
of observations, in which ERU and QoS have an equal impact in
the summation, it is now possible to compare the summations of ERU and
QoS within the different scaling methods or traffic patterns included
in our evaluation trials.

