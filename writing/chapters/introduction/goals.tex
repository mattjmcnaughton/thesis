This thesis is most concerned with maximizing the efficient resource utilization
(ERU) and quality of service (QoS) metrics with respect to the Kubernetes cluster manager.
As such, this thesis pursues three goals:

\begin{enumerate}
  \item Given an application running on a Kubernetes cluster, we seek to
    determine a method which ensures quality of
    service stays consistently high regardless of variation in
    certain external factors. While it is
    difficult to make guarantees regarding quality of service, because
    application performance is dependent on a number of uncontrollable, varying
    external factors, it is possible to
    ensure each application has, and is utilizing, the resources it needs to
    function efficiently.
  \item A simplistic solution to the first goal of ensuring a high application
    quality of service is to just give each application many more resources than
    it will ever request.\footnote{Assigning an application more resources than it needs
    to function is defined as over-provisioning.}
    Yet, this over-provisioning is inefficient and costly. Thus, our
    methods for ensuring a high quality of service must also ensure the
    maintenance, or improvement, of the efficient resource utilization metric.
    As such, we add an additional goal: given a certain number of applications
    running on a Kubernetes cluster,
    we seek to determine a method which ensures Kubernetes allocates resources
    as efficiently as possible, while still comfortably
    supporting the applications' current, and future, resource needs and
    ensuring a high quality of service.
  \item Given that Kubernetes is an open-source project, we seek to implement, test, and
    evaluate a proposed enactment of the previous two goals.
    The methods we pursue will
    in part be dictated by the current structure and implementation of
    Kubernetes. Tests will be conducted using the Google Compute
    Engine\cite{google-compute-engine} on both
    simulated and real Kubernetes user data. The eventual goal is for this
    thesis' improvements to be merged into the production version of Kubernetes,
    and present another resource allocation method to Kubernetes users.
\end{enumerate}

