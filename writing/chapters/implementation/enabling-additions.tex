This section analyzes the specific additions enabling predictive auto-scaling.
The changes are broken intro four groups: Recording Pod Initialization Time,
Storing Previous CPU Utilizations, Autoscaling Predictively, and Enabling
Predictive Autoscaling. The first two groups do not directly relate to
predictive auto-scaling, but rather are the scaffolding that make predictive
auto-scaling possible. The third group, Autoscaling Predictively, builds upon
the first two sections in implementing an alternative control flow, which when
enabled, causes the auto-scaler to operate predictively instead of reactively.
The last group of changes discusses how to turn on this alternative control
flow, and thus enable predictive auto-scaling. While breaking this thesis' code
into discrete compartments is in part just a reflection proper software design,
it is also done to work within the framework of making open-source contributes
to Kubernetes. Kubernetes prefers small, self-contained pull requests, which can
be added without introducing breaking changes. Implementing the sections sequentially, ensures
we never submit code which will not run or build.

\subsection{Recording Pod Initialization Time}

This thesis' implementation of predictive auto-scaling predicts \textit{pod
initialization time} into the future. As was made explicit in section
\ref{architecture-predictive-autoscaling-in-relation-to-kubernetes-benefits-of-predictive-autoscaling},
this time frame allows any replica pods created to be ready to share in the work by the
time in which they are needed.

Pod \textit{initialization} time is distinct from pod \textit{creation} time and
the idea of a pod \textit{running}. We define an initialized pod as one that is
ready to perform computational work. This contrasts with a \textit{created} pod,
which merely signifies that all the containers within the pod are created. It
does not signify that these containers have started running, performed any
initialization tasks, and are now ready to share in computational work. The time
necessary to create all containers for a pod is not equal to the time necessary
for all containers within the pod to perform their computational work.
Furthermore, a \textit{running} pod is merely a pod in which all containers have
been created, and at least one container is running or in the process of
starting. This description does not guarantee that the containers have started
and are ready to perform work \cite{k8s-pod-states}. With conceptions
of pod creation and pod's
running being insufficient to determine whether a pod has initialized, we must
find an alternative mechanism.

Fortunately, Kubernetes defines the idea of a \textit{Readiness Probe}. A
readiness probe shows whether a pod is ready to handle incoming requests.
Services use this probe to determine whether to pass work along to a newly
created replica pod through ensuring that pods with long startup times do not
receive proxy traffic until ready to handle it. Pods must implement a readiness
probe, or else it will be assumed that if the pod is running, it is ready
\cite{k8s-pod-states}. Each container within a pod defines its own readiness
probe, by specifying an HTTP endpoint that will return a successful HTTP status
code when sent a GET request if the replica pod is ready
\cite{k8s-working-with-containers}.\footnote{There are alternative methods of
defining a readiness probe, but specifying an HTTP endpoint is the most logical
within the context of this thesis.} In terms of this thesis, a
pod being \textit{ready} and a pod being \textit{initialized} are analogous
terms.

We can rely on the existence of the readiness probe to
measure the amount of time it took for a pod
to initialize. Each pod records the time at which it was first ordered into
existence. Subtracting the time at which the pod first came into existence
from the time at which a request to the readiness probe was first successful.
results in the pod initialization time we desire. However, while Kubernetes
records the time at which the pod first came into existence as
\textit{pod.CreationTimestamp}, it does not record the time at which the
readiness probe first returned \textit{Success}. Fortunately, Kubernetes pods
record their current state and all states in which they have been, and the time
at which they assumed said states. As such, the following algorithm
calculates the average initialization time for all pods controlled by the auto-scaler.

\begin{itemize}
  \item For all pods controlled by the auto-scaler.
    \begin{itemize}
      \item If the pod was ever in the ready state, find the first time that
        occurred. If the pod has never been in the ready state, skip this pod.
      \item If a \textit{InitializationTime} value has not already been recorded for
        this pod, then subtract the \textit{CreationTime} from the time at which the
        pod entered the ready phase, and record this time as
        \textit{PodInitializationTime}
      \item Add the value for \textit{InitializationTime} to
        \textit{TotalInitializaitonTime}.
      \end{itemize}
    \item Divide \textit{TotalInitializationTime} by the total number of pods that
      have been ready at some point.
\end{itemize}

This algorithm allows us to record the initialization time for each pod
controlled by the auto-scaler, as well as calculate the average pod
initialization time for all pods controlled by this auto-scaler.\footnote{We
record the value in the \textit{Annotations} map for each pod,
which is basically a map for writing
values that are not necessarily in the Pod API object. If predictive
auto-scaling becomes particularly popular, \textit{InitializationTime} may be
included as a field in the Pod API object, although making such a change would
be a fairly substantial process.}

At this point, all recorded initialization time values factor into the average.
This inclusion could lead to a problem with extreme outliers drastically
affecting the average value. Perhaps in future iterations of predictive pod
auto-scaling it would make sense to only average values that fit within some
multiple of the standard deviation.

With this implementation, we can now use initialization time to determine how
far into the future to predict the state of the application on the cluster, and
thus how far into the future to auto-scale.


\subsection{Storing Previous CPU Utilizations}

Recording pod initialization time tells us how far into the future
we want to predict the resource usage of the application. However, we still need
a method of predicting the resource usage of the application at this point in
time. To make this prediction of future CPU utilization values,
we need to store multiple previous CPU utilization values.

The following changes must be made to the Kubernetes to facilitate this
recording. To start,
consider the contents of a \code{HorizontalPodAutoscaler} object, which is
just an Kubernetes API object representing an auto-scaler. This object contains
two objects entitled \code{HorizontalPodAutoscalerSpec} and
\code{HorizontalPodAutoscalerStatus}. Traditionally within
Kubernetes, \code{Spec} represents the desired state of the object and
\code{Status} represents the current state of the object. Before the addition
of predictive auto-scaling, the \code{HorizontalPodAutoscalerStatus} object
contained fields for the current and desired number of replicas, the last time
at which scaling occurred, and the current average CPU utilization percentage for
all pods controlled by this auto-scaler object. This last field, entitled
\code{CurrentCPUUtilizationPercentage}, provides part of the information
needed to estimate and predict the future CPU utilization
percentage \cite{k8s-horizontal-pod-autoscaler-object}.
However, to make any decent estimation, we need to know the previous CPU
utilization percentages at least as far into the past as we wish to predict into
the future.

Thus, we start tracking a new field entitled
\code{PreviousCPUUtilizationPercentages}. This field is a list of average previous CPU
utilization percentages in integer form. Fortunately, Kubernetes already
implements code that updates the \code{CurrentCPUUtilizationPercentage} value
at a set duration interval. We modify the code such that every time a new
\code{CurrentCPUUtilizationPercentage} is recorded, we add it onto the queue of
\code{PreviousCPUUtilizationPercentages} along with the timestamp at which
this observation occurred. We implement our queue such that it
is of a fixed length, and newer utilization percentage readings bump older ones
from the queue should we exceed the total number of observations we feel we need
to keep to be able to make accurate predictions about the future. In other words, at time
$t_{j}$, the \code{HorizontalPodAutoscaler} object will have access to
CPU utilization percentage values from $t_{i}$ to $t_{j}$, where $t_{i}$ is the
first observation recorded after $t_{i} - (p * c)$, where $p$ is the
amount of time it takes an
average pod being run by this auto-scaler to initialize,\footnote{The
amount of time it takes an
average pod being run by this auto-scaler to initialize is how far into the
future we seek to predict the state of the cluster.} and $c$ is a constant
multiplier.\footnote{In our current implementation, $c$ has a value of
$20$, indicating that we will predict based on measures twenty times as far from
in distance from the current time as the point in
the future at which we wish to predict. However, if $p * c$ is greater than 3
minutes, we simply record observations 3 minutes into the past.} With this recorded info,
we can now easily find a simple line of best fit
for the graph in which time is the independent variable and CPU utilization is
the dependent variable. Extrapolating with this line of best fit allows us to
predict the future CPU utilization of our pods.


\subsection{Autoscaling Predictively}

Now that we have methods to calculate the average pod initialization time and
keep records of previous average CPU utilizations, we can auto-scale
predictively. To do so, we must add an alternative branch of execution to the
current method of reactive auto-scaling.

With reactive auto-scaling, Kubernetes calculates the average current CPU utilization
across all pods and divides this number by the target CPU utilization
percentage to obtain the \textit{UsageRatio}. This usage ratio is then multiplied
by the current number of replica pods, resulting in the desired number of
replica pods. A similar process occurs through predictive auto-scaling, with one
significant change. Instead of calculating the average current CPU utilization,
we seek to calculate the average predicted CPU utilization at
\textit{PodInitializationTime} into the future. Given this value, we again
calculate \textit{UsageRatio} and multiply that value by the current number of
replica pods to give us the target number of replica pods.

Thus, the final step to implementing predictive auto-scaling is calculating
average predicted CPU utilization. There are a variety of different methods that
we could use for making this prediction, but for our initial experiments, we
choose the simplest method of generating a linear line of best fit for a
plotting of \textit{Time}, $x$, as the independent variable and
\textit{CPUUtilization}, $y$ as the dependent variable. As the points on our plot, we
use all of our previous CPU utilization measurements and the current CPU
utilization measurement, with \textit{Time}
recorded as Unix seconds\footnote{Unix seconds are the number of seconds from a
specific date in 1970} and CPU utilization recorded as a average percent CPU
utilization. Given these separate lists of $x$ and $y$ values, we can now find a
line of best fit. We define the line of best fit as the line minimizing the
squares of any derivations between predicted and actual average CPU utilization.
We calculate the slope of this line, $b$, with the following equation:

\[ b = \frac{Cov_{xy}}{Var_{x}}\]

$Cov_{xy}$ is a measure of how \textit{Time} and \textit{CPUUtilization} vary
with respect to each other, while $Var_{{x}}$ is a measure of the squared standard
deviation of all the different \textit{Time} values from the mean \textit{Time}
measurement.

Additionally, it has been proven that a linear line of best fit calculated in
this manner will pass through the sample mean of \textit{Time} and
\textit{CPUUtilization} respectively. Thus, we can calculate the y-intercept of
our line of best fit, $a$, by subtracting $b * \overline{x}$ from $\overline{y}$. With
these follows, we have a linear line of best fit, which we can use to make
simple predictions about future CPU utilization \cite{line-of-best-fit}.

Our final step is to get a \textit{Time} value to plug into this line of best
fit to receive a future prediction.\footnote{As was mentioned in the
Architecture section, we only utilize the predictive method when up-scaling.
Otherwise, if we notice the line of best fit has a negative slope, we simply
return the current CPU estimation, essentially reactively auto-scaling. The
similarity is because a replica pod is deleted almost immediately, and thus it
does not make sense to use the measure of pod initialization time to predict
into the future.} As we want to predict
\textit{PodInitializationTime} into the future, we simply add
\textit{PodInitializationTime}, measured in seconds, to the current time, for a
new value $t_{p}$. By calculating $a + b * t_{p}$, we have a prediction of the
future average CPU utilization. As mentioned previously, given this prediction,
we can now plug it in as if it was the current CPU utilization, and proceed with
the typical reactive method of auto-scaling.

Naturally, our method here is making a considerable assumption that a linear
line of best fit can accurately model CPU utilization and also the amount of
time we attempt to extrapolate when making our prediction is not too extreme.
Our evaluation section will help us how concerned we should be about these
potentially challenging aspects. Fortunately, should a linear line of best fit
not prove to be a suitable first attempt, our implementation is designed such
that it would be easy to examine a number of alternative modeling solutions,
such as quadratic, exponential, or logarithmic lines of best fits. It would even
be possible to try all the different modeling options, and ultimately select
the one that had proven itself most accurate.


\subsection{Enabling Predictive Autoscaling}

With all of the pieces for predictive auto-scaling in place, the final step is
making it so that a user running pods on the Kubernetes cluster can turn
prediction on and off. We seek a lightweight method for accomplishing this, so
as to not have to make drastic changes to the Kubernetes user interface before
being confident that predictive auto-scaling is a generally useful addition.

Fortunately, Kubernetes makes it possible to attach key/value pairs to any
resource through the concept of annotations. As was mentioned in the section on
recording pod initialization time, the annotations of an object are a useful way
to record information without going through the long process of making an
official update to the API for an object in Kubernetes. Thus, to turn on
auto-scaling for an individual auto-scaler object, we simply record in the annotations for
that object the key \textit{predictive} with the value \textit{true}. This
annotation can easily be done using Kubernetes command line
client \cite{k8s-kubectl-annotate}. For example,
if our auto-scaler had the name \textit{foo}, the following command would turn
on prediction.

\begin{minted}{bash}
  kubectl annotate hpa foo predictive=`true'
\end{minted}

It follows that auto-scaling could just as easily be turned off by rewriting the
annotated value to anything other than \textit{true}.

\begin{minted}{bash}
  kubectl annotate hpa foo predictive=`false'
\end{minted}

Within the code for determining the number of replica pods to create when
auto-scaling, we can easily check if the \textit{predictive} annotation is
\textit{true} for that particular auto-scaler. If the value is not set, or
anything other than true, the auto-scaler will reactively auto-scale as normal.
This flexibility allows different auto-scalers to utilize different auto-scaling
methods. Additionally, it allows a single auto-scaler to utilize different
auto-scaling methods throughout its lifetime.

If the benefits of predictive auto-scaling are substantial, then it will make
sense to transition this value from one recorded in \textit{Annotations} to a
more permanent field defined a auto-scaler's
\textit{HorizontalPodAutoscalerSpec} object. A field entitled
\textit{Predictive} could be added to this object which, if set to true, would
turn on predictive auto-scaling. This addition would enable to configure a
auto-scaler to perform predictively from the start, as opposed to having to first
create the auto-scaler and then turn on prediction. Making prediction a part of
the static configuration for an auto-scaler has the benefit of linking predictive
behavior with all other pod state, especially as configuration files are used
among multiple projects. However, until substantial benefits of predictive
auto-scaling are demonstrated, it does not make sense to undertake this effort.

