With all of the pieces for predictive auto-scaling in place, the final step is
making it so that a user running pods on the Kubernetes cluster can turn
prediction on and off. We seek a lightweight method for accomplishing this, so
as to not have to make drastic changes to the Kubernetes user interface before
being confident that predictive auto-scaling is a generally useful addition.

Fortunately, Kubernetes makes it possible to attach key/value pairs to any
resource through the concept of annotations. As was mentioned in the section on
recording pod initialization time, the annotations of an object are a useful way
to record information without going through the long process of making an
official update to the API for an object in Kubernetes. Thus, to turn on
auto-scaling for an individual auto-scaler object, we simply record in the annotations for
that object the key \textit{predictive} with the value \textit{true}. This
annotation can easily be done using Kubernetes command line
client \cite{k8s-kubectl-annotate}. For example,
if our auto-scaler had the name \textit{foo}, the following command would turn
on prediction.

\begin{minted}{bash}
  kubectl annotate hpa foo predictive=`true'
\end{minted}

It follows that auto-scaling could just as easily be turned off by rewriting the
annotated value to anything other than \textit{true}.

\begin{minted}{bash}
  kubectl annotate hpa foo predictive=`false'
\end{minted}

Within the code for determining the number of replia pods to create when
auto-scaling, we can easily check if the \textit{predictive} annotation is
\textit{true} for that particular auto-scaler. If the value is not set, or
anything other than true, the auto-scaler will reactively auto-scale as normal.
This flexibility allows different auto-scalers to utilize different auto-scaling
methods. Additionally, it allows a single autoscaler to utilize different
auto-scaling methods throughout its lifetime.

If the benefits of predictive auto-scaling are substantial, then it will make
sense to transition this value from one recorded in \textit{Annotations} to a
more permananet field defined a autoscaler's
\textit{HorizontalPodAutoscalerSpec} object. A field entitled
\textit{Predictive} could be added to this object which, if set to true, would
turn on predictive auto-scaling. This addition would enable to configure a
autoscaler to perform predictively from the start, as opposed to having to first
create the autoscaler and then turn on prediction. Making prediction a part of
the static configuration for an autoscaler has the benefit of linking predictive
behavior with all other pod state, especially as configuration files are used
among multiple projects. However, until substantial benefits of predictive
auto-scaling are demonstrated, it does not make sense to undertake this effort.
