Recording pod initialization time tells us how far into the future
we want to predict the resource usage of the application. However, we still need
a method of predicting the resource usage of the application at this point in
time. To make this prediction of future CPU utilization values,
we need to store multiple previous CPU utilization values.

The following changes must be made to the Kubernetes to facilitate this
recording. To start,
consider the contents of a \code{HorizontalPodAutoscaler} object, which is
just an Kubernetes API object representing an auto-scaler. This object contains
two objects entitled \code{HorizontalPodAutoscalerSpec} and
\code{HorizontalPodAutoscalerStatus}. Traditionally within
Kubernetes, \code{Spec} represents the desired state of the object and
\code{Status} represents the current state of the object. Before the addition
of predictive auto-scaling, the \code{HorizontalPodAutoscalerStatus} object
contained fields for the current and desired number of replicas, the last time
at which scaling occurred, and the current average CPU utilization percentage for
all pods controlled by this auto-scaler object. This last field, entitled
\code{CurrentCPUUtilizationPercentage}, provides part of the information
needed to estimate and predict the future CPU utilization
percentage \cite{k8s-horizontal-pod-autoscaler-object}.
However, to make any decent estimation, we need to know the previous CPU
utilization percentages at least as far into the past as we wish to predict into
the future.

Thus, we start tracking a new field entitled
\code{PreviousCPUUtilizationPercentages}. This field is a list of average previous CPU
utilization percentages in integer form. Fortunately, Kubernetes already
implements code that updates the \code{CurrentCPUUtilizationPercentage} value
at a set duration interval. We modify the code such that every time a new
\code{CurrentCPUUtilizationPercentage} is recorded, we add onto the queue of
\code{PreviousCPUUtilizationPercentages} along with the timestamp at which
this observation occurred. We implement our queue such that it
is of a fixed length, and newer utilization percentage readings bump older ones
from the queue should we exceed the total number of observations we feel we need
to keep to be able to make accurate predictions about the future. In other words, at time
$t_{j}$, the \code{HorizontalPodAutoscaler} object will have access to
CPU utilization percentage values from $t_{i}$ to $t_{j}$, where $t_{i}$ is the
first observation recorded after $t_{i} - (p * c)$, where $p$ is the
amount of time it takes an
average pod being run by this auto-scaler to initialize,\footnote{The
amount of time it takes an
average pod being run by this auto-scaler to initialize is how far into the
future we seek to predict the state of the cluster.} and $c$ is a constant
multiplier.\footnote{In our current implementation, $c$ has a value of
$20$, indicating that we will predict based on measures twenty times as for from
in distance from the current time as the point in
the future at which we wish to predict.} With this recorded info,
we can now easily find a simple line of best fit
for the graph in which time is the independent variable and CPU utilization is
the dependent variable. Extrapolating with this line of best fit allows us to
predict the future CPU utilization of our pods.
