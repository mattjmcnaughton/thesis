While this thesis accomplishes much in the realm of auto-scaling, and more
specifically predictive auto-scaling, there are multiple exciting paths
available for future exploration. Predominately these paths relates to expanding
our implementation of predictive auto-scaling, as well as expanding the
conditions under which we perform evaluation. Hopefully this process will be
positively cyclical in that new types of evaluation will engender new
implementation variables, which will enable new avenues for evaluation, and so
on.

First off, it would be fascinating to consider alternative traffic patterns.
Both of the increase-decrease and flash-crowd test plans are linear, yet real
world network traffic does not follow a linear pattern. Thus it would be
fascinating to analyze the performance of predictive auto-scaling when the
traffic pattern is polynomial, exponential, logarithmic, etc. It is likely that
we would find that the linear prediction method we have implemented does not
perform effectively in these scenarios. This recognition suggests implementing
different prediction mechanisms as another exciting avenue for future work.
Specifically, instead of only creating a linear line of best fit, we could
create a number of line of best fits, modelling the different possible equations
for a line, and then choose the one that best explained our observations so far.
Such improvements would vastly expand the scope under which predictive
auto-scaling would be useful.

In a similar vein, we mentioned how the current implementation of a threshold time
interval that most be observed between scalings was at time detrimental to
Kubernetes. It would be interesting to vary the length of this interval, in
order if it was possible to maintain the benefits of such an interval on prevent
thrasing, while also diminishing any negative impacts on predictive
auto-scaling.

Additionally, as was previously mentioned, when work on this thesis began,
Kubernetes only supported scaling based on CPU utilization. Yet, since then, it
is now possible to auto-scale with respect to custom metrics such as memory
usage. While our current auto-scaling implementation only works for CPU
utilization, the general algorithm should work, with small modifications, for
any metric that can be predictively auto-scaled. Particularly interesting would
be auto-scaling with respect to network congestion on the pod.

Fortunately, the majority of the tooling we built for this thesis is general
enough to be used for the majority of our areas of future exploration.
Furthermore, care was taken to implement predictive auto-scaling in an easily
extensible manor. As both the predictive auto-scaling implementation, and the
testing infrastructure for performing evaluations are open-sourced, any
interested individual should be able to contribute to future work.

Finally, one of the most important goals of this thesis remains as future work.
Much of the initial excitement about Kubernetes arose from its status as an
increasingly popular open-source application. While our predictive auto-scaling
implementation has not yet been merged into the Kubernetes master branch, we
remain workiing with core Kubernetes engineers to continue to implementation
process, and are excited about the potential for such contributions happening.
