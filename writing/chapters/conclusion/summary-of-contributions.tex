This thesis makes a number of contributions to distributed systems, cluster
managers, auto-scaling and Kubernetes. Most importantly, we posited,
implemented, and evaluated predictive auto-scaling in Kubernetes. To begin, we
formalized the problem of auto-scaling, suggesting a consistent definition of
what success in auto-scaling looks like. Next, we conceived of a
modification to the reactive auto-scaling
algorithm to support predictive auto-scaling in Kubernetes. Then, we implemented
our proposal, which will hopefully be part of the general Kubernetes code base
soon. Next, we built an extensive test framework for evaluating predictive
auto-scaling. These evaluatory efforts provided greater insight into
when predictive auto-scaling is and is not
effective and beneficial as measured through improving the summation of
efficient resource utilization and quality of service.
Finally, we suggested a number of avenues to
continue exploring predictive auto-scaling in Kubernetes, and cluster managers
in general. Overall, we took one more step along the path of making distributed
systems more accessible and reliable, which enables the solving of ever larger
and more important problems.
