While just recently described to the public in a 2015 paper, the
Borg cluster manager from Google has been in production use for over a
decade \cite{borg}. Borg incorporates
many different objectives, seeking to abstract resource management and error
handling, maintain high availability, and efficiently utilize resources on the
cluster. It is responsible for managing the hundreds of
thousands of batch and production jobs run everyday at Google.
Borg utilizes a monolithic scheduler, as jobs specify the
resources they need, and a single Borg scheduler decides how to admit and
schedule said jobs. Finally, Borg is not open-source. In fact, it was not even
publicly announced or described until 2015, despite running at Google for over a
decade. However, Kubernetes, Google's open-source cluster manager and the focus
of this thesis, incorporates many of the lesson's learned from Borg.
