Finally, we arrive at the cluster manager that is the focus of this thesis:
Kubernetes. Kubernetes also originates at Google, although it is open source.
Kubernetes is also the most
recent of the cluster managers we consider in this background chapter.\footnote{Kubernetes
became public in the summer of 2014.} The recent
explosion in popularity of containerization\footnote{We discuss both the motivations and
technology behind containerization in section
\ref{architecture-building-blocks-of-kubernetes-containerization}. Briefly, containerization is
packing everything an application needs to run into a single \textit{container}
and then running that container on any desired computer.}
heavily impacts the development and implementation of Kubernetes. Specifically,
Kubernetes is seen as part of a new paradigm of developing applications
through the use of microservices.\footnote{Again, we will
discuss microservices in greater detail
in the section \ref{architecture-building-blocks-of-kubernetes-microservices}.
Essentially, microservices are the division of applications
into small, easily scalable services which communicate with each other across
the network.} Kubernetes predominantly focuses on effectively running service
jobs, which require high-availability and potentially varying amounts of resources.
Additionally, Kubernetes currently allows one scheduler per namespace, making it
somewhat of a combination of the monolithic and shared-state schedulers.
Finally, Kubernetes is an open source project, yet is also used in production at
Google, and available to the public through the Google Container
Engine \cite{google-container-engine}.

We choose Kubernetes as the cluster manager on which to conduct our experiments
for a number of reasons. First, unlike many of the aforementioned cluster
managers, Kubernetes focuses on service jobs. Services jobs
have particularly stringent requirements for availability and are also the most
likely to have varying resource needs. Both of these conditions are closely linked
with the previously stated goals of this thesis, and Kubernetes is the cluster
manager that stands to benefit the most if we achieve our goals. Additionally,
Kubernetes is the only open source cluster manager focusing on long-running
services. Working with a open source cluster manager allows us to benefit from
the previous work of others, as well as expand the potential benefits of any
successful work.
