The pod is the smallest deployable unit that Kubernetes can create and
manage. A pod can contain one or more
containerized applications, and one or more pods can run on a physical host
machine. Multiple applications should run on the same pod
if these applications need to be run in the same
physical machine; otherwise, the applications should be in separate pods.
Containerized applications within a pod can see each other's processes,
access the same IP network, and share the same host name. Like well-designed
containers within the microservices model, well-designed containers and pods should be
focused, stateless, and concurrent, as Kubernetes assumes that pods can be deleted,
created, and replicated at will. The user can either submit a
single pod to Kubernetes to schedule and run on a node in the cluster, or
multiple replica pods
can be created by a replication controller \cite{k8s-pods}.
