As mentioned in the previous section, pods can be created by a
\textit{replication controller}. A replication controller
is responsible for ensuring that a given
number of replica pods are constantly running. It does this by restarting any
deleted or terminated pods. Though the replication controller performs a
seemingly simple task, it is useful for scaling the number of pods and
performing a rolling update of the pod \cite{k8s-replication-controllers}.

Additionally, auto-scaling is closely associated with replication controllers, as
auto-scalers are attached to replication controllers for pods. In short, the
auto-scaler tells the replication controller how many pods should exist, and the
replication controller ensures that number of pods exist and are operating
successfully. This relationship will be discussed in considerably more detail later when
examining the architecture and implementation of auto-scaling in Kubernetes in
Section \ref{architecture-autoscaling-in-kubernetes}
\cite{k8s-horizontal-pod-autoscaler-proposal}.
