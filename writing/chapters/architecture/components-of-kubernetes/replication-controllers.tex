As mentioned in the previous section, pods can be created by a replication
controller. A replication controller is responsible for ensuring that a given
number of replica pods are constantly running. It does this by restarting any
deleted or terminated pods. Though the replication controller performs a
seemingly simple task, it is useful for scaling the number of pods and for
performing a rolling update of the pod \cite{k8s-replication-controllers}.

Additionally, auto-scaling is closely associated with replication controllers, as
auto-scalers are attached to replication controllers for pods. When an
auto-scaler is attached, a replication controller no longer ensures a given
number of replica pods are running, but rather ensures that the number of
replica pods will result in pods consuming certain percentages of resources.
This relationship will be discussed in considerably more detail later when
examining the architecture and implementation of auto-scaling in Kubernetes
\cite{k8s-horizontal-pod-autoscaler-proposal}.
