The current algorithm for determining the number of replicas at each interval
of the control loop is fairly simple. It is assumed that auto-scaling is
occurring based on the pod's CPU utilization percentage, but this same algorithm
could be applied to other resources without the loss of generality.
To begin, it requires the definition of
three variables. Let \textit{TargetNumOfPods} be defined as the number of
replica pods which should exist. The replication controller will be responsible
for ensuring these pods exist. Let \textit{SumCurrentPodsCPUUtilization} be
found by multiplying the average pod CPU utilization by the current number of
replica pods. Finally, let \textit{TargetCPUUtilization} be the desired level of
per pod CPU utilization percentage as specified by the user when they created
the autoscaler. The algorithm can now be written as follows:

\[ \mbox{TargetNumOfPods} = \mbox{ceiling(SumCurrentPodsCPUUtilization /
TargetCPUUtilization)} \]

The \textit{ceiling} function simply ensures that there is no attempt to create
fractions of pods. Additionally, to keep auto-scaling from being overly
sensitive, scaling will only occur if the current resource utilization is
outside of a 10\% tolerance range \cite{k8s-horizontal-pod-autoscaler-proposal}.

Implementing this thesis' new auto-scaling additions, namely utilizing
predictive feedback control, will require modifications to this algorithm.
