In just a short time, Kubernetes has become a powerful option for open-source
cluster management and container orchestration. Kubernetes is particularly
important in that decreases the barrier of entry for exciting new technologies,
particularly containers, and programming paradigms, particularly microservices,
and grants users to benefits of portability, extensibility, flexibility, and
automation. Kubernetes is based on lower level objects called pods, replication
controllers, and services, and builds upon these to implement a model reactive
control method of auto-scaling, which it calls horizontal pod auto-scaling. This
thesis builds upon this work to implement a model predictive control method of
auto-scaling, entitled predictive horizontal pod auto-scaling. The belief is
that predictive horizontal pod auto-scaling will increase the summation of
quality of service and efficient resource utilization. This hypothesis will be
verified through evaluation.
