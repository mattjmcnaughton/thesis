Google often describes Kubernetes as ``an open source orchestration system for
Docker containers''\cite{k8s-website}. From this statement, it is clear that
containers, and more specifically Docker containers, inform much of Kubernetes.
To begin, it is important to understand the benefits of containers in comparison
to other similar options, and also to understand the technological underpinnings
of containerization and the resulting weaknesses.

First, containers are not virtual machines. Though both seek to silo different
applications running on the same physical machine, the manner in which they
seek to accomplish this task is very different. Virtual machines sit between the
operating system and the hardware, ensuring complete separation between two
virtual machines running on the same physical machine. While this strong
separation ensures applications running on separate virtual machines cannot
affect each other, placing a virtual machine monitor between the operating
system and the hardware has efficiency costs. More specifically, virtual
machines are substantially more memory intensive, and have a higher startup cost,
than lightweight containers \cite{distributed-systems-principles-and-paradigms}.
In contrast, containers should be thought of as ``lightweight wrappers around a
single Unix process''\cite[pg. 15]{docker-up-and-running}. As such, containers
require substantially less memory and can be started, destroyed, and restarted
in a matter of seconds. However, these ephemeral benefits come at the cost of
limited isolation between two containers running on the same physical host.
While there are ways to increase container isolation, it is not possible to
achieve the complete separation offered by virtual
machines \cite{docker-up-and-running}.

More formally, containers can be thought of as a single Unix process bundled
with the operating system and application specific files needed to run said process.
Containerizing a process involves saving a snapshot of the Linux operating system
running the process. This snapshot can be given to any platform which can run
the container, and the platform will then be able to simulate running the
application on the operating system bundled in the container.

So far, Docker has been the leading containerization platform. While Docker is a
young platform, which has only been in development for since 2013, it has seen
tremendous popularity and adoption. Again, while the containerization technology
upon which Docker is built was a part of Linux for over a decade, it was the
Docker platform that made it truly accessible \cite{docker-up-and-running}.
Kubernetes is one of many options for running multiple containers in a
production environment.
