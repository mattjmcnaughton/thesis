Additionally, Kubernetes fits within the recent interest in, and adoption of,
microservices. Microservices is a new paradigm of developing computer
applications. It is particularly suited towards developing large, complex
applications requiring many computing resources - the same types of applications
that are typically benefit from\/require being run on a cluster.

In short, microservices are ``small, autonomous services that work
together.''\cite[pg. 2]{building-microservices-designing-fine-grained-systems}
More specifically, microservices reflect the following characteristics:

\begin{itemize}
  \item \underline{Focused}: Each microservice should perform a single task.
    Furthermore, multiple microservices should be as loosely coupled as
    possible. Each microservice should present a single API for performing the
    single task, with which other microservices can communicate without
    understanding more nuanced implementation
    details.\cite{building-microservices-designing-fine-grained-systems}
  \item \underline{Stateless}: Individual instances of microservices should be stateless, in
    that a single instance can be deleted at any time without loosing any state.
    This requirement often leads to application's being containerized, and
    communicating with external, more permanent databases.
  \item \underline{Concurrent}: Multiple instances of a microservice should be
    able to work together to divide up the work presented to a single
    abstraction API. For example, if there is a microservice to support
    a search API, then it should be possible to run multiple instances of this
    search microservice and balance API requests between them so that they
    concurrently share work without interference.
\end{itemize}

Microservice design principles are particularly supportive of horizontal
auto-scaling. Microservices should be easily replicable and easily replaceable.
In other words, because microservices are stateless and can be run concurrently,
it is possible to easily scale a microservice by replicating it and then
dividing the work between the two. Auto-scaling down is supported, as it is
possible to delete an instance of the microservice without worrying about
loosing any important state. When the design principles of
microservices are followed, horizontal auto-scaling is possible, and its
implementation supports Kubernetes goal of easy automation. Additionally,
applications that follow microservice design principles are particularly easy to
containerize.

As will become evident in the following sections, Kubernetes is designed to
easily run applications developed based on the microservices model.
