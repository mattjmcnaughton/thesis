Finally, it is important to consider the ways in which Kubernetes can be run.
Kubernetes is a cluster manager, and as such can be run on any number of commodity
computers. These computers can be provided in many different ways. Perhaps the
simplest method is running Kubernetes on a single virtual machine on one's own
commodity computer. This method simulates running a single-node cluster using
Kubernetes as the cluster manager. Obviously this method does not facilitate
running production applications. If one does wish to run production
applications, the simplest method is using the Google Container Engine, which
hosts Kubernetes for the user such that the user just has to submit applications to
run. Alternatively, the user can host Kubernetes themselves on a number of
platforms including Google Compute Engine, Microsoft Azure, Rackspace, and
Amazon Web Services.\footnote{Ultimately we used Amazon Web Services to
host Kubernetes when performing our evaluations.}
Kubernetes ability to run on a number of cloud providers
allows users to run Kubernetes with a cluster of the exact size needed for
their applications. Overall, Kubernetes' flexibility assists the user in
accomplishing a number of different aims \cite{getting-started-k8s}.
