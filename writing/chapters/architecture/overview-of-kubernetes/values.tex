As with all large computer programs, a number of values drive the development of
Kubernetes. The values must pertinent to this thesis are as follows:

\begin{itemize}
  \item \underline{Portability}: As will be seen in the next section,
    Kubernetes can be run in a number of different environments. As a result,
    Kubernetes presents potential solutions to a number of different problems, and
    increasing Kubernetes performance, as this thesis attempts to do, will assist in
    this ability.
  \item \underline{Extensibility}: Kubernetes is built to
    easily incorporate pluggable new changes which can be easily enabled or
    disabled. Understanding the importance of pluggability will motivate this
    thesis' suggested modifications to Kubernetes.
  \item \underline{Automatic}: Kubernetes places an emphasis on important
    cluster management tasks occurring without necessary user involvement. For
    example, if an application crashes, Kuberentes will automatically restart
    it. This emphasis on automation means this thesis' work on auto-scaling
    should be particularly valuable, as it improves Kubernetes ability to
    accomplish one of its fundamental goals.
\end{itemize}

In part, Kubernetes can be distinguished from other open-source cluster mangers
based on these goals, and as previously mentioned, these goals align with the
goals of this thesis \cite{what-is-k8s}.
