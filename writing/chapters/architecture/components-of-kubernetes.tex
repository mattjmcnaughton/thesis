Introducing the Kubernetes specific vocabulary used through the cluster manager
will make the following discussions of auto-scaling easier. In addition to
describing each component of Kubernetes in general, Figure
\ref{fig:kubernetes-visualization-no-autoscaler} presents an overview of the
interaction between the different building blocks which compose Kubernetes.
Figure \ref{fig:kubernetes-visualization-no-autoscaler} also gives a sense of
the lifecycle of an external request to an application running on Kubernetes.

\begin{figure}[!h]
  \centerline{\includegraphics[scale=.25]{kubernetes-visualization-no-autoscaler.png}}
  \caption{A visualization of Kubernetes}
  \label{fig:kubernetes-visualization-no-autoscaler}
\end{figure}

\subsection{Pods}

The \textit{pod} is the smallest deployable unit that Kubernetes can create and
manage. A pod can contain one or more
containerized applications, and one or more pods can run on a physical host
machine. Multiple applications should run on the same pod
if these applications need to be run in the same
physical machine; otherwise, the applications should be in separate pods.
Containerized applications within a pod can see each others' processes,
access the same IP network, and share the same host name. Like well-designed
containers within the microservices model, well-designed pods should be
focused, stateless, and concurrent, as Kubernetes assumes that pods can be deleted,
created, and replicated at will. The user can either submit a
single pod to Kubernetes to schedule and run on a node in the cluster, or
multiple replica pods
can be created by a replication controller \cite{k8s-pods}.


\subsection{Replication Controllers}

As mentioned in the previous section, pods can be created by a replication
controller. A replication controller is responsible for ensuring that a given
number of replica pods are constantly running. It does this by restarting any
deleted or terminated pods. Though the replication controller performs a
seemingly simple task, it is useful for scaling the number of pods and for
performing a rolling update of the pod \cite{k8s-replication-controllers}.

Additionally, auto-scaling is closely associated with replication controllers, as
auto-scalers are attached to replication controllers for pods. When an
auto-scaler is attached, a replication controller no longer ensures a given
number of replica pods are running, but rather ensures that the number of
replica pods will result in pods consuming certain percentages of resources.
This relationship will be discussed in considerably more detail later when
examining the architecture and implementation of auto-scaling in Kubernetes
\cite{k8s-horizontal-pod-autoscaler-proposal}.


\subsection{Services}

The final main architectural building block of Kubernetes is the service.
Services are necessary because pods are ephemeral. As a pod may be deleted or
replaced at any time, it is important that no entity is trying to communicate
with a specific pod, because that pod may disappear. What is needed is a
consistent endpoint with which entities wishing to communicate with a pod can
always contact. A service is just such a consistent endpoint. Services prevent a
single, long-running access points for multiple replica pods, as it receives
requests to the pods, and load-balances them among the replicas.\footnote{The replication of
these pods is handled by a replication controller.} This endpoint is used within
a Kubernetes cluster as pods wish to communicate with each other and it can also
be exposed outside of the cluster. Again, the concept of a service is
particularly important to auto-scaling, as when the replication
controller creates pod replicas, the services ensures work is balanced across
them. This load-balancing, and the knowledge of the new replicas on which to
share the load, is entirely automated.\cite{k8s-services}

