As described in the Background chapter, this thesis examines the impacts of
implementing Model Predictive Control auto-scaling, instead of the current Model Reactive
Control auto-scaling. With a greater understanding of the building blocks of
Kubernetes, and the current implementation of auto-scaling in Kubernetes, it is
now possible to understand Model Predictive Control in a Kubernetes specific
context. Given the Kubernetes specific auto-scaling implementaiton is horizontal
pod auto-scaling, the updated, predictive version investigated in this thesis
will be referred to as predictive horizontal pod auto-scaling.

Converting Kubernetes from reactive horizontal pod auto-scaling to predictive
horizontal pod auto-scaling requires one major conceptual modification to the
auto-scaling process. As the name predictive implies, auto-scaling no longer
occurs in reaction to the current resource consumption of the pod. Rather, it occurs
predictively based on the pod's predicted future resource consumption. The
amount of time ahead for which resource consumption is predicted is dependent on
the amount of time it takes a replica pod to be ready to partake in the work the
service is balancing across all replica pods.\footnote{For example, a pod may
run a web server takes multiple minutes to initialize, and until that web server
is initialized, the pod cannot handle any web requests load balanced to it by
the service} This interval of time is referred to as \textit{PodInitializationTime}.

Aside from the switch to using predictive resource measurement, the
general architecture of predictive horizontal pod auto-scaling is similar to the
architecture of horizontal pod auto-scaling. Again, the autoscaler operates as a
control loop, yet at each looped interval, it utilizes the current resource
utilization of pods to predict the resource utilization of the pods after
\textit{PodInitializationTime}. Assuming the resource in question is CPU
utilization percentage, the auto-scaler will create or delete replica pods
to ensure the future CPU utilization percentage will be within the target range
the user specified when the autoscaler. It is important to remember that, while
the new replica pods are created immediately, they are not actually initialized
and reducing the work load of the other replica pods until after
\textit{PodInitializationTime}.
