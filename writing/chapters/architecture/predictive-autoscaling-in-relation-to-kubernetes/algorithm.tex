A couple of changes need to be made to the current reactive horizontal pod
auto-scaling algorithm in order for it to incorporate prediction. Again, we
assume that auto-scaling is occurring based on the metric of
CPU utilization percentage. Additionally, some of the variables from the
reactive horizontal pod auto-scaling algorithm appear again. Again, let
\code{TargetNumOfPods} be defined as the number of replica pods which should
exist and let \code{TargetCPUUtilization} be the desired level of per pod CPU
utilization percentage that the used specified when the auto-scaler was created.
Let the function \code{LineBestFit} be a line of best fit calculated
to fit a plotting of observation time, $x$, and previous CPU utilization
$y$.\footnote{We will discuss the method for calculating this line of best fit
in the implementation section.} Finally, let
\code{PredictionTime}, $t$, be the time in the future for
which we seek to predict the \code{FutureCPUUtilization}. We calculate
\code{PredictionTime} as follows:

\begin{align*}
  t = \mbox{CurrentTime} + \mbox{PodInitializationTime}
\end{align*}

We then use $t$ and the \code{LineOfBestFit} function to
calculate \code{FutureCPUUtilization} as follows:

\begin{align*}
  \mbox{FutureCPUUtilization} = \mbox{LineOfBestFit($t$)}
\end{align*}

The variable \code{SumFuturePodsCPUUtilization} can be defined by multiplying
\code{FuturePodCPUUtilization} by the current number of replica pods.
With these variables defined, it is possible to calculate
\code{TargetNumOfPods} as follows:

\[ \mbox{TargetNumOfPods} = \mbox{ceiling(SumFuturePodsCPUUtilization /
TargetCPUUtilization)} \]

The \code{ceiling} function definition and the tolerance range of 10\% remains from
reactive horizontal pod auto-scaling algorithm. With these changes,
\code{TargetNumOfPods} is now determined predictively.
