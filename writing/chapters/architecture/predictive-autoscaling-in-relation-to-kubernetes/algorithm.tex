A couple of changes need to be made to the current reactive horizontal pod
auto-scaling algorithm in order for it to incorporate prediction. Again, we
assume that auto-scaling is occuring based on the resource utilization metric of
CUP utilization percentage. Additionally, some of the variables from the
reactive horizontal pod auto-scaling algorithm appear again. Again, let
\textit{TargetNumOfPods} defined as the number of replica pods which should
exist and let \textit{TargetCPUUtilization} be the desired level of per pod CPU
utilization percentage that the used specified when the autoscaler was created.
As new variables, let \textit{CurrentPodCPUUtilization} be the CPU utilization
percentage for a single pod and let \textit{CurrentPodCPUUtilizationDerivative}
be the derivative of the CPU utilization for each pod. As in the previous section,
let \textit{PodInitializationTime} be equal
to the average amount of time it takes for replica pod to initialize,
where initialized is defined as being ready to share in the work the
service balances across the replica pods.\footnote{It is assumed that
\textit{PodInitializationTime} and \textit{CurrentPodCPUUtilizationDerivative}
utilize the same measurements of time.} It is then possible to calculate
\textit{FuturePodCPUUtialization}, a single pod's predicted CPU utilization
percentage after \textit{PodInitializationTime} with the following
equation:\footnote{This equation makes assumptions about a linear progression of
CPU utilization percentage of pods. While this seems like a safe initial
assumption, it remains an area for future experimentation.}

\begin{align*}
  \mbox{FuturePodCPUUtiliazation} = \mbox{CurrentPodCPUUtilization } + \\
  \mbox{(PodInitializationTime * CurrentPodCPUUtilizationDerivative)}
\end{align*}

The variable \textit{SumFuturePodsCPUUtilization} can be defined as the
summation of all of the \textit{FuturePodCPUUtilization} values for each pod.
With these variables defined, it is possible to calculate
\textit{TargetNumOfPods} as follows:

\[ \mbox{TargetNumOfPods} = \mbox{ceiling(SumFuturePodsCPUUtilization /
TargetCPUUtilization)} \]

The \textit{ceiling} function definition and the tolerance range of 10\% remainds from
reactive horizontal pod auto-scaling algorithm. With these changes,
\textit{TargetNumOfPods} is now determined predictively.
