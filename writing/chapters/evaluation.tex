Given the addition of auto-scaling, we can now progress to
evaluation. This section contains a more thorough definition of what entails
successful modification of the current Kubernetes auto-scaling behavior, the metrics
used to determine if efforts were successful, an analysis of the control groups
with which the new predictive auto-scaling will be compared, a description of
the environment and process used for evaluation, the results of said
evaluations, and the real world impacts of this thesis' findings.

\section{Goals of Evaluation}

\subsection{Predictive Auto-scaling's Impact}

It is important to present a number of different
scenarios, based on independent variables that will be discussed later, in which
to evaluate predictive auto-scaling in comparison to the alternative methods
of assigning resources. This type of evaluation focuses on how predictive
auto-scaling is an improvement, or regression, on the currently existing
options. Furthermore, this type of evaluation focuses on to what extent
predictive auto-scaling's difference from current implementations
is statistically and practically significant. Finally, this type of evaluation
provides a number of visualizations and summary statistics, allowing for
easily accessible comparisons between different types of resource assignments
in different scenarios.


\subsection{Scenario Analysis}

Our evaluation provides further insight into predictive auto-scaling
through analyzing the impacts of predictive auto-scaling in different scenarios.
In other words, we seek to identify the scenarios in which the benefits or
detriments of predictive auto-scaling are the most pronounced, and also the
scenarios in which predictive auto-scaling has little impact. This method of
analysis is useful in recommending when to enable predictive auto-scaling
and also suggests avenues for future work.



\section{Evaluation Metrics}

\subsection{Efficient Resource Utilization}

We define efficient resource utilization as a measure of whether an application
has enough, but not too many resources given to it by the operating system or
the cluster manager. For example, editing a text file in Vim on a supercomputer
would be terrible efficient resource utilization since editing a text file
requires only a small fraction of the supercomputer's available CPU and memory,
meaning the unneeded resources are wasted. In contrast, running a web browser on
a laptop is proper efficient resource utilization, because a web browser requires
an appropriate percentage of the laptop's available resources.
In the context of Kubernetes, an application with
poor efficient resource utilization would be a web server
that uses many replica pods, each reserving considerable resources,
to serve a very low volume of web requests. In such a situation, the resources
reserved for the application would be entirely underutilized.

Specifically, we measure efficient resource utilization with respect to
Kubernetes through examining the percentage of idle CPU.
The amount of CPU that a pod reserves is the summation of
all resources that containers within the pod reserve
\cite{k8s-compute-resources}. If our application is only using a small amount of
that reserved CPU to run, than a large amount of CPU will be left idle. The
larger the percentage of CPU that is left idle, the worse the efficient resource
utilization, as many resources are just sitting unused.
We measure our specific metric for efficient resource
utilization in percentage of CPU that is idle and we name it
\code{IdleCPU}. When we use this metric to indicate efficient resource utilization,
if the ERU value is high, the application is not using resources efficiently. If
the metric is low, the application is using resources efficiently.\footnote{In a
similar vein, a low measure for quality of service is actually preferable to a
high value for quality of service when we are measuring quality of service
through request response time. However, when summing ERU and QoS we consider
the inversion of this measurement of ERU, meaning a large summation of ERU and
QoS is preferable to a small summation of ERU and QoS.}

Our decision to use idle load to measure efficient resource utilization
in Kubernetes makes the assumption that
creating pods equates to reserving the pod's resources such
that they cannot be used by other pods.
In the default case, the previous statement is not necessarily true,
yet it is possible to craft specialized pods which validate this equality. To
start, remember that pods contain containers. When Kubernetes receives a pod, it
seeks to schedule all of its containers on a physical node within the cluster.
By default, containers within the pod run with no bounds on their CPU and memory
beyond the constrictions of the physical node on which they are scheduled. As such,
declaring $x$ number of pods does not give any guarantees of resource usage, as
the amount of resources available to the containers within the pod vary
drastically based on their specific node \cite{k8s-limit-range}. Without
modification, this variability would undermine \code{IdleCPU} as our metric for
ERU.

Fortunately, there is a way to configure Kubernetes such that a pod
equates to a static number of resources.\footnote{Right now in Kubernetes,
resources relates to either CPU or memory. CPU is requested in cores. Memory is
requested in bytes of RAM.} This configuration involves
setting resource requests and limits for each container within the pod. A
resource request for a container indicates the minimum amount of resources that
should always be available. A pod will not be scheduled on a node within the
cluster, unless that node can guarantee the requested amount of resources to all
containers within the pod. A resource limit for a container indicates the
maximum amount of resources that a container can claim. Depending on the
resource, a container exceeding the maximum amount of resources will either be
throttled (CPU) or killed (memory).\footnote{It is also possible to configure
Kubernetes such that a container using too much CPU is killed.} A pod's resource
request/limit is the summation of the resource request/limit for all of its
containers. Setting a container's, or pod's, resource request equal to its
resource limit essentially guarantees that the existence of a pod represents the
claiming and utilization of a static amount of resources
\cite{k8s-compute-resources}. Ensuring static provisioning
is reserving a consistent amount of resources
allows us to still examine idle CPU percentage, our way of investigating
efficient resource utilization.

The efficient resource utilization metric has direct links to the costs of
running applications on a cluster manager. If applications are given resources
they do not need, and the cluster manager does not reclaim these unused
resources, then additional applications added to the cluster must claim new
resources. The inability to utilize inefficient applications' wasted resources
requires the expansion of the cluster. This increase in cost will be felt
both by those running the cluster and those running an application on the
cluster.


\subsection{Quality of Service}

We additionally define quality of service as a measure of how well an application
is accomplishing its goal. There does not exist a singular consistent specific
metric for measuring quality of service, as measures of quality of service are dependent on
the specific application. Furthermore, it is difficult to measure quality of
service as a variety of difficult-to-control-for external factors impact an
application's ability to perform its goal.

In the context of the typical web application run on Kubernetes, we measure
quality of service based on the server side
\code{ResponseTime} to an HTTP request. An application
with a high quality of service will have a low response time, while an
application with a low quality of service will have a high response time. We
measure our specific metric for quality of service in seconds.

The quality of service metric has links to the type of application which can be
run on Kubernetes. As Kubernetes supports as best as possible a high quality of
service, more and more important applications will run on Kubernetes. For
example, if Kubernetes works to improve the quality of service of an
application, important web applications serving vital medical data or political
information will seek Kubernetes as a platform on which to run.


\subsection{Summation of ERU and QOS}

We are most interested in testing for an improvement in the summation of
efficient resource utilization and quality of service metrics. It is
easy to improve quality of service by decreasing efficient resource
utilization, as we can just assign the application the largest amount of
resources it could ever need. It is equally easy to improve efficient resource
utilization by decreasing quality of service, as we can just assign an
application the fewest amount of resources it will ever use. As such, we want to
ensure that this thesis improves efficient resource utilization or quality of
service, without negatively impacting the other. This realization leads us to
evaluating our modifications to auto-scaling by measuring its impact on the
summation of efficient resource utilization and quality of service.

While conceptually summing efficient resource utilization and quality of service
is simple, care must be taken when combining the specific metrics of
\code{IdleCPU} and \code{ResponseTime}. The metrics are measured in unrelated
units. Furthermore, the scale for these metrics may be entirely different,
meaning that small changes in one could completely overshadow larger changes in
the other. Additionally, for both \code{IdleCPU} and \code{ResponseTime} low
measures are desirable, while intuitively with summation of ERU and QoS, high
measures are desirable.

We combine these ERU and QoS measurements through the following process.
First, we gather all measurements of ERU and QoS,
distinguished by the variables $E_{A}$ and $Q_{A}$ respectively, from our
predictive and reactive measurements for a single combination of independent
variables. Next, we define $e_{t}$ and $q_{t}$ as the
respective ERU and QoS measurements at time $t$. We first
normalize these measurements through calculating their respective z-scores,
by first subtracting the individual observation value
from the mean of all observations, and then dividing by the standard deviation of all
observations. This operation leaves us with $ne_{t}$ and $nq_{t}$ respectively.
Our next step relates to how we interpet measurements for ERU and QoS, and how
we interpret measurements for the summation of ERU and QoS. With the current
metrics we use for measuring ERU and QoS individually, smaller values indicate
``better'' performance. However, with the summation of ERU and QoS, it makes
intuitive sense that higher values should indicate ``better'' performance.
Thus, we negate $ne_{t}$ and $nq_{t}$, before we finally sum
$-ne_{t}$ and $-nq_{t}$ together to get $s_{t}$, where $s_{t}$ is the summation
of ERU and QoS at time $t$.
In short, we add the negation of the z-score for
ERU and QoS. Mathematically, this process can be written as
follows:

\begin{align*}
  ne_{t} &= ((e_{t} - \mbox{MEAN}(E_{A})) / \mbox{STDDEV}(E_{A})) \\
  nq_{t} &= ((q_{t} - \mbox{MEAN}(Q_{A})) / \mbox{STDDEV}(Q_{A})) \\
  s_{t} &= -ne_{t} + -nq_{t}
\end{align*}

Given a measurement of the summation of ERU and QoS for a set
of observations, in which ERU and QoS have an equal impact in
the summation, it is now possible to compare the summations of ERU and
QoS within the different scaling methods or traffic patterns included
in our evaluation trials.



\section{Control Groups}

To determine the impact of adding prediction to horizontal pod auto-scaling, we
must establish baseline standards of performance with which we can perform comparisons.
These ``normal'' standards of performance come from the control group, which
includes all of the methods of scaling pods before this thesis. These methods
can be divided into the two general categories of \textit{static} and
\textit{reactive auto-scaling}. The inclusion in our control group of all
previous methods of auto-scaling allows us to answer one fundamental question of
this thesis: does adding prediction to Kubernetes auto-scaling improve its
ability to reliably and resourcefully run containerized applications?

\subsection{Static}

The first, and the simplest method, of scaling pods is \textit{static}
provisioning. Static provisioning requires one wishing to deploy an application on
Kubernetes to determine ahead of time a constant amount of pods for that
application. Any desire to update that static value will require a manual
change. Put simply, with the static method there will be a constant number of
pods, and the application will have a constant amount of resources, throughout
its entire lifetime, regardless of the amount of work the application is asked
to perform.

There are multiple possible heuristics for statically assigning resources to an
applications, as it is possible to over, under, or average provision.

% @TODO Include graphs for all of the examples below showing the results of
% static provisioning.

\begin{itemize}
  \item \textbf{Over Provision}: With over provisioning, an application is given
    the greatest amount of resources that it will ever require. With respect to
    horizontal pod auto-scaling, over provisioning means the user of the
    application statically sets the replication controller to ensure that $x$ pods
    always exist, where $x$ is the number of pods needed to
    maintain high quality of service when the application is
    asked to perform the most work.\footnote{In this discussion, references to
      \textit{most}, \textit{least}, and \textit{average} work assume that there
      exist bounds on the work the external environment can ask the application to
      do.} While over provisioning ensures a high quality of service, it has extremely
    poor efficient resource utilization.
  \item \textbf{Under Provision}: With under provisioning, an application is
    given the least amount of resources that it will ever require. Again, in the
    context of horizontal pod auto-scaling, under provisioning leads to user to
    statically set the replication controller to ensure the existence of
    $y$ pods, where $y$ is the number of pods needed to maintain quality of
    serice when the application is asked to perform the least work. Under
    provisioning ensures efficient resource utilization, as the application will
    never reserve any resources and then leave them idle. However, in all
    situations except for when the application performs the minimum possible
    amount of work, quality of service will suffer because the application does
    not have enough resources.
  \item \textbf{Average Provision}: With average provisioning, an application is
    given the average amount of resources that it needs. With respect to
    horizontal pod auto-scaling, average provisioning guides the user to
    statically set the replication controller to maintain $z$ pods, where
    $z$ is the number of pods needed to maintain quality of service when the
    application is asked to perform the average amount of work. Average
    provisioning can be seen as somewhat of a middle ground between under and
    over provisioning, offering decent quality of service and efficient resource
    utilization.
\end{itemize}

Our discussion of static provisioning in Kubernetes makes the assumption that
creating a static number of pods equates to reserving a static amount of
resources. In the default case, the previous statement is not necessarily true,
yet it is possible to craft specialized pods which validate this equality. To
start, remember that pods contain containers. When Kubernetes receives a pod, it
seeks to schedule all of its containers on a physical node within the cluster.
By default, containers within the pod run with no bounds on their CPU and memory
beyond the constrictions the physical node on which they are scheduled. As such,
declaring $x$ number of pods does not give any guarantees of resource usage, as
the amount of resources available to the containers within the pod vary
drastically based on their specific node \cite{k8s-limit-range}. This variability challenges the
stability of static provisioning, and weakens its ability to serve as a control
group with which we can compare predictive horizontal auto-scaling.

Fortunately, there is a way to modify configure Kubernetes such that a static number of
pods equates to a static number of resources.\footnote{Right now in Kubernetes,
resources relates to either CPU or memory. CPU is requested in cores. Memory is
requested in bytes of RAM.} This configuration involves
setting resource requests and limits for each container within the pod. A
resource request for a container indicates the minimum amount of resources that
should always be available. A pod will not be scheduled on a node within the
cluster, unless that node can guarantee the requested amount of resources to all
containers within the pod. A resource limit for a container indicates the
maximum amount of resources that a container can claim. Depending on the
resource, a container exceeding the maximum amount of resources will either be
throttled (CPU) or killed (memory).\footnote{It is also possible to configure
Kubernetes such that a container using too much CPU is killed.} A pod's resource
request\/limit is the summation of the resource request\/limit for all of its
containers. Setting a container's, or pod's, resource request equal to its
resource limit essentially guarantees that the existence of a pod represents the
claiming and utilization of a static amount of resources
\cite{k8s-compute-resources}. Ensuring static provisioning
is reserving a consistent amount of resources
allows us to still examine idle CPU percentage, our way of investigating
efficient resource utializaiton.


\subsection{Reactive Auto-scaling}

Additionally, previous to this thesis, it was possible to scale applications in
Kubernetes using horizontal, reactive pod based auto-scaling. We examine the implementation
and utilization of this method of scaling in depth in the \textit{Autoscaling in
Kubernetes} section, so we will not repeat it here.

However, some additional detail assists in understanding the potential values of
CPU utilization percentage that result from reactive horizontal auto-scaling.
Remember that the current implementation of reactive horizontal auto-scaling
occurs by having the auto-scaler create sufficient replica pods such that each
replica pods operates within a specified range of CPU utilization percentage.
\footnote{If we decide to enact resource requests\/limits on our replica pods, we can also
calculate the exact amount of resources being utilized, as opposed to just a
percentage. However, we are not particularly concerned about non-percentage
values because we measure efficient resource utilization using CPU utilization
percentage, instead of total CPU utilization.} If all replica pods initialize
and share in the work immediately, than we could expect CPU utilization
percentage to consistently stay within a small range of the value the user
specified. For example, if the user instructed the auto-scaler to auto-scale
such that all pods utilized 70\% of available CPU, and the range was $\pm 2$,
then we would expect CPU utilization, and thus efficient resource utilization,
to stay within $68 - 72$ CPU utilization percentage.
However, our justification for introducing prediction
the horizontal pod auto-scaling is an understanding that replica pods will not
always immediately initialize. In the times in which we are waiting for our pods
to initialize, we can expect CPU utilization to derivate from the expected
range. Establishing reactive horizontal pod auto-scaling as a contrl group
assists us in determining whether the addition of prediction improves upon what
existed in Kubernetes before this thesis began.



\section{Independent Variables}

\subsection{Traffic Request Pattern}

We are also interested in the impacts of different traffic patterns on scaling
performance. As such, we send our test web server application web requests in a
variety of patterns and examine which traffic pattern the scaling method handles
well, and which traffic pattern the scaling method does not handle well.
Moreover, we also examine scaling methods in relation to each other with respect
to different traffic patterns, seeking to answer for which traffic patterns
predictive auto-scaling is beneficial and which traffic patterns predictive
auto-scaling is detrimental or meaningless.

@TODO Discuss time length for pattern and max requests per second and why these
values were chosen.

In this thesis we examine four different traffic patterns that we feel are
fairly indicative of the different traffic patterns a web application may face.
We entitle these patterns \textit{constant},
\textit{step-increase-decrease}, \textit{constant-increase-maintain}, and
\textit{flash-crowd}.\footnote{There are of course an infinite number of traffic
patterns that we could examine, and examining other options is an exciting
opportunity for future work.} We describe each of these patterns, and offer a
visual representation for each, below.

@TODO Once the time length and the max request per second values are set, then
include screenshotted graphs from JMeter to show the patterns.

@TODO Once these patterns are finalized, include a description of why they were
selected and what they represent, as well as technical info about the amount of
requests.

\begin{itemize}
  \item \textbf{constant}:
  \item \textbf{step-increase-decrease}:
  \item \textbf{constant-increase-maintain}:
  \item \textbf{flash-crown}:
\end{itemize}


\subsection{Pod Initialization Time}

As will be discussed in detail in the later \textit{Tools} section, we've created a web
server application that allows us to specify any \textit{pod
initialization time} we desire. As such, we have considerable flexibility with
respect to what initialization time values we test. Thus, we decided to test the
following values: 1s, 5s, 60s, 100s.

@TODO Find pit values indicative of certain use cases and test those (i.e. light
weight Go web server, heavy rate Rails server, application registering with
manager, etc.). Create a list stating why each value was chosen
and the use case it represents.

We believe that there will be a \textit{sweet spot} of pod initialization times
for which predictive auto-scaling demonstrates the most benefits over reactive
auto-scaling. Obviously the smaller the pod intialization time value, the lesser
the difference between reactive and predictive auto-scaling, as predictive
auto-scaling predicts into the future a time closer and closer to the current
moment. However, the larger the pod initialization time value, the further into
the future we must predict the state of the application. While large pod
initialization times have the potential for considerable benefits when using
predictive auto-scaling, we can only realize that potential if predictions of
future application state are accurate. As the prediction window gets larger and
larger, accuracy becomes substantially more difficult to obtain.



\section{Methodology}

\subsection{Tools and Environment}

\input{chapters/evaluation/methodology/tools-and-environment}

\subsection{Evaluation Process}

Given the tools created and utilized for evaluation, we must consider how we
will utilize these tools in order to create a robust, automated testing
environment.

\subsubsection{Hosting}

For evaluation, we must run \textit{test-server} on a hosted Kubernetes
instance. It is necessary to use a hosted Kubernetes instance, instead of just
running Kubernetes locally, because we will be sending our pod an amount of
traffic too great for any single commodity machine to handle. Additionally, we
want to simulate running Kubernetes in as realistic a production environment as
possible, and of course all instances of running Kubernetes in production
require hosting Kubernets on external cloud servers.

There are a couple of different options for a hosting service which will provide
the machines for our Kubernetes cluster. The simplest method
of using Kubernetes is to use \textit{Google Container
Engine}. Google Container Engine is a version of Kubernetes hosted by Google
itself. The number of machines managed by Kubernetes is completely abstracted,
as the user only interacts with the Kubernetes command line tool,
\textit{kubectl}. While this method is admired for its simplicity, it is not
feasible for this thesis. Because we want to be able to test our modifications
to Kubernetes, without waiting for them to be accepted in the stable version of
Kubernetes used for Google Container Engine, we must instead use a platform that
allows greater control \cite{getting-started-k8s}.

Fortunately, Kubernetes can be run on a number of cloud providers, including
\textit{Google Compute Engine}, \textit{Amazon AWS}, and \textit{Microsoft
Azure} \cite{getting-started-k8s}.
The Kubernetes source code provides a number of simple scripts for
configuring one of these providers to run Kubernetes. Importantly, the version
of Kubernetes running on these providers can be any version we desire, meaning
that we can test our modified version that incorporates predictive auto-scaling,
even if our updates are not yet merged into a stable Kubernetes master version.
Because of previous development experience, we decided to
pursue hosting on Amazon AWS. Kubernetes typically runs 1 \textit{m3.medium}
EC2 instance as the master and 4 \textit{t2.micro} instances as workers, all
running in the \textit{us-west-2a} region. These defaults make
sense for the workload we expect \cite{getting-started-k8s-aws}.

Additionally, for simplicity's sake, we decided to host the InfluxDB database
instance used for storing our evaluation data. It would have also been possible
to run an instance of InfluxDB ourselves on an Amazon EC2 machine, but the
potential cost benefit did not message the time and complexity costs. Because
all of the data being stored is small key/value pairs, we only use a 10GB
Storage, 1GB RAM, 1 Core machine \cite{influxdb-pricing}.


\subsubsection{Kubernetes Configuration}

Additionally, we need a method for configuring
\textit{test-server} to incorporate the different variables that we wish to
test. Specifically, we need a way to ensure that \textit{test-server} can be run
on a Kubernetes cluster utilizing a variety of different methods for scaling,
and also that we can control the amount of time it takes for a pod to
initialize. In addition, we need \textit{test-server} to know the exact values
of its independent variables so that it can record them to the database,
ensuring all data is properly labelled. Our \textit{test-server} application
reads all of these dynamic values from Unix \textit{environment} variables.

It is possible to utilize Kubernete's configuration language to work with this
method of controlling independent variable values through environment variables.
We place our containerized \textit{test-server} application within a pod, and
Kubernetes allows the specification of environment variables within a pod. The
only issue is that these environment variables in the pod configuration file
must be static. We solve this issue by creating a template of our pod
configuration file, with indications of the dynamic environment variables. We
can then run a custom python that reads in configuration values, and creates
distinct configuration files incorporating each of these values. Thus, each
different file specifies a different pod configuration. Utilizing the proper
set of independent variables for a pod is as simple as creating a pod from the
correct configuration file. This entire process has been automated, meaning this
implementation detail has been largely abstracted.


\subsubsection{Running Tests}

Given the powerful tooling described in the previous section, the process for
running a set of tests is completely automatable and quite simple. Each test
must specify a traffic pattern, a scaling method, and a pod initialization time.
These variables influence Kubernetes' \textit{ReplicationController} and
\textit{HorizontalPodAutoscaler} objects. Thus, while some configuration files
for Kubernetes test objects are consistent between tests, the configuration
files for the aforementioned varying \textit{ReplicationControllers} and
\textit{HorizontalPodAutoscalers} must be selected based on which test we are
running.

We do this selection using environment variables, which allow us to run the
tests with the following single \textit{make} command.

\begin{minted}{bash}
  export TS_RC=test-server-controller-reactive-5s.yaml;
  export TG_RC=traffic-generator-increase-decrease-test-plan.yaml; export HPA=TRUE; make
  test_start
\end{minted}

The above command indicates a wish to start a test instance with reactive
auto-scaling, a five second pod initialization time, and an
\textit{increase-decrease} traffic pattern. It also rebuilds all containerized
applications and ensures the configuration files are up to date.

Once complete, all of the pods, replication controllers, services, and
autoscalers on Kubernetes can be destroyed with the following \textit{make}
command.

\begin{minted}{bash}
  make test_stop
\end{minted}

As such, running a single test requires very little human involvement.


\subsubsection{Interpreting Results}

Just as we have automated the process for running our evaluation tests, we have
also automated the process for interpreting the results from these tests. As all
of the results from our evaluation tests are stored in InfluxDB, we need to
write a script that retrieves these results in aggregated 1 minute intervals,
sums ERU and QoS for each observation, and generates graphs and summary
statistics comparing the difference in the summation of ERU and QoS for
predictive and reactive auto-scaling. We run this same script for all
combinations of traffic pattern and pod initialization time that we are
interested in. A more in-depth discussion of the analysis performed and the
combinations of traffic pattern and pod initialization time that we wish to test
will occur in the section \ref{evaluation-results}.




\section{Results}
\label{evaluation-results}

\subsection{Optimal Predictive Auto-scaler}

\subsubsection{Future Resource Utilization Prediction Method}

\input{chapters/evaluation/results/optimal-predictive-auto-scaler/future-resource-utilization-prediction-method}


\subsection{Impact of Predictive Auto-scaling}

As has been discussed in the previous sections, we are interested in visualizing
and understanding the difference in performance between predictive and
horizontal auto-scaling for one pod initialization time and four
different traffic patterns. As such, we have four different tests on which we
compare ERU and QoS: step-ladder traffic pattern with 135s pod initialization
time, jagged-edge traffic pattern with 135s pod initialization time,
increase-decrease traffic pattern with 135s pod
initialization time and flash-crowd traffic pattern with 135s pod initialization time.

For each test on the matrix, we provide two different sources of information.
First, we generate a graph comparing the summation of ERU and QoS across the
evaluation time for predictive and reactive auto-scaling. Additionally,
we provide statistical measurements for the difference of predictive and
reactive auto-scaling at the same point in the evaluation sequence (i.e. we
compare the summation of ERU and QoS after 10 minutes for predictive with the
summation of ERU and QoS after 10 minutes for reactive). With respect to
statistical measurements, we calculate a one-sided p-value
based on the null hypothesis that the difference
between the summation of ERU and QoS for predictive and reactive auto-scaling is
$0$. As we are interested in seeing if predictive auto-scaling performs better
than reactive auto-scaling, we calculate a one-sided p-value, $p$, with the alternative
hypothesis that the difference between the summation of ERU and QoS for
predictive and reactive auto-scaling is greater than $0$. We test for
significance at the $5\%$ significance level, meaning that if $p < 0.05$, we can
reject our null hypothesis in favor of our alternative hypothesis that
predictive auto-scaling performs better than reactive auto-scaling for this
combination of traffic pattern and pod initialization time.

\subsubsection{135s and step-ladder}

\begin{figure}[!h]
  \centerline{\includegraphics[scale=.70]{step-ladder-labelled.png}}
  \caption{A comparison of the summation of ERU and QoS for
    predictive and reactive auto-scaling for 135s, step-ladder}
  \label{fig:135s-step-ladder-labelled}
\end{figure}


Figure \ref{fig:135s-step-ladder-labelled} contains a graph
showing predictive and reactive auto-scaling's different
summations of efficient resource utilization and quality of service over the
course of trial. The traffic pattern super imposed below reflects the load
placed on the sample application, indicating the effect of the traffic pattern
on the summation of ERU and QoS. Given the larger the summation of ERU and QoS
the more desirable, times when the predictive line rises above the reactive line
reflect moments at which predictive auto-scaling is outperforming reactive
auto-scaling. Specifically, we can see three moments, labelled $1, 2,$ and
$3$ on the graph, in which predictive auto-scaling is particularly effective in
comparison to reactive auto-scaling. These moments help reveal a true strength of
predictive auto-scaling. While reactive auto-scaling under provisions because it
purely predicts based on the current moment, predictive auto-scaling is able to
recognize the general linearly upward pattern and auto-scale accordingly. As
such, the predictively auto-scaled application always has the resources it needs
to remain performant, as can be seen when examining the comparison of just
negated QoS in Figure \ref{fig:135s-step-ladder-only-qos}.\footnote{Again,
because we negated QoS when showing this graph, the larger the negated QoS
measure, the more performant the application, so it is desirable for the
predictive line to rise above the reactive line.} Finally, there is no cost in
ERU for auto-scaling, as can be seen when we just compare ERU in Figure
\ref{fig:135s-step-ladder-only-qos}.\footnote{This observation holds true
throughout the majority of the thesis. When the summation of predictive and
reactive auto-scaling diverges, it is because in variation of QoS. ERU stays
relatively constant between the two, which is necessary because it shows QoS
improvements are not coming at the cost of decreased ERU.
Thus for the remainder of the traffic
patterns we show only the summation of ERU and QoS with the understanding that
predominately QoS is contributing.} Overall, the included graphs clearly
demonstrate the benefits of predictive auto-scaling for this traffic pattern.

\begin{figure}[!h]
  \centerline{\includegraphics[scale=.70]{graph_135s_step-ladder_v2_only-eru.png}}
  \caption{A comparison of negated ERU for predictive and reactive auto-scaling
  for 135s, step-ladder}
  \label{fig:135s-step-ladder-only-eru}
\end{figure}

\begin{figure}[!h]
  \centerline{\includegraphics[scale=.70]{graph_135s_step-ladder_v2_only-qos.png}}
  \caption{A comparison of negated QoS for predictive and reactive auto-scaling
  for 135s, step-ladder}
  \label{fig:135s-step-ladder-only-qos}
\end{figure}

\begin{table}[htbp]
  \centering
  \caption{Difference in Predictive and Reactive Auto-scaling for 135s, step-ladder}
  \label{tab:135s-step-ladder}
\begin{tabular}{l c}\hline\hline
    \multicolumn{1}{c}{\textbf{Measure}} & \textbf{Value} \\ \hline
     p-value & 0.315 \\
     z-score & 0.482 \\
     std\_dev & 1.084 \\
     mean & 0.522
  \end{tabular}
\end{table}


While Figure \ref{fig:135s-step-ladder-labelled} shows the benefits of predictive
auto-scaling on the \textit{step-ladder} traffic pattern,
we are additionally interested in knowing if said benefits are
stastically significant.
As can be seen from the summary statistics in Table \ref{tab:135s-step-ladder},
with a p-value of .315 we are not
able to reject our null hypothesis that there is no difference between the
summation of ERU and QoS for predictive and reactive auto-scaling in favor of
our alternative hypothesis that there is a positive difference in the summation
of ERU and QoS for predictive and reactive auto-scaling. Essentially, this
p-value indicates that if there was truly no different between predictive and
reactive auto-scaling, we could expect to get these results about three out of
ten times we ran trials. Additionally, for the remainder of our trials with
different traffic patterns we were unable to obtain statistical significance.
This lack of statistical significance is likely the result of too little data,
given that we only ran one trial for each traffic-pattern, instead of combining
the results of multiple trials. Additionally, it may also be the case that the
inclusion of buffer periods, and periods when the requests per second are
sufficiently low such that only one pod needs to exist, is muting the
differences that occur when auto-scaling begins.


\subsubsection{135s and jagged-edge}

\begin{figure}[!h]
  \centerline{\includegraphics[scale=.70]{jagged-edge-labelled.png}}
  \caption{A comparison of the summation of ERU and QoS for
    predictive and reactive auto-scaling for 135s, jagged-edge}
  \label{fig:135s-jagged-edge-labelled}
\end{figure}


Figure \ref{fig:135s-jagged-edge-labelled} contains a graph
showing predictive and reactive auto-scaling's different
summations of efficient resource utilization and quality of service over the
course of the \textit{jagged-edge} trial.
The traffic pattern super imposed below reflects the load
placed on the sample application, indicating the effect of the traffic pattern
on the summation of ERU and QoS. Again, we significant times for which
predictive auto-scaling has a higher summation of ERU and QoS, and as such is
outperforming reactive auto-scaling. The decrease in the summation of ERU and
QoS for reactive auto-scaling, labelled with the $2$ on the graph, again
reflects the advantageous of predictive auto-scaling's ability to understand the
general linear pattern and not drastically under-provision as the result of
temporary downturns.

\begin{table}[htbp]
  \centering
  \caption{Difference in Predictive and Reactive Auto-scaling for 135s, jagged-edge}
  \label{tab:135s-jagged-edge}
\begin{tabular}{l c}\hline\hline
    \multicolumn{1}{c}{\textbf{Measure}} & \textbf{Value} \\ \hline
     p-value & 0.374 \\
     z-score & 0.320 \\
     std\_dev & 1.062 \\
     mean & 0.340
  \end{tabular}
\end{table}

Again, while Figure \ref{fig:135s-jagged-edge-labelled} shows the benefits of predictive
auto-scaling on the \textit{jagged-edge} traffic pattern, Table
\ref{tab:135s-jagged-edge} shows that we are unfortunately not able to claim
statistical significance with respect to these results.


\subsubsection{135s and increase-decrease}

\begin{figure}[!h]
  \centerline{\includegraphics[scale=.70]{increase-decrease-labelled.png}}
  \caption{A comparison of the summation of ERU and QoS for
    predictive and reactive auto-scaling for 135s, increase-decrease}
  \label{fig:135s-increase-decrease-labelled}
\end{figure}

Figure \ref{fig:135s-increase-decrease-labelled} contains a graph
showing predictive and reactive auto-scaling's different
summations of efficient resource utilization and quality of service over the
course of the \textit{increase-decrease} trial. In contrast to our previous two
traffic patterns, we find that predictive auto-scaling is not particularly
beneficial in this context. Specifically, if we look at the moment labelled
$1$ on \ref{fig:135s-increase-decrease-labelled}, we an instance in which
predictive auto-scaling suffers a severe performance decrease in comparison to
reactive auto-scaling. We trace this decrease again to under-provsioning. In
this scenario, the introductory buffer period has caused our linear prediction
algorithm to underestimate the slope of the line indicating the rise in load. As
such, the reactive auto-scaling algorithm actually has a more aggressive opinion
of the load the application will face. As this aggressive understanding is
confirmed by our actual increase, reactive auto-scaling outperforms predictive
auto-scaling on the \textit{increase-decrease} traffic-pattern.

\begin{table}[htbp]
  \centering
  \caption{Difference in Predictive and Reactive Auto-scaling for 135s, increase-decrease}
  \label{tab:135s-increase-decrease}
\begin{tabular}{l c}\hline\hline
    \multicolumn{1}{c}{\textbf{Measure}} & \textbf{Value} \\ \hline
     p-value & 0.638 \\
     z-score & -.0352 \\
     std\_dev & 0.680 \\
     mean & -0.234
  \end{tabular}
\end{table}

Figure \ref{fig:135s-increase-decrease-labelled} shows that predictive
auto-scaling is actually slightly detrimental on the \textit{increase-decrease} traffic pattern,
Still, Table \ref{tab:135s-increase-decrease} shows that we are not able to claim
statistical significance with respect to these results, and thus should not be
too confident that prediction plays a negative effect. Rather, it appears to
have very little impact in either direction on this traffic pattern.


\subsubsection{135s and flash-crowd}

\begin{figure}[!h]
  \centerline{\includegraphics[scale=.70]{flash-crowd-short-labelled.png}}
  \caption{A comparison of the summation of ERU and QoS for
    predictive and reactive auto-scaling for 135s, flash-crowd.}
  \label{fig:135s-flash-crowd-labelled}
\end{figure}

Figure \ref{fig:135s-flash-crowd-labelled} contains a graph
showing predictive and reactive auto-scaling's different
summations of efficient resource utilization and quality of service over the
course of the \textit{flash-crowd} trial. Again,
we find that predictive auto-scaling is not particularly
beneficial in this context. Specifically, if we look at the moment labelled
$1$ on Figure \ref{fig:135s-flash-crowd-labelled}, we see another instance in which
predictive auto-scaling suffers a severe performance decrease because of
under-provisioning. Because this traffic pattern occurs over such a short
interval, the predictive auto-scaling algorithm is unable to fully recognize and
respond to the flash crowd, and is instead hampered by previous measurements
with a significantly lesser slope. Our line-of-best-fit for prediction has too
small a slope, and thus predictive auto-scaling functions worse than reactive
auto-scaling.

\begin{table}[htbp]
  \centering
  \caption{Difference in Predictive and Reactive Auto-scaling for 135s,
  flash-crowd.}
  \label{tab:135s-flash-crowd}
\begin{tabular}{l c}\hline\hline
    \multicolumn{1}{c}{\textbf{Measure}} & \textbf{Value} \\ \hline
     p-value & 0.802 \\
     z-score & -.850 \\
     std\_dev & 0.695 \\
     mean & -0.591
  \end{tabular}
\end{table}

Figure \ref{fig:135s-flash-crowd-labelled} shows that predictive
auto-scaling is fairly detrimental on the \textit{flash-crowd} traffic pattern.
Given Table \ref{tab:135s-flash-crowd}, we are still not able to claim
statistical significance with respect to these results, but we should be fairly
confident that this current iteration of predictive auto-scaling is not an advisable addition
when expecting flash crowds.



\subsection{Scenarios}

Given the evaluation of a typical application benefiting from horizontal
auto-scaling on two distinct test patterns, we can comment on the general
scenarios in which predictive auto-scaling is effective and beneficial, and the
situations in which reactive auto-scaling is either less effective or
detrimental.

To begin, we consider when predictive auto-scaling is less
effective. Predictive auto-scaling offers little distinction from reactive
auto-scaling when the pod initialization time approaches 0s. Thus, predictive
auto-scaling is not particularly interesting for containerized web server
applications. This lack of difference is particularly noticeable given the
multi-minute threshold Kubernetes imposes between when auto-scalings can occur.
Our decision to not even evaluate a 5s pod initialization time reflected our
understanding of the lack of utility for predictive auto-scaling for web servers
that start particularly quickly.

Predictive auto-scaling has a greater positive and negative impact
with a longer pod initialization time, as we saw when auto-scaling on an
application simulating downloading shard data. From our graphs, we could see
repeatable performance differences, and scenarios in which predictive
auto-scaling offered the most benefits and scenarios in which reactive
auto-scaling offered the most benefits. Specifically, predictive auto-scaling
allows us to respond to quickly to increases in load, ensuring that we
substantially react to initial spikes. While such behavior is beneficial in the
immediate aftermath of increased load, it can raise challenges if an even
greater scaling would otherwise be performed at a time within the
threshold Kubernetes imposes in
which scaling cannot occur. Particularly if predictive auto-scaling
underestimated the initial amount to scale, or only created a small replica
number of pods, scaling too early can lead to
performance decreases during the threshold period. Reactive auto-scaling avoids
these performance decreases by performing its scaling actions during period of
peak load, and thus creating greater replica pods. In short, predictive
auto-scaling offers a trade off between quickly and immediately responding to
increased load, while taking the risk that even greater increases during the
threshold time while go unaddressed until the threshold period ends.

With an eye towards the real-world, there are a variety of scenarios in
predictive auto-scaling would be particularly useful. Specifically, the ability
to ensure we respond aggressively to an initial burst in traffic corresponds
nicely with notification services which will notify users and then achieve a
very steep, quick increase in users, before an equally steep decline. If this
burst occurs within the span of the threshold in which auto-scaling is
prevented, there will be no penalty to predictive auto-scalings aggressive
scaling. Furthermore, we can examine other scenarios in which we would rather
have high performance at the beginning of a heavy load period than the end. For
example, if we have a peer to peer system established in which all nodes in the
system originally query a single node, until they themselves have the
information and can be queriable, we would prefer a predictive auto-scaling
implementation. Predictive auto-scaling should sufficiently handle the initial
burst, and if there are any performance degradations during the threshold
period, they will be in part diminished by the previous successful queries of
the first node ensuring other nodes have the data and can thus serve as
additional replicas. This performance is in comparison to reactive auto-scaling,
which may not be able to handle the additional burst, and thus the system would
crash as no nodes would have been able to get the shared data. There are likely
a variety of additional operations and systems matching this general theme in
which predictive auto-scaling is particularly useful.



\section{Summary}

In all, our evaluation makes a number of contributions. First, it outlines the
evaluatory goals of our thesis and how they will be measured. It then discusses
in depth how we will measure the success of our implementation, through ERU,
QoS, and the sum of ERU and QoS. It then introduces reactive auto-scaling as the
predominant control group, and different traffic patterns and pod initialization
times as the predominant
independent variables. It highlights a representative application and pod
initialization time on which to perform our evaluation. It also describes the
considerable tooling created to make evaluation possible, and the process for
using said tools. Finally, it discusses our results, which shows scenarios in
which predictive auto-scaling is
beneficial compared to reactive auto-scaling, and extrapolates the real-world
impacts of our specific results.

