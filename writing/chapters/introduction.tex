Over the past few decades, an explosion in the need for computing resources,
and the existence of inexpensive, interconnected computers, has
driven a significant increase in the feasibility and benefits of distributed
systems \cite{distributed-systems-principles-and-paradigms}. Before progressing
to a discussion of this thesis' specific contributions to the field of distributed
systems, we present a broad overview of the field. We then conduct analysis of
greater and greater focus, until finally discussing the benefits and impacts of
predictive auto-scaling in Google's Kuberentes container cluster manager.

First, we consider the origin of distributed systems as a field of computer
science. Before the availability of cheap, powerful microprocessors and reliable,
efficient local-area networks (LANs), computational tasks could only be
performed on a singular computer
\cite{distributed-systems-principles-and-paradigms}. If a task was too
computationally expensive for a commodity PC, the only solution was to run
it on a larger, more powerful supercomputer. However, as cheap microprocessors
increased the availability of affordable computers, and LANs fostered quick inter-computer
communication, a new
model of performing resource intensive computation arose.
In this distributed systems model, a collection of individual
computers function as a single mass of computing resources
to solve a given computational task
\cite{distributed-systems-principles-and-paradigms}.

Second, we consider the ever-growing interest in unlocking and implementing the
benefits of distributed systems. A number of forces have driven
increased interest in distributed systems over
the past decade. The first, and most obvious, factor is the Internet and its
substantial impact on the role of computers in everyday life.
As more people connected to the Internet, through computers,
mobile phones, and tablets, an increasing number of interactions became
computerized. Consumption, communication, research, and more all
became possible on the Internet. Subsequently, large amounts of computing resources
were needed to store the data, and perform the computational tasks, related to these
interactions. Closely coupled with this trend is the rise of ``Big Data''.
In 2013, the digital universe contained 4.4 zettabytes of data
\cite{the-digital-universe-of-opportunities}.\footnote{A
  zettabyte equals $10^{21}$ bytes, which equals 1 billion
terabytes.} Without
multiple computers working together it would be impossible to store and process
this incredible volume of data.

Today, it is nearly impossible to do
anything in modern society without interacting with a distributed system and
creating new digital data. Driving a car, trading a stock, visiting a doctor,
checking an email, and even playing a simple video game, are all activities that
distributed systems facilitate and improve \cite{distributed-systems-concepts-and-design}.
As life becomes more
computerized, and as the volume of data humans generate and hope to process
grows, distributed systems will only increase in importance.
Furthermore, research into distributed systems makes it possible to
continue to unlock, and make available to the general public,
the incredible power of networked, cooperating computers. As the distributed systems
supplying massive computational power become more
accessible, because of decreased cost, increased ease of use and
improved reliability, we can
computationally address an ever-increasing number of challenging, important problems.

There are a number of different models for computing tasks requiring high levels
of computing resources, including supercomputing, cluster computing, and grid
computing. In this thesis, we focus on cluster computing. Cluster computing
groups together similar commodity PCs on the same LAN to offer a singular mass
of computing resources. Specifically, we focus on the
cluster manager, an integral component of cluster computing. Cluster managers
abstract all of the management details of the distinct
nodes in the cluster, and instead present a single collection of computing
resources on which the user can run jobs or applications. More succinctly,
a cluster manager ``admits, schedules, starts, restarts, and monitors the full
range of applications'' on the cluster \cite{borg}. Overall, a cluster
manager can be thought of like an operating system
for a cluster of networked computers. There are a
variety of different cluster managers, the most important of which will be
discussed in the next chapter, each pursuing different objectives. This
thesis will ultimately focus on Kubernetes, an open-source cluster
manager from Google \cite{k8s-website}.

Cluster managers seek to accomplish a number of different goals, and as a
result, multiple metrics measure success. For example, Microsoft's Autopilot is
predominantly concerned with application up-time, and thus measures success
with respect to reliability and downtime \cite{autopilot}.
Alternatively, a number of cluster managers measure themselves based on
efficient resource utilization (ERU) \cite{borg}. Essentially, efficient
resource utilization relates to the percent of cluster resources which are
actually being used by applications to perform computation/store data.
One such measurement of this metric, cluster
compaction, examines how many machines could be removed from the cluster, while
still comfortably running the cluster's current application load
\cite{evaluating-job-packing-in-warehouse-scale-computing}. This metric is
particularly important, because the more efficient the cluster manager is at
utilizing resources, the less the cluster costs, and the more accessible cluster
computing becomes to the general public. A final important
cluster management metric is quality of service (QoS). Quality of service measures the
ability of an application to function at a specified
performance level, despite ever-changing external factors.\footnote{In many instances, the most
prominent varying external factor is changes
in the load on the application. For example, for an application serving a
website, changes in the number of people requesting web pages from the website
would vary the application load.} Again, this metric is particularly important because
increasing the robustness of applications run on cluster managers means
these applications can be trusted with increasingly important tasks. Attempts to
maximize efficient resource utilization and quality of service
often lead to the cluster manager implementing
auto-scaling, a behavior we will examine in great depth throughout this thesis. Cluster
managers predominantly differ with respect to which metrics they optimize
for, and the process by which this optimization occurs.

\section{Goals}

This thesis is most concerned with maximizing the efficient resource utilization
(ERU) and quality of service (QOS) metrics with respect to the Kubernetes cluster manager.
As such, this thesis pursues three goals:

\begin{enumerate}
  \item Given an application running on a Kubernetes cluster, we seek to
    determine a method which ensures quality of
    service stays consistently high regardless of external factors. While it is
    difficult to make guarantees regarding quality of service, because
    application performance is dependent on a number of uncontrollable, varying
    external factors, it is possible to
    ensure each application has, and is utilizing, the resources it needs to
    function property. Given the cluster manager grants the application the
    resources it needs to function given the current external load,
    the cluster manager has done all it can to ensure a high quality of service.
  \item A simplistic solution to the first goal of ensuring a high application
    quality of service is to just give each application many more resources than
    it requests. Yet, this overallocation is inefficient and costly. Thus, our
    methods for ensuring a high quality of service must also ensure the
    maintenance, or improvement, of the efficient resource utilization metric.
    Thus, we add an additional goal: given a certain number of applications
    running on a Kubernetes cluster,
    we seek to determine a method which ensures the cluster is
    as small as possible, while still comfortably
    supporting the application's current, and future, resource needs.
  \item Given Kubernetes is an open-source project, we seek to implement, test, and
    evaluate a proposed enactment of the previous two goals.
    Thus, the methods we pursue will
    in part be dictated by the current structure and implementation of
    Kubernetes. Tests will be conducted using the Google Compute
    Engine\cite{google-compute-engine} on both
    simulated and real Kubernetes user data. The eventual goal is for this
    thesis' improvements to be merged into the production version of Kubernetes
    used to run 1000s of applications at Google everyday.
\end{enumerate}



\section{Contributions}

This thesis presents our given contributions to Kubernetes. Kubernetes seeks to
ensure high application quality of service and efficient resource utilization,
and our contributions look to further its ability to accomplish these
goals. As such, we present not only new methodology, but also new, working
implementations with the accompanying evaluation. We demonstrate the effectiveness
of our modifications in comparison to the non-modified Kubernetes using both
simulated and real-world datasets. Finally, we
discuss the experiences of making these modifications to Kubernetes, as well as
avenues for future improvements with respect to Kubernetes and cluster managers in
general.



\section{Contents}

@TODO - This section cannot be written until the thesis is completed.

