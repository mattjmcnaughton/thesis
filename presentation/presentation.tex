\documentclass{beamer}

\usetheme{CambridgeUS}

% Import this package for citations.
\usepackage[backend=bibtex]{biblatex}
\bibliography{presentation}

%%%%%%%%%%%%%%%%%%%%%%%% TITLE PAGE %%%%%%%%%%%%%%%%%%%%%%%%%%%%%%%%%

\title{Predictive Auto-scaling in the Kubernetes Cluster Manager}

\author{F.~Matt McNaughton\inst{1}, S. ~Jeannie Albrecht\inst{1}, \and
  T.~Brendan Burns\inst{2}}
% - Give the names in the same order as the appear in the paper.
% - Use the \inst{?} command only if the authors have different
%   affiliation.

\institute[Williams College] % (optional, but mostly needed)
{
  \inst{1}%
  Department of Computer Science\\
  Williams College
  \and
  \inst{2}%
  Lead Engineer for Kubernetes\\
  Google}

\date{Department Proposal Talk, 2016}

\subject{Distributed Systems}
% This is only inserted into the PDF information catalog. Can be left
% out.

% Delete this, if you do not want the table of contents to pop up at
% the beginning of each subsection:
\AtBeginSubsection[]
{
  \begin{frame}<beamer>{Outline}
    \tableofcontents[currentsection,currentsubsection]
  \end{frame}
}

% Let's get started
\begin{document}

% Display the title page based on the specified information above.
\begin{frame}
  \titlepage
\end{frame}

\begin{frame}[allowframebreaks]{Outline}
  \tableofcontents
  % You might wish to add the option [pausesections]
\end{frame}

\section{Goals}

\subsection{General}

\begin{frame}{General}
  Contribute to distributed system's ability to reliably and resourcefully
  perform large, varying amounts of computational work.

  % This presentation and thesis focus on work that is much too big to be
  % performed by any single commodity machine. For example, serving an extremely
  % popular website like "www.google.com".
\end{frame}

\subsection{Specific}

\begin{frame}{Specific}
  We seek to maximize the sum of two metrics: Efficient Resource Utilization and
  Quality of Service.
\end{frame}

\begin{frame}{Efficient Resource Utilization (ERU)}
  A measure of whether an application is efficiently using the resources it is
  given.

  % Resource can be anything - CPU, memory, network bandwidth\ldots The most
  % common, and the one we will investigate throughout this thesis, is CPU.

  % An example of poor resource utilization is running a webserver handling one
  % static file request per minute on a super
  % computer. It is probably using less than 1\% of the available CPU.
\end{frame}

\begin{frame}{Quality of Service (QOS)}
  A measure of whether the application is accomplishing its stated purpose.

  % Measurements of quality of service depend on the purpose of the application. For
  % example, if the application is a webserver, a measure of quality of service
  % might be the time it takes for the application to respond to a client's
  % request.

  % A critical assumption of this thesis is that quality of service is at least
  % in some ways impacted by access to a proper amount of computing resources.
  % This makes intuitive sense. Returning to the example of the webserver, if
  % the webserver is allocated 1\% of a commodity PC's CPU, but receives thousands of
  % requests per minute, it will have a horrid response time and thus a low
  % quality of service.
\end{frame}

\begin{frame}{Balancing ERU and QOS}
  Our goal is to maximize the summation of ERU and QOS. We want one of the
  following:

  \begin{itemize}
    \item ERU to increase and QOS to stay constant.
      % If ERU increases, but QOS stays constant, we are able to reduce the
      % amount of resources our application needs without harming performance.
    \item ERU to stay constant and QOS to increase.
      % If QOS increases, but ERU stays constant, we are able to increase the
      % performance of our application while maintaining the amount of resources
      % used.
    \item Both!
      % This is the dream\ldots Better performance and less resources used.
  \end{itemize}

  Accomplishing these goals can have substantial real world impacts.

  % If we improve ERU, then less resources are needed to run applications. Less
  % resources translates to lower costs. These savings can be substantial if
  % massive amounts of computation are being performed. If we improve quality of
  % service, we can begin to run more and more mission-critical applications.

  % Overall, the goal is to lower the costs and increase the reliability of
  % performing massive amounts of computation. Increase the groups of people who
  % can obtain large amounts of computing power and increase the problems to
  % which this mass of computing power can be applied.
\end{frame}

\section{Accomplishing General Goals: Cluster Managers and Kubernetes}

\subsection{Benefits of Cluster Managers}

\begin{frame}{Cluster Managers and their Benefits}
  Cluster managers abstract the notion of individual computers to present
  multiple, network connected computers as a single chunk of computing resources.

  Cluster duties include:

  \begin{itemize}
    \item Admitting/running/monitoring user submitted jobs.
    \item Allocating resources to jobs on the cluster.
  \end{itemize}

  % Think of a cluster manager like a very high-level operating system for
  % multiple computers. But instead of offering access to the resources of a
  % single computer, it offers resources to hundreds.
\end{frame}

\subsection{Overview of Cluster Managers}

\begin{frame}{Overview of Cluster Managers}

% @TODO Expand.
There are a variety of different cluster managers:

\begin{itemize}
  \item Borg
    % One of the first cluster managers from Google. Was not discussed publicly
    % until a couple of years ago, but has been around for years. Handles almost
    % all matters of running applications on the cluster.
  \item Mesos
    % If Borg is Ubuntu, then Mesos is the linux kernel. Handles low level
    % resources.
  \item Kubernetes
    % The cluster manager that we will study in this thesis. Open source cluster
    % manager designed to run applications on a cluster.
    % We are trying to accomplish our specific goals of improving the summation
    % of ERU and QOS on the Kubernetes cluster manager.
\end{itemize}

\end{frame}

\subsection{Kubernetes}

\begin{frame}{Details of Kubernetes}
  Cluster managers each have their own way of talking about running applications
  on the cluster\ldots Here are the most important terms:

  % @TODO Add an image.

  \begin{itemize}
    \item Pod
      % Pods contain the processes containing a single instance of the
      % application. For example, to place an application running a web server
      % with a key value store, we would run one containerized process running
      % Apache and one containerized process running Redis in the pod.
      % Application's are assumed to be written such that multiple pods running the same
      % application can run together to share the work. A pod can be destroyed
      % or created at any time (should be stateless).
    \item Replication Controller
      % Replication controller's control the amount of replication for a pod.
      % Essentially, if we tell our replication controller we want three pods to
      % be running, it will ensure that three pods always exist (create new ones
      % if pod dies).
    \item Service
      % Provides a single point of access for all of the pods associated with a
      % replication controller. Let's say we have a pod containing a process
      % running a web server. A replication controller specifies that there
      % should be five replicas of the pod. Now let's say someone wants to
      % communicate with the webserver. We don't want an external client to have
      % to know about how replication is occurring, so instead, we specify a
      % single consistent address - this is called a service.
  \end{itemize}
\end{frame}

\section{Auto-scaling}

\subsection{Benefits of Auto-scaling}

\subsection{Overview of Auto-scaling}

\subsection{Current State of Auto-scaling in Kubernetes}

\section{Predictive Auto-scaling in Kubernetes}

\subsection{Theoretical}

\subsection{Implementation}

\section{Status of Work}

\subsection{Current State}

\subsection{Future}

\begin{frame}{Citations}
  Check out the k8s website.\cite{k8s-website}.
\end{frame}

% Bibliography
\appendix
\section<presentation>*{\appendixname}
\subsection<presentation>*{For Further Reading}

\begin{frame}[allowframebreaks]
  \frametitle<presentation>{Citations}

  \printbibliography
\end{frame}

\end{document}


