\documentclass[letterpaper,11pt,twocolumn]{article}
\usepackage{graphicx,times}
\usepackage{cite} % Cite allows the use of bibtex.

\begin{document}

\title{Predictive Autoscaling of Pods in the Kubernetes Container Cluster
Manager}
\date{\today}

\author{
  {\rm Matthew McNaughton}\\
  Advisor Jeannie Albrecht\\
  Computer Science Thesis, Williams College
}

\maketitle

\thispagestyle{empty}

\section{Background and Motivation}
Society is becoming increasingly dependent on computation. Every facet of human
life, be it finance, healthcare, education, social interaction, etc\ldots, now
involves computers.\cite[pg. 4]{distributed-systems-concepts-and-design}
Naturally, singular computing workstations cannot provide the computational
power necessary to execute the demanded tasks at such a massive scale.\cite[pg.
2]{distributed-systems-principles-and-paradigms} However, cluster computing,
in which commodity level PCs linked by a
high-speed LAN offer a singular mass of resources,
can provide the requisite performance.\cite[pg.
17]{distributed-systems-principles-and-paradigms}
A cluster is used to run applications, and a cluster manager is responsible for
admitting, scheduling, starting, restarting, and monitoring the specified
applications on the cluster.\cite[pg. 1]{borg}. While cluster computing was largely
researched and implemented at large private corporations in the past, it is
increasingly becoming available to the general public. Kubernetes, an open
source cluster manager from Google, leads this effort.\cite{k8s-website}
Interest in, and the importance of, cluster computing will only increase as it
becomes more accessible to common programmer and more vital to processing the ever
growing massive of computing tasks.

Kubernetes manages all aspects of the cluster through admitting, running, and
restarting applications, monitoring and displaying application health, and maintaining the
underlying machines composing the cluster. Understanding the Kubernetes method
of cluster management requires understanding terms specific to this cluster
manager. Most importantly, Kubernetes defines the pod as the smallest deployable unit of
computing. A pod contains containers and data
volumes: a container is a single application coupled with the complete file system
necessary for execution, while a data volume is a persistent directory intended
to preserve data beyond the variable lifetime of the
container.\cite{docker-website} Containers, and the applications within them,
are intended to be started, stopped, deleted and replicated at will, without risking
the loss of important data, hence the need for persistent data volumes.
A pod can comprise of a collection of containers and
data volumes designed to perform a singular job, although typically a pod
includes a single container, and is responsible for performing the job the
application within the container dictates. Pods can be used to deploy web
servers, databases, data-processing frameworks, and more.\cite{k8s-pods} Additionally,
Kubernetes defines the concept of a replication controller, which simply ensures
a given number of a specified pod are running at all
times.\cite{k8s-replication-controllers} Replication
controllers are used in conjunction with services, which provide a long-lasting,
logical abstraction for a collection of identical temporary pods.
Services provide load balancing from any incoming request to the pods the
service abstracts.\cite{k8s-services} For example, a service would abstract
multiple pods all running the Apache web server by passing incoming requests to
a different singular pod each time. Understanding the concepts of pods,
replication controllers, and services is necessary for any work with Kubernetes.

Services have various performance requirements and face ever
changing external factors. For example, a service abstracting multiple pods
running a web server application
may be asked to process a highly variable number of requests. As requests spike,
an ever increasing amount of computation is spread across the pods. If the
replication controller defines a constant number of replication pods and the
load to the service increases, each pod will be asked to a use more resources.
If the constant replication value is not high enough, one or more pods may attempt to
compute outside of its capacities with respect to CPU or memory.
In contrast, if the constant replication value is too high,
and the load is balanced such that no pod operates anywhere
close to capacity, costly computing resources are wasted.
In many use cases, one or more pods operating below or beyond their capacity
is detrimental to the health and efficiency of the service and cluster.

Kubernetes solves the dual concerns mentioned above through varying the amount
of pod replication, a common method of scaling distributed systems.\cite[pg.
15]{distributed-systems-principles-and-paradigms} More specifically, Kubernetes
implements an optional behavior entitled horizontal pod autoscaling, which works
as follows. Consider a service $S$ balancing load to a set of identical pods $P =
\{p_{1} \cdots p_{n}\}$. For any pod $p_{i} \in P$, we define a target value for
resource consumption, signified as $target(p_{i})$. An example target resource
consumption value could be $80\%$ CPU usage or 100 mb of RAM. At a specified
interval, the autoscaler checks the resource usage across all pods, defined as
$curr(p_{i})$ for all $p_{i} \in P$, and adjusts the number of replicas so that
average resource usage conforms to target resource usage. Ultimately, given a
current set of pods $P_{c} = \{p_{1} \cdots p_{n}\}$, the target set
of pods is $P_{t} = \{p_{1} \cdots p_{m}\}$, where $m = \sum_{i =
0}^{n} curr(p_{i}) / target(p_{i})$. For example, if there currently exists
$P = \{p_{1}\}$ such that $curr(p_{1}) = 100\%$ CPU usage, while
$target(p_{i}) = 50\%$ CPU usage, the autoscaler would replicate
$p_{1}$. With the existence of this replica $p_{2}$, such that $P =
\{p_{1}, p_{2}\}$, we are now ensured $curr(p_{1}) = curr(p_{2}) =
target(p_{i})$. The exact same process works for scaling down if the current CPU
usage of each pod is too lower compared to the target resource
usage.\cite{k8s-horizontal-pod-autoscaler-proposal}

While horizontal pod autoscaling is useful in increasing Kubernetes' ability to
handle varying loads, autoscaling could be more responsive. Specifically, the
current method of autoscaling fails to consider the amount of time necessary to
create a pod, which we will call
$create(p_{i})$.\cite{brendan-burns-conversation} $create(p_{i})$ is a function
of the complexity and build time for the containers within the pod $p_{i}$. For
example, the create time for pod comprised of a container running
a simple C program may be only a couple
of seconds, while the create time for a pod comprised of multiple containers each
utilizing a complex Java library may be minutes or in extreme cases hours. To
illustrate the necessity of considering pod creation time, consider the
following example.
To begin, consider again a service $S$ balancing load to a set of identical pods
$P = \{p_{1}\}$, such that $create(p_{i}) = 3000s \mbox{ and } target(p_{i}) =
50\%$ CPU usage. Imagine at $t = 0s, curr(p_{1}) = 100\%$ CPU usage. Naturally
$p_{1}$ operating above target resource consumption suggests replicating
$p_{1}$ so that $P = \{p_{1}, p_{2}\}$ and $curr(p_{1}) = curr(p_{2}) =
target(p_{i})$. Yet because $create(p_{i}) = 3000s$, it is not until $t = 3000s$
that the replication is complete, load can be balanced across $P =
\{p_{1}, p_{2}\}$, and resource consumption returns to the target. In certain
contexts, particularly those in which operating outside of target resource
consumption is damaging or the pod's creation time is non-trivial, this delay is
extremely detrimental. Clearly, work can still be done to increase Kubernetes'
scaling responsiveness.

\section{Question}

In this thesis, we will seek to address the deficiencies previewed in the
motivation section. Specifically, we will answer the following question:
How can we decrease the amount of time pods spend operating outside of target
resource consumption?

\section{Hypothesis}

We hypothesize it is possible to decrease the amount of time pods spend
operating outside of target resource consumption through predictive autoscaling.
Predictive autoscaling attempts to scale based on the pods' future, not present, resource
consumption, so that a pod required in the future can begin
the creation process before any pod operates outside the target value,
and be functional as soon as it is needed.
More formally, consider again the example given at the end of the previous
section.
This time, imagine that instead of beginning the replication process at
$t = 0s$, when $curr(p_{1}) = 100\%$ CPU usage, we instead predicted at
$t = -3000s$ that at $t = 0s, curr(p_{1}) = 100\%$ CPU usage, and we began to
create $p_{2}$. After the $3000s$ necessary for the creation of $p_{2}$, both
pods would be available, and $curr(p_{1}) = curr(p_{2}) = target(p_{i})$. As
opposed to if we waited until $t = 0s$ to begin creating $p_{2}$, we do not have
to spend any time operating outside of target resource consumption. Predictive
scaling increases the responsiveness of autoscaling, by decreasing the impact of
pod creation times, and thereby minimizing the amount of time any pod must
function beyond target resource consumption.

\section{Implementation and Methodology}

Kubernetes is a rare and interesting project in that it is both completely open
source and in production use at Google.\cite{google-container-engine}
Thus, we will be implementing our work
with predictive autoscaling on the actual Kubernetes code, with the eventual
goal of our work becoming part of the production Kubernetes distribution.

Considering our goal of merging this code into a system as heavy-utilized as
Kubernetes, it is important to show tangible successes or failures through
evaluation. Fortunately, a fairly defined methodology exists
for evaluating cluster managers like
Kubernetes.\cite[pg. 355]{omega} The first aspect of evaluation utilizes a
lightweight simulator based on the statistical structure of typical input to
Kubernetes. For example, testing the ability of Kubernetes implementing
predictive autoscaling to respond to a spike in web traffic,
would require simulating the highly variable incoming web requests through statistical
modelling. The second evaluation replays historic data, in order to make exact
conclusions about how Kubernetes with predictive autoscaling would have operated in a
specific situation. This analysis is typically more restrictive, because it can
be difficult, and time-consuming, to obtain and replay exact data.

Evaluation will involve tracking the internal metrics Kubernetes already
records, including the amount of time a pod is running, and the specific
computational metrics for the pod (ie, CPU, memory\dots). Work will be
necessary to decide which metrics are best suited to measuring scaling responsiveness.

\section{Current State}

Preliminary work with Kubernetes has been extremely encouraging. Most importantly,
the lead developer of Kubernetes, Brendan Burns '98, is a Williams graduate who has
been very generous with his time and expertise as we have discussed potential
areas of exploration. I am hopeful that I will be able to continue
working with Brendan going forward.

Naturally, Kubernetes is a large open source project, and as such there are
certain standards and customs which must be followed. Based on past open source
work and Kubernetes extensive documentation, I have already made small
contributions to Kubernetes, and am hopeful about the possibility of making
larger contributions, including the predictive scaling examined in this thesis.

\bibliography{proposal}
\bibliographystyle{plain}

\end{document}
